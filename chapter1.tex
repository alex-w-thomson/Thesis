% +--------------------------------------------------------------------+
% | Sample Chapter 1
% |
% | This file provides examples of how to
% | - insert a figure with a caption
% | - construct a table with a caption
% | - create subsections within the chapter
% | - insert a reference to a Figure or Table
% | - make a citation
% +--------------------------------------------------------------------+

\cleardoublepage

% +--------------------------------------------------------------------+
% | Replace "Chapter Title" below with the title of your chapter.
% | LaTeX will automatically number the chapters.
% +--------------------------------------------------------------------+

\chapter{Introduction to Leibniz-type rules}
\label{makereference1}

Leibniz-type rules have been extensively studied due to their connections to partial differential equations that model many real world situations such as shallow water waves and fluid flow. In this chapter we introduce some of the definitions and history of the development of Leibniz-type rules that motivated the results to be discussed in Chapters \ref{chapter2} and \ref{chapter3} of this manuscript. First consider the Leibniz rule taught in Calculus courses, which expresses the derivatives of a product of functions as a linear combination of derivatives of the functions involved; more specifically, for functions $f$ and $g$ sufficiently smooth, it holds that
\[\partial^\alpha (fg)(x) = \sum_{\beta \leq \alpha} \binom{\alpha}{\beta} \partial^{\alpha - \beta} f(x) \partial^{\beta} g(x) = \partial^\alpha f(x) g(x) + f(x) \partial^\alpha g(x) + \cdots ,\]
for $\alpha,\beta \in \mathbb{N}^n_0$.
In an analogous way, fractional Leibniz rules give estimates of the smoothness and size of a product of functions in terms of the smoothness and size of the factors. For instance, for $f$ and $g$ in the Schwartz class $\mathcal{S}(\mathbb{R}^n)$, it holds that
\begin{equation}\label{def:leibniz}
\norm{D^s (fg)}{L^r} \lesssim \norm{D^s f}{L^{p_1}}\norm{g}{L^{\tilde{p}_1}} + \norm{f}{L^{p_2}}\norm{D^s g}{L^{\tilde{p}_2}},
\end{equation}
where $1/p = 1/p_1 + 1/\tilde{p}_1 = 1/p_2 + 1/\tilde{p}_2$, $1<p_1,p_2,\tilde{p}_1,\tilde{p}_2\leq \infty$, $1/2 <p\leq \infty$, and $s>n(1/\text{min}(r,1) - 1)$ or $s$ is an even whole number. The homogeneous fractional differentiation operator of order $s$, $D^s$, is defined as \[D^s f(x) = \int_\rn |\xi|^s \widehat{f}(\xi) e^{2\pi i x\cdot \xi} d\xi,\]
where $\widehat{f}$ is the Fourier transform of $f$.
For $s>0$, the operator $D^s$ is naturally understood as taking $s$ derivatives of its argument. Indeed, in the case $s=2$, $D^2f = c\Delta f$, where $\Delta = \sum_{j=1}^n \partial^2_{x_j}$ is the Laplacian operator. Furthermore, if $s$ is a positive integer,

%the $\dot{W}^{k,p}$ norm and $\norm{D^k \cdot}{L^p}$ are equivalent norms where 
%\[\norm{f}{\cdot{W}^{k,p}} = \sum_{|\alpha|=k} \norm{\partial^\alpha}{L^p}.\] 

\[\norm{D^s f}{L^p} \sim \sum_{|\alpha| = s} \norm{\partial^\alpha f}{L^p},\]
where $|\alpha| = \alpha_1 + \alpha_2 + ... + \alpha_n$ for $\alpha = (\alpha_1,\alpha_2,...,\alpha_n) \in \mathbb{N}^n_0.$

 Another version of (\ref{def:leibniz}) is obtained by using the inhomogeneous sth order fractional differentiation operator $J^s$:
\begin{equation}\label{def:ileibniz}
\norm{J^s(fg)}{L^r} \lesssim \norm{J^s f}{L^{p_1}}\norm{g}{L^{\tilde{p}_1}} + \norm{f}{L^{p_2}}\norm{J^s g}{L^{\tilde{p}_2}}.
\end{equation}
Similarly to its homeogenous counterpart, the operator $J^s$ is defined through the Fourier transform as 
\[ J^s f(x) = \int_{\rn} (1+|\xi|^2)^\frac{s}{2} \widehat{f}(\xi)e^{2\pi i x\cdot\xi} d\xi\]
and can be interpreted as taking derivatives up to order $s$ of $f$.
%Then $s>0$ it is interpreted as taking up to $s$ derivatives of the function it is applied to as when $s=k$ is a positive integer $\norm{J^k(f)}{L^p} = \norm{f}{W^{k,p}}$ where $\norm{f}{W^{k,p}} = \sum_{|\alpha|\leq k} \norm{\partial^\alpha f}{L^p}$.
 
The estimates (\ref{def:leibniz}) and (\ref{def:ileibniz}) are also known as Kato-Ponce inequalities due to the foundational work of Kato-Ponce \cite{MR951744}, where the estimate (\ref{def:ileibniz}) was proved in the case $1<r=p_1=\tilde{p}_2<\infty$ and $p_2=\tilde{p}_1=\infty$, with applications to the Cauchy problem for Euler and Navier-Stokes equations. This result was extended by Gulisashvili-Kon \cite{MR1420922}, who showed (\ref{def:leibniz}) and (\ref{def:ileibniz}) for the cases $s>0$, $1<r<\infty$, and $1 < p_1,p_2,\tilde{p}_1,\tilde{p}_2\leq\infty$ in connection to smoothing properties of Schr\"odinger semigroups. Grafakos-Oh \cite{MR3200091} and Muscalu-Schlag \cite{MR3052499} established the cases for $1/2 <r\leq 1$ and the case $r=\infty$ was completed in the work of Bourgain-Li \cite{MR3263081} and Grafakos-Maldonado-Naibo \cite{MR3189525}. Applications of the estimates (\ref{def:leibniz}) and (\ref{def:ileibniz}) to Korteweg-de Vries equations were studied by Christ-Weinstein \cite{MR1124294} and Kenig-Ponce-Vega \cite{MR1211741}.

In the estimates (\ref{def:leibniz}) and (\ref{def:ileibniz}) the two functions $f$ and $g$ are related through pointwise multiplication. Throughout the rest of this manuscript we will consider bilinear estimates in the spirit of (\ref{def:leibniz}) and (\ref{def:ileibniz}) where the two functions are related through a pseudodifferential operator. Let $\sigma(x,\xi,\eta)$ be a complex-valued, smooth function for $x,\xi,\eta\in\rn$. We define the \textit{bilinear pseudodifferential operator}  associated to $\sigma$, $T_\sigma$, by 
\begin{equation}\label{psydo}
T_{\sigma}(f,g)(x) = \int_{\mathbb{R}^{2n}} \sigma(x,\xi,\eta) \widehat{f}(\xi)\widehat{g}(\eta)e^{2\pi i x\cdot(\xi+\eta)}d\xi d\eta. 
\end{equation}

We call $\sigma$ as the \textit{symbol} of the operator $T_\sigma$, when $\sigma$ is independent of $x$, $\sigma$ is also referred to as the \textit{multiplier} of the \textit{bilinear multiplier operator} $T_\sigma$. We note that $\sigma \equiv 1$ gives $T_\sigma(f,g) = fg$. 


Throughout this manuscript we will study Leibniz-type rules associated to bilinear pseudodifferential operators that are of the form 
\begin{equation}\label{h_general_estimate}
\norm{D^s T_\sigma(f,g)}{Z} \lesssim \norm{D^sf}{X_1}\norm{g}{Y_1} + \norm{f}{X_2}\norm{D^s g}{Y_2} ,
\end{equation}
\begin{equation}\label{i_general_estimate}
\norm{J^sT_\sigma(f,g)}{Z} \lesssim \norm{J^sf}{X_1}\norm{g}{Y_1} + \norm{f}{X_2}\norm{J^sg}{Y_2} ,
\end{equation}
for a variety of function spaces $X_1$, $X_2$, $Y_1$, $Y_2$, and $Z$.

%When $\sigma \equiv 1$, the Fourier inversion formula implies that $T_\sigma(f,g) = fg$. Therefore, estimates of the form (\ref{h_general_estimate}) and (\ref{i_general_estimate}) include, as particular cases, the estimates (\ref{def:leibniz}) and (\ref{def:ileibniz}) presented in Section 1 by choosing $Z$, $X_1$, $Y_1$, $X_2$, and $Y_2$ to be appropriate Lebesgue spaces.


In the particular case that $\sigma \equiv 1$ and $X_1$, $X_2$, $Y_1$, $Y_2$, $Z$ are appropriate Lebesgue spaces (\ref{h_general_estimate}) and (\ref{i_general_estimate}) recover (\ref{def:leibniz}) and (\ref{def:leibniz}) respectively.

In Chapter \ref{chapter2} we will present Leibniz-type rules (\ref{h_general_estimate}) and (\ref{i_general_estimate}) in the setting of Besov and Triebel-Lizorkin spaces based on certain quasi-Banach spaces. Such bilinear estimates will be proved for bilinear Coifman-Meyer multiplier operators. A particular case of the results in Chapter \ref{chapter2} is the following fractional Leibniz rule, and its inhomogeneous counterpart, in the context of Hardy spaces:

\begin{equation}\label{KP:Hardy}
\norm{D^s(fg)}{H^p(w)} \lesssim \norm{D^s f}{H^{p_1}(w_1)} \norm{g}{H^{\tilde{p}_1}(w_2)} +  \norm{f}{H^{p_2}(w_1)}   \norm{D^s g}{H^{\tilde{p}_2}(w_2)} 
\end{equation}
where $0<\p,p_1,\tilde{p}_1, p_2, \tilde{p}_2 <\infty$, $\frac{1}{p} = \frac{1}{p_1} + \frac{1}{\tilde{p}_1} = \frac{1}{p_2} + \frac{1}{\tilde{p}_2}$. Recalling that $H^p = L^p$ for $1<p<\infty$, (\ref{KP:Hardy}) extends and improves (\ref{def:leibniz}). Indeed, 

The techniques used are quite flexible and allow the method of proof to be adapted to many different function spaces. One of the main results in this chapter is as follows.

%\begin{theorem}\label{thm:CM:TL:B}  For $m \in \re,$ let $\sigma(\xi,\eta),$ $\xi,\eta\in\rn,$ be a Coifman-Meyer multiplier of order $m.$ Consider  $0 < p, p_1, p_2  \le \infty$  such that $\hcline$ and  $0 < q \leq \infty;$ let  $w_1,w_2\in A_\infty$ and set $w=w_1^{{p}/{p_1}} w_2^{{p}/{p_2}}.$ 
%If $0 < p,p_1,p_2 < \infty$ and  $s > \tau_{p,q}(w),$  it holds that
%\begin{equation}\label{KP:CM:TL}
%\norm{T_\sigma(f,g)}{\tlw{p}{s}{q}{w}} \lesssim \norm{f}{\tlw{p_1}{s+m}{q}{w_1} } \norm{g}{H^{p_2}(w_2)} +  \norm{f}{H^{p_1}(w_1)}   \norm{g}{\tlw{p_2}{s+m}{q}{w_2} } \quad \forall f, g \in \swz.
%\end{equation}
%If $0< p, p_1,p_2\leq \infty$ and $s > \tau_p(w)$, it holds that
%\begin{equation}\label{KP:CM:B}
%\norm{T_\sigma(f,g)}{\besw{p}{s}{q}{w}} \lesssim \norm{f}{\besw{p_1}{s+m}{q}{w_1} } \norm{g}{H^{p_2}(w_2)} +  \norm{f}{H^{p_1}(w_1)}   \norm{g}{\besw{p_2}{s+m}{q}{w_2} } \quad \forall f, g \in \swz,
%\end{equation}
%where the Hardy spaces $H^{p_1}(w_1)$ and $H^{p_2}(w_2)$ must be replaced by $L^\infty$ if $p_1=\infty$ or $p_2=\infty,$ respectively.
%
%If $w_1=w_2$ then different pairs of $p_1, p_2$ can be used on the right-hand sides of \eqref{KP:CM:TL} and \eqref{KP:CM:B}; moreover, if $w\in A_\infty,$ then 
%\begin{equation}\label{KP:CM:TL2}
%\norm{T_\sigma(f,g)}{\tlw{p}{s}{q}{w}} \lesssim \norm{f}{\tlw{p}{s+m}{q}{w} } \norm{g}{L^\infty} +  \norm{f}{L^\infty}   \norm{g}{\tlw{p}{s+m}{q}{w}} \quad \forall f, g \in \swz,
%\end{equation}
%where $0<p<\infty,$ $0<q\le\infty$ and $s>\tau_{p,q}(w).$
%\end{theorem}

%By the equivalences 
%\begin{equation}\label{lifting}
%\norm{T_\sigma(f,g)}{\besw{p}{s}{q}{w}} \sim \norm{D^s T_\sigma(f,g)}{\besw{p}{0}{q}{w}} \text{ and } \norm{T_\sigma(f,g)}{\tlw{p}{s}{q}{w}} \sim \norm{D^s T_\sigma(f,g)}{\tlw{p}{0}{q}{w}}
%\end{equation}
%and using estimates \ref{KP:CM:TL} we obtain estimates of the form \ref{h_general_estimate}. In particular, in the case that $\sigma \equiv 1$ we obtain results that both extend and improve the estimates \ref{def:leibniz} and \ref{def:ileibniz}. This is because in this case it holds that $\norm{D^s (fg)}{\tlw{p}{0}{2}{w}} \sim \norm{D^s fg)}{H^p(w)}$ and using this we obtain the following corollary.
%
%\begin{corollary}\label{coro:KP:Hardy} 
%Consider  $0 < p, p_1, p_2  < \infty$  such that $\hcline;$ let  $w_1,w_2\in A_\infty$ and set $w=w_1^{{p}/{p_1}} w_2^{{p}/{p_2}}.$ 
%If  $s > \tau_{p}(w),$ it holds that
%\begin{equation}\label{KP:Hardy}
%\norm{D^s(fg)}{H^p(w)} \lesssim \norm{D^s f}{H^{p_1}(w_1)} \norm{g}{H^{p_2}(w_2)} +  \norm{f}{H^{p_1}(w_1)}   \norm{D^s g}{H^{p_2}(w_2)} \quad \forall f, g \in \swz.
%\end{equation}
%If $w_1=w_2$ then different pairs of $p_1, p_2$ can be used on the right-hand side of \eqref{KP:Hardy}; moreover, if $w\in A_\infty,$ then 
%\begin{equation*}
%\norm{D^s(fg)}{H^p(w)} \lesssim \norm{D^s f}{H^{p}(w)} \norm{g}{L^\infty} +  \norm{f}{L^\infty}   \norm{D^s g}{H^{p}(w)} \quad \forall f, g \in \swz,
%\end{equation*}
%where $0<p<\infty$ and $s>\tau_{p}(w).$
%\end{corollary}

 Indeed, the inequality \eqref{KP:Hardy} extends the range of $p$, $p_1$, $\tilde{p}_1$, $p_2$, and $\tilde{p}_2$ by allowing $0<p,p_1,\tilde{p}_1,p_2,\tilde{p}_2<\infty$ while \eqref{def:leibniz} requires that $1<p_1,p_2<\infty$. Additionally (\ref{KP:Hardy}) allows for the $H^p$ norm on the left-hand side which is generally larger than $||\cdot||_{L^p}$. 
  
The techniques used in the proofs of the results in Chapter \ref{chapter2} are quite flexible and allow \eqref{h_general_estimate} and \eqref{i_general_estimate} in Triebel-Lizorkin and Besov spaces based in weighted Lebesque spaces, weighted Lorrentz spaces, weighted Morrey spaces, and variable-exponent Lebesgue spaces.
In particular, the proofs make use of Nikol'skij representations of such function spaces. These representations have been used in unweighted settings such as the work of Nikol'skij \citep{MR0374877}, Meyer \citep{MR639462}, Bourdad \citep{MR673825}, Triebel \citep{MR3024598}, and Yamazaki \citep{MR837335}. 

As an application of the results in Chapter \ref{chapter2} we obtain scattering properties to certain systems of partial differential equations which involve fractional powers of the Laplacian. Solutions of these systems scatter to a solution which can be viewed as the action of a bilinear pseudodifferential operator on appropriate arguments to which the main results of this chapter can be applied. 

In Chapter 3 we present Leibniz-type rules in Besov and local Hardy spaces for bilinear pseudodifferential operators associated to symbols in bilinear H\"ormander classes of critical order. For such symbols we prove bilinear estimates of the form 

\begin{equation}
\norm{D^s T_\sigma(f,g)}{B^0_{p,q}} \lesssim \norm{D^s f}{B^0_{p_1,q}} \norm{g}{h^{p_2}} + \norm{f}{h^{p_1}}\norm{D^s g}{B^-_{p_2,q}}
\end{equation}
where $B^0_{p,q}$ and $h^p$ denote Besov and Hardy spaces respectively, and $0<p<\infty$, $0<p_1,p_2\leq\infty$ are such that $1/p = 1/p_1 + 1/p_2$, $0<q\leq\infty$, and $s>\text{max}\{0,n(1/p - 1)\}$.

%\begin{theorem}\label{thm:critical_besov}
%Let $0<p<\infty$ and $0<p_1,p_2\leq\infty$ be such that $1/p = 1/p_1 + 1/p_2$, $0<q\leq\infty$, $s>\text{max}\{0,n(1/p - 1)\}$, and $0\leq \delta\leq \rho <1$. If $\sigma\in BS^{m(\rho,p_1,p_2)}_{\rho,\delta}$ then it holds that 
%\begin{equation}\label{critical_est}
%\norm{T_\sigma(f,g)}{B^s_{p,q}} \lesssim \norm{f}{B^s_{p_1,q}} \norm{g}{h^{p_2}} + \norm{f}{h^{p_1}}\norm{g}{B^s_{p_2,q}} \quad \forall f,g\in\mathcal{S}(\rn),
%\end{equation}
%where $h^{p_1}$ and $h^{p_2}$ must be replaced by $L^\infty$ if $p_1 = \infty$ or $p_2 = \infty$ respectively. 
%\end{theorem}

Boundedness properties of bilinear pseudodifferential operators with symbols
in the bilinear H\"ormander classes have been extensively studied in the settings of
Lebesgue and Hardy spaces; see B\'enyi-Bernicot-Maldonado-Naibo-Torres \citep{MR2660466}, B\'enyi-Chaffee-Naibo \citep{benyi2018strongly}, B\'enyi-Maldonado-Naibo-Torres \citep{MR1986065}, MR2660466}, Brummer-Naibo \citep{MR3750234}, Herbert-Naibo \citep{MR3627725}, \citep{MR3211086}], Koezuka-Tomita \citep{MR3750316}, Michalowski-Rule-Staubach \citep{MR3165300}, Miyachi-Tomita
\citep{MR3179688, miyachi2018bilinear, miyachi2018bilinear2}, Naibo \cite{MR3393696, MR3411149}, Rodr\'iguez-L\'opez-Staubach \citep{MR3035059}, and the references therein.

The proofs of the results in Chapter \ref{chapter3} utilize Nikol'skij representations for Besov spaces and appropriate spectral decompositions of the symbol $\sigma$. The techniques used are inspired by bilinear techniques used in Naibo \cite{MR3393696} and techniques for linear operators in Johnsen \citep{MR2163627}, Marschall \citep{MR1376592}, and Park \citep{Park}. Results related to estimate \ref{critical_est} were proved for the forbidden class $BS^0_{1,1}$ in Koezuka-Timita \cite{MR3750316} and Naibo \citep{MR3393696}. Concerning bilinear pseudodifferential operators with symbols belonging to the subcritical classes $BS^m_{\rho,\delta}$ with $m<m(\rho,p_1,p_2)$ and $1\leq p_1,p_2,p \leq \infty$ ($1 < p_1,p_2,p < \infty$ when $\rho = \delta = 0$) estimate \ref{critical_est} was shown in Naibo \citep{MR3393696} Theorem 1.3. Theorem \ref{thm:critical_besov} extends this result to the critical classes and allows for the indices to be in the wider range $(0,\infty)$. 

We close this chapter by referencing serveral works in connection with the study of the bilinear estimates \eqref{h_general_estimate} and \eqref{i_general_estimate}. In \cite{MR3750234}, Brummer-Naibo studied Leibniz-type rules in function spaces that admit a molecular decomposition and a $\varphi$-transform characterization in the sense of Frazier-Jawerth \cite{MR808825, MR1070037}. In the context of Lebesgue spaces and mixed Lebesgue spaces, estimates of the type (\ref{h_general_estimate}) were studied in Hart-Torres-Wu \cite{HTW} for bilinear multiplier operators with minimal smoothness assumptions on the multipliers. Related mapping properties for bilinear pseudodifferential operators with symbols in the bilinear H\"ormander classes were studied by B\'enyi-Torres \cite{MR1986065} and B\'enyi-Nahmod-Torres \cite{MR2250054} in the setting of Sobolev spaces, by B\'enyi \cite{MR1996120} in the setting of Besov spaces, and by Naibo \cite{MR3393696} and Koezuka-Tomita \cite{MR3750316} in the context of Triebel-Lizorkin spaces. Additionally, versions of (\ref{def:leibniz}) and (\ref{def:ileibniz}) in weighted Lebesgue spaces were proved in Cruz-Uribe-Naibo \cite{MR3513582}, while Brummer-Naibo \cite{BrNa2017} proved (\ref{h_general_estimate}) and (\ref{i_general_estimate}) in weighted Lebesgue spaces for Coifman-Meyer multiplier operators. 


%\begin{figure}[htb]%t=top, b=bottom, h=here
%
%    \includegraphics[height=2.5in]{figures/graph.png}
%
%    \caption[Optional: Short caption to appear in List of
%    Figures]{Full caption to appear below the Figure}
%
%    \label{figure1}
%\end{figure}

% +--------------------------------------------------------------------+
% |To create cross-references to figures, tables and segments
% |of text, LaTeX provides the following commands:
% |   \label{marker}
% |   \ref{marker}
% |   \pageref{marker}
% | where {marker} is a unique identifier.
% |
% | In the line above, we use \label{figure1} to mark a location
% | we wish to refer to later.  LATEX replaces \ref by the number of
% | the chapter, section, subsection, figure, or table after which the
% | corresponding \label command was issued. \pageref prints the page
% | number of the page where the \label command occurred.
% |
% +--------------------------------------------------------------------+

%See the file chapter1.tex for examples of the commands used to
%insert a figure or table, add a caption, etc.  Here is an example of
%a table:

%\begin{table}[ht]
%
%% +--------------------------------------------------------------------+
%% | We include the command \begin{center} to center the table
%% | horizontally on the page.  Note use of the command \end{center}
%% | to turn off centering after the table is defined.
%% +--------------------------------------------------------------------+
%    \begin{center}
%
%% +--------------------------------------------------------------------+
%% | The table is created with this command
%% |
%% | \begin{tabular}[pos]{table spec}
%% |
%% | The "pos" argument specifies the vertical position of the table
%% | relative to the baseline of the surrounding text.  Use t, b, or c
%% | to specify alignment at the top, bottom, or center.
%% |
%% | The "table spec" command defines the format of the table
%% |   l for a column of left-aligned text
%% |   r for a column of right-aligned text
%% |   c for centered text
%% |   p{width} for a column containing justified text with line breaks
%% |   | for a vertical line
%% |
%% |  In this example, the caption is made to appear above the table
%% |  by positioning the \caption command before the \begin{tabular
%% |  command. To position the caption below the table, insert the
%% |  \caption command after the \end{tabular} command.
%% +--------------------------------------------------------------------+
%    \caption{Caption to appear above the table}
%    \begin{tabular}[c]{|c|c|c|}
%        \hline
%        Column 1 Heading & Column 2 Heading & Column 3 Heading \\
%        \hline
%        Col 1 Row 1 & Col 2 Row 1 & Col 3 Row 1\\
%        Col 1 Row 2 & Col 2 Row 2 & Col 3 Row 2\\
%        Col 1 Row 3 & Col 2 Row 3 & Col 3 Row 3\\
%        \hline
%    \end{tabular}
%
%    \label{table1}
%   \end{center}
%\end{table}



% +--------------------------------------------------------------------+
% | Replace \section headings below with the title of your
% | subsections.  LaTeX will automatically number the subsections 1.1,
% | 1.2, 1.3, etc.
% +--------------------------------------------------------------------+

%\section{Making References to Figures or Tables}
%\label{makereference1.1}
%
%It is possible to create cross-references and hyperlinks to items or
%sections within your paper.  For example, here is a reference to
%Fig.~\ref{figure1} mentioned at the beginning of this chapter and a
%reference to the Table~\ref{table1}.
%
%\section{Making a Reference to a Chapter Subsection}
%\label{makereference1.2}
%
%In this section, we refer back to text mentioned in
%Section~\ref{makereference1.1} on page~\pageref{makereference1.1}.
%
%\section{Making a Citation}
%\label{makereference1.3}
%
%Here's an example of a citation to a single
%work.~\citep{CT:Weiner:1999} It's also possible to make multiple
%citations.~\citep{CT:Phillips:1985, ARP:Loy:1974}
%
%This template uses BibTeX to manage and format citations.  BibTeX is
%not the only way to create a bibliography within LaTeX, but it's
%generally considered to be the best option for long documents like a
%thesis or dissertation.~\citep{CT:Gould:1988}  There are a few more
%sample citations in this paragraph so you can see examples of how
%in-text references are made and how the bibliography is
%formatted.~\citep{ARP:Melinger:1991} See the file "BibTeX Guide.pdf"
%for information on how to use BibTeX.
