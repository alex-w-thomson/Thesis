% +--------------------------------------------------------------------+
% | Sample Chapter 3
% +--------------------------------------------------------------------+

\cleardoublepage

% +--------------------------------------------------------------------+
% | Replace "This is Chapter 3" below with the title of your chapter.
% | LaTeX will automatically number the chapters.                      
% +--------------------------------------------------------------------+

\chapter{Bilinear H\"ormander Classes of Critical Order}\label{chapter3}
\label{makereference3}

\section{Introduction}

In this chapter we obtain Leibniz-type rules for bilinear multiplier operators associated to symbols in the H\"ormander classes of critical order in the setting of local Hardy spaces. First, we will discuss the bilinear H\"ormander classes $BS^m_{\rho,\delta}$ and what it means for these symbols to be of critical order. Given $0\leq \delta \leq \rho \leq 1$ and $m\in\re$, a complex-valued function $\sigma = \sigma(x,\xi,\eta)$, $x,\xi,\eta \in \rn$, belongs to the bilinear H\"ormander class $BS^m_{\rho,\delta}$ if for any multiindices $\alpha,\beta,\gamma \in \mathbb{N}^n_0$ there exists a positive constant $C_{\alpha,\beta,\gamma}$ such that 
\begin{equation}\label{def:Bmrd}
|\partial_x^\alpha \partial_\xi^\beta \partial_\eta^\gamma \sigma(x, \xi, \eta)| \leq C_{\alpha, \beta, \gamma} (1+|\xi|+|\eta|)^{m +\delta \abs{\alpha}-\rho(\abs{\beta+\gamma})} \quad \forall x, \xi, \eta \in \rn.
\end{equation}
Then for any $\sigma \in BS^m_{\rho,\delta},$ the bilinear pseudodifferential operator $T_\sigma$ associated to $\sigma$ is defined as in \ref{psydo}. There has been a considerable amount of effort devoted to studying bilinear pseudodifferential operators associated to symbols in the classes $BS^m_{\rho,\delta}$. One fundamental aspect of the study of such symbols is their symbolic calculus for the transposes of operators associated to them. This was established in the works B\'enyi-Torres \citep{MR1986065} and B\'enyi-Maldonado-Naibo-Torres \citep{MR2660466}. Another important aspect of the study of these operators is their boundedness properties in a variety of function spaces. Operators associated to symbols in $BS^0_{1,0}$ can be realized as Calder\'on-Zygmund operators. As a consequence operators associated to symbols in $BS^0_{1,0}$ are bounded from $L^{p_1}(\rn) \times L^{p_2}(\rn)$ to  $L^p(\rn)$ for  $1 < p_1, p_2 < \infty$ and $1/2<p <\infty$ related through $\hcline.$ These operators also satisfy the endpoint mappings $L^\infty(\rn) \times L^\infty(\rn) \rightarrow BMO(\rn)$ and $L^1(\rn) \times L^1(\rn) \rightarrow L^{1/2, \infty}(\rn)$. For the development of the Calder\'on-Zygmund theory see Coifman-Meyer \citep{MR518170}, Kenig-Stein \citep{MR1713146}, and Grafakos-Torres \citep{MR1880324}. Operators with symbols in the forbidden class $BS^0_{1,1}$ may fail to be bounded in Lebesgue spaces. As such operators associated to these symbols are better understood in other settings. In B\'nyi et al. ~\cite{MR1996120, MR2250054, MR1986065} estimates in Sobolev spaces were obtained for such operators. For results in the settings of Besov and Triebel-Lizorkin spaces see Brummer--Naibo~\cite{MR3750234}, Koezuka--Tomita~\cite{MR3750316} and Naibo~\cite{MR3393696}.


Given  $0\le \delta\le \rho< 1$  and $0< p_1,p_2,p\le \infty$  related   by $\hcline,$ define 
$$
m(\rho, p_1,p_2):=-n(1-\rho)\max({1}/{2},\,{1}/{p_1},{1}/{p_2},\, 1-1/p, 1/p-1/2).
$$ 

B\'enyi et al.~\cite{MR3205530} proved that if $1\le p_1,p_2, p\le \infty,$    $m<m(\rho,p_1,p_2)$ and $\sigma\in BS^m_{\rho,\delta}$  then $T_\sigma$ is bounded from $L^{p_1}(\rn)\times L^{p_2}(\rn)$ to $L^p(\rn).$  On the other hand, Miyachi--Tomita~\cite{MR3179688} proved that  if $m>m(\rho, p_1,p_2),$ with $0< p_1,p_2,p\le \infty,$ there are symbols in $BS^m_{\rho,\rho}$ for which the associated  bilinear pseudodifferential operators are not bounded from $H^{p_1}(\rn)\times H^{p_2}(\rn)$ to $L^p(\rn)$ and therefore are not bounded from $L^{p_1}(\rn)\times L^{p_2}(\rn)$ to $L^p(\rn)$; recall that $H^{p}(w)(\rn)=L^p(w)(\rn)$ if $1<p<\infty)$ and $\norm{f}{H^p(w)} \leq \norm{f}{L^p(w)}$ when $0<p\leq 1$. In the case that $p=\infty$ $L^p(\rn)$ should be replaced by $BMO(\rn)$. Because of the two results stated above the class $BS^{m(\rho,p_1,p_2)}_{\rho,\delta}$ is referred to as a critical class and $m(\rho,p_1,p_2)$ is called a critical order. 

We now turn our attention to the critical classes. Miyachi--Tomita~\cite{MR3179688} showed that the symbols in $BS^{m(0,p_1,p_2)}_{0,0}$ with $0<p_1,p_2,p\le  \infty $ give rise to operators that are bounded from  $h^{p_1}(\rn)\times h^{p_2}(\rn)$ to $h^p(\rn),$ where $h^{r}(\rn)$ denotes a local Hardy space (recall that $h^{r}(\rn)=L^r(\rn)$ if $1<r<\infty$)  and $h^{r}(\rn)$ should be replaced with $bmo(\rn)$ if $r=\infty.$   In the case that $p_1 = p_2 = \infty$ Naibo~\cite{MR3411149}  proved  that if $\sigma$ is in the critical class $BS^{m(\rho, \infty,\infty)}_{\rho,\delta}$ with $0\le \delta\le \rho<1/2 $ then $T_\sigma$ is bounded from $L^{\infty}(\rn)\times L^{\infty}(\rn)$ to $BMO(\rn).$ Recently the theory of boundedness properties in the setting of Lebesgue and  Hardy  spaces for operators with symbols in the critical classes was completed in Miyachi--Tomita~\cite{MT1, MT2}: operators with symbols of critical order $ m(\rho, p_1,p_2),$ with $0\le \delta\le \rho<1$  and $0< p_1,p_2,p\le \infty,$ are bounded from $H^{p_1}(\rn)\times H^{p_2}(\rn)$ to $L^p(\rn),$ where  $L^p(\rn)$ should be replaced by $BMO(\rn)$ if $p=\infty.$

In this chapter we prove estimates in the setting of Besov and Hardy spaces for bilinear pseudodifferential operators associated to symbols in the critical classes $BS^{m(\rho,p_1,p_2)}_{\rho,\delta}$. The main result of this chapter is the following theorem.

\begin{theorem} \label{thm:main1}
Let $0<p<\infty$ and  $0<p_1,p_2\le \infty$ be such that $\hcline,$ $0<q\le \infty,$   $0\le\delta\le \rho<1$ and   $\sigma\in BS^{m(\rho,p_1,p_2)}_{\rho,\delta}.$ If $s>\tpline,$ then it holds that
\begin{equation}\label{eq:main1}
\norm{T_\sigma(f,g)}{\B{s}{p}{q}}\lesssim \norm{f}{\B{s}{p_1}{q}}\norm{g}{h^{p_2}} +\norm{f}{h^{p_1}}\norm{g}{\B{s}{p_2}{q}}\quad \forall f,g\in \sw,
\end{equation}
where $h^{p_1}$ and $h^{p_2}$ must be replaced by $L^\infty$ if $p_1=\infty$ or $p_2=\infty,$ respectively. Moreover,  if there exits $\varepsilon>0$ such that the Fourier transform of  $\sigma(\cdot,\xi,\eta)$ is  supported outside the set  $\{\zeta\in\rn:\abs{\zeta}<\varepsilon (\abs{\xi}+\abs{\eta})\}$ for all  $\xi,\eta\in\rn$ such that $1/32\abs{\xi}\le \abs{\eta}\le 32 \abs{\eta},$ then \eqref{eq:main1} holds for any $s\in\re.$
\end{theorem}

For symbols in the forbidden class $BS^{0}_{1,1}$ Leibniz-type rules in the spirit of \eqref{eq:main1} have been proved in B\'enyi~\cite{MR1996120}, Koezuka--Tomita~\cite{MR3750316} and Naibo~\cite{MR3393696}.  Another related result for symbols in the sub-critical classes $BS^m_{\rho,\delta}$ with $m<m(\rho,p_1,p_2)$ and $1\le p_1,p_2\le \infty$ and in the critical classes $BS^{m(0,p_1,p_2)}_{0,0}$ with $1<p_1,p_2,p<\infty,$ \eqref{eq:main1} was proved in Naibo~\cite[Theorem 1.3]{MR3393696} under the condition $s>\tpline.$ Theorem~\ref{thm:main1} extends this result to all critical classes and with additional assumptions on $\sigma$ allows for any $s\in\re$.

The proof of Theorem~\ref{thm:main1} uses the fact that operators with symbols in $BS^{(0,p_1,p_2)}_{0,0}$ that are localized at certain dyadic frequencies are bounded in the setting of local Hardy spaces; no other boundedness properties of operators with symbols in the bilinear H\"ormander classes are required in the proof. The tools employed are inspired by bilinear techniques in Naibo~\cite{MR3393696} and linear ones in   Johnsen~\cite{MR2163627}, Marschall~\cite{MR1376592} and   Park~\cite{MR3759556}.

\section{Preliminaries}