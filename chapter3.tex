% +--------------------------------------------------------------------+
% | Sample Chapter 3
% +--------------------------------------------------------------------+

\cleardoublepage

% +--------------------------------------------------------------------+
% | Replace "This is Chapter 3" below with the title of your chapter.
% | LaTeX will automatically number the chapters.                      
% +--------------------------------------------------------------------+

\chapter{Bilinear H\"ormander Classes of Critical Order}\label{chapter3}
\label{makereference3}

\section{Introduction}

In this chapter we obtain Leibniz-type rules for bilinear multiplier operators associated to symbols in the H\"ormander classes of critical order in the setting of local Hardy spaces. First, we will discuss the bilinear H\"ormander classes $BS^m_{\rho,\delta}$ and what it means for these symbols to be of critical order. Given $0\leq \delta \leq \rho \leq 1$ and $m\in\re$, a complex-valued function $\sigma = \sigma(x,\xi,\eta)$, $x,\xi,\eta \in \rn$, belongs to the bilinear H\"ormander class $BS^m_{\rho,\delta}$ if for any multiindices $\alpha,\beta,\gamma \in \mathbb{N}^n_0$ there exists a positive constant $C_{\alpha,\beta,\gamma}$ such that 
\begin{equation}\label{def:Bmrd}
|\partial_x^\alpha \partial_\xi^\beta \partial_\eta^\gamma \sigma(x, \xi, \eta)| \leq C_{\alpha, \beta, \gamma} (1+|\xi|+|\eta|)^{m +\delta \abs{\alpha}-\rho(\abs{\beta+\gamma})} \quad \forall x, \xi, \eta \in \rn.
\end{equation}
Then for any $\sigma \in BS^m_{\rho,\delta},$ the bilinear pseudodifferential operator $T_\sigma$ associated to $\sigma$ is defined as in \ref{psydo}. Boundedness properties of bilinear pseudodifferential operators with symbols
in the bilinear H\"ormander classes have been extensively studied in the settings of
Lebesgue and Hardy spaces; see B\'enyi-Bernicot-Maldonado-Naibo-Torres \citep{MR2660466}, B\'enyi-Chaffee-Naibo \citep{benyi2018strongly}, B\'enyi-Maldonado-Naibo-Torres \citep{MR1986065, MR2660466}, Brummer-Naibo \citep{MR3750234}, Herbert-Naibo \citep{MR3627725}, \citep{MR3211086}], Koezuka-Tomita \citep{MR3750316}, Michalowski-Rule-Staubach \citep{MR3165300}, Miyachi-Tomita
\citep{MR3179688, miyachi2018bilinear, miyachi2018bilinear2}, Naibo \cite{MR3393696, MR3411149}, Rodr\'iguez-L\'opez-Staubach \citep{MR3035059}, and the references therein. One fundamental aspect of the study of such symbols is their symbolic calculus for the transposes of operators associated to them. This was established in the works B\'enyi-Torres \citep{MR1986065} and B\'enyi-Maldonado-Naibo-Torres \citep{MR2660466}. Another important aspect of the study of these operators is their boundedness properties in a variety of function spaces. Operators associated to symbols in $BS^0_{1,0}$ can be realized as Calder\'on-Zygmund operators. As a consequence operators associated to symbols in $BS^0_{1,0}$ are bounded from $L^{p_1}(\rn) \times L^{p_2}(\rn)$ to  $L^p(\rn)$ for  $1 < p_1, p_2 < \infty$ and $1/2<p <\infty$ related through $\hcline.$ These operators also satisfy the endpoint mappings $L^\infty(\rn) \times L^\infty(\rn) \rightarrow BMO(\rn)$ and $L^1(\rn) \times L^1(\rn) \rightarrow L^{1/2, \infty}(\rn)$. For the development of the Calder\'on-Zygmund theory see Coifman-Meyer \citep{MR518170}, Kenig-Stein \citep{MR1713146}, and Grafakos-Torres \citep{MR1880324}. Operators with symbols in the forbidden class $BS^0_{1,1}$ may fail to be bounded in Lebesgue spaces. As such operators associated to these symbols are better understood in other settings. In B\'nyi et al. ~\cite{MR1996120, MR2250054, MR1986065} estimates in Sobolev spaces were obtained for such operators. For results in the settings of Besov and Triebel-Lizorkin spaces see Brummer--Naibo~\cite{MR3750234}, Koezuka--Tomita~\cite{MR3750316} and Naibo~\cite{MR3393696}.


Given  $0\le \delta\le \rho< 1$  and $0< p_1,p_2,p\le \infty$  related   by $\hcline,$ define 
$$
m(\rho, p_1,p_2):=-n(1-\rho)\max({1}/{2},\,{1}/{p_1},{1}/{p_2},\, 1-1/p, 1/p-1/2).
$$ 

B\'enyi et al.~\cite{MR3205530} proved that if $1\le p_1,p_2, p\le \infty,$    $m<m(\rho,p_1,p_2)$ and $\sigma\in BS^m_{\rho,\delta}$  then $T_\sigma$ is bounded from $L^{p_1}(\rn)\times L^{p_2}(\rn)$ to $L^p(\rn).$  On the other hand, Miyachi--Tomita~\cite{MR3179688} proved that  if $m>m(\rho, p_1,p_2),$ with $0< p_1,p_2,p\le \infty,$ there are symbols in $BS^m_{\rho,\rho}$ for which the associated  bilinear pseudodifferential operators are not bounded from $H^{p_1}(\rn)\times H^{p_2}(\rn)$ to $L^p(\rn)$ and therefore are not bounded from $L^{p_1}(\rn)\times L^{p_2}(\rn)$ to $L^p(\rn)$; recall that $H^{p}(w)(\rn)=L^p(w)(\rn)$ if $1<p<\infty)$ and $\norm{f}{H^p(w)} \leq \norm{f}{L^p(w)}$ when $0<p\leq 1$. In the case that $p=\infty$ $L^p(\rn)$ should be replaced by $BMO(\rn)$. Because of the two results stated above the class $BS^{m(\rho,p_1,p_2)}_{\rho,\delta}$ is referred to as a critical class and $m(\rho,p_1,p_2)$ is called a critical order. 

We now turn our attention to the critical classes. Miyachi--Tomita~\cite{MR3179688} showed that the symbols in $BS^{m(0,p_1,p_2)}_{0,0}$ with $0<p_1,p_2,p\le  \infty $ give rise to operators that are bounded from  $h^{p_1}(\rn)\times h^{p_2}(\rn)$ to $h^p(\rn),$ where $h^{r}(\rn)$ denotes a local Hardy space (recall that $h^{r}(\rn)=L^r(\rn)$ if $1<r<\infty$)  and $h^{r}(\rn)$ should be replaced with $bmo(\rn)$ if $r=\infty.$   In the case that $p_1 = p_2 = \infty$ Naibo~\cite{MR3411149}  proved  that if $\sigma$ is in the critical class $BS^{m(\rho, \infty,\infty)}_{\rho,\delta}$ with $0\le \delta\le \rho<1/2 $ then $T_\sigma$ is bounded from $L^{\infty}(\rn)\times L^{\infty}(\rn)$ to $BMO(\rn).$ Recently the theory of boundedness properties in the setting of Lebesgue and  Hardy  spaces for operators with symbols in the critical classes was completed in Miyachi--Tomita~\cite{MT1, MT2}: operators with symbols of critical order $ m(\rho, p_1,p_2),$ with $0\le \delta\le \rho<1$  and $0< p_1,p_2,p\le \infty,$ are bounded from $H^{p_1}(\rn)\times H^{p_2}(\rn)$ to $L^p(\rn),$ where  $L^p(\rn)$ should be replaced by $BMO(\rn)$ if $p=\infty.$

In this chapter we prove estimates in the setting of Besov and Hardy spaces for bilinear pseudodifferential operators associated to symbols in the critical classes $BS^{m(\rho,p_1,p_2)}_{\rho,\delta}$. The main result of this chapter is the following theorem.

\begin{theorem} \label{thm:main1}
Let $0<p<\infty$ and  $0<p_1,p_2\le \infty$ be such that $\hcline,$ $0<q\le \infty,$   $0\le\delta\le \rho<1$ and   $\sigma\in BS^{m(\rho,p_1,p_2)}_{\rho,\delta}.$ If $s>\text{max}(0,n(1/p - 1)),$ then it holds that
\begin{equation}\label{eq:main1}
\norm{T_\sigma(f,g)}{\B{s}{p}{q}}\lesssim \norm{f}{\B{s}{p_1}{q}}\norm{g}{h^{p_2}} +\norm{f}{h^{p_1}}\norm{g}{\B{s}{p_2}{q}}\quad \forall f,g\in \sw,
\end{equation}
where $h^{p_1}$ and $h^{p_2}$ must be replaced by $L^\infty$ if $p_1=\infty$ or $p_2=\infty,$ respectively. Moreover,  if there exits $\varepsilon>0$ such that the Fourier transform of  $\sigma(\cdot,\xi,\eta)$ is  supported outside the set  $\{\zeta\in\rn:\abs{\zeta}<\varepsilon (\abs{\xi}+\abs{\eta})\}$ for all  $\xi,\eta\in\rn$ such that $1/32\abs{\xi}\le \abs{\eta}\le 32 \abs{\eta},$ then \eqref{eq:main1} holds for any $s\in\re.$
\end{theorem}

Results related to estimate \ref{critical_est} were proved for the forbidden class $BS^0_{1,1}$ in Koezuka-Timita \cite{MR3750316} and Naibo \citep{MR3393696}. Concerning bilinear pseudodifferential operators with symbols belonging to the subcritical classes $BS^m_{\rho,\delta}$ with $m<m(\rho,p_1,p_2)$ and $1\leq p_1,p_2,p \leq \infty$ ($1 < p_1,p_2,p < \infty$ when $\rho = \delta = 0$) estimate \ref{critical_est} was shown in Naibo \citep{MR3393696} Theorem 1.3. Theorem \ref{thm:critical_besov} extends this result to the critical classes and allows for the indices to be in the wider range $(0,\infty)$. 

The proof of Theorem~\ref{thm:main1} uses the fact that operators with symbols in $BS^{(0,p_1,p_2)}_{0,0}$ that are localized at certain dyadic frequencies are bounded in the setting of local Hardy spaces; no other boundedness properties of operators with symbols in the bilinear H\"ormander classes are required in the proof. The tools employed are inspired by bilinear techniques in Naibo~\cite{MR3393696} and linear ones in   Johnsen~\cite{MR2163627}, Marschall~\cite{MR1376592} and   Park~\cite{MR3759556}.

\section{Preliminaries}


\subsection{Marchall's inequality for bilinear pseudodiffernetial operators}
In this section we let the maximal function $\mathcal{M}_r$ be defined as in \ref{maximal_Mr}. Recall that by properties of the Hardy-Littlewood maximal function $\mathcal{M}$, it follows that if $r\leq p\leq \infty$ then 
\begin{equation}
\norm{\mathcal{M}_r(f)}{L^p} \lesssim \norm{f}{L^p}.
\end{equation}

The following result follows from   Marschall~\cite[p.118, Proposition 5(a)]{MR1376592} and  Johnsen~\cite[p.275, Proposition 4.1]{MR2163627}:

\begin{lem}\label{lem:Marschall} Consider $F\in \mathcal{S}(\re^N)$ and let $\Sigma=\Sigma(X, \zeta)$ be a  symbol in $C^\infty(\re^N \times \re^N)$ such that for some polynomial $P(\zeta),$
\[
\abs{\Sigma(X, \zeta)}\lesssim P(\zeta)\quad \forall X,\zeta\in\re^N.
\]
Suppose there exists $k_0\in \ent$ such that
$$
\supp (\Sigma(X, \cdot)) \subset  \{ \zeta \in \re^N:  |\zeta| \le 2^{k_0}\}\quad \forall X\in \re^N
$$
and
$$
\supp(\widehat{F})  \subset  \{ \zeta \in \re^N:  |\zeta| \le 2^{k_0}\}.
$$
If $0<r \le 1$ and $\norm{\Sigma(X, 2^{k_0} \cdot)}{\dot{B}^{N/r}_{1,r}(\re^N)} \mathcal{M}_r(F)(X)$ is locally integrable in $\re^N,$ it holds that
\begin{equation}\label{eq:Marschall:N}
|T_{\Sigma}(F)(X)| \lesssim \norm{\Sigma(X, 2^{k_0} \cdot)}{\dot{B}^{N/r}_{1,r}(\re^N)} \mathcal{M}_r(F)(X) \quad \forall X \in \re^N,
\end{equation}
where the implicit constant is independent of $\Sigma,$ $F$ and $k_0.$
\end{lem}

We will use Lemma \ref{lem:Marschall} to prove the following bilinear version of Marschall's inequality:
\begin{lemma}\label{lem:biMarschall}  Consider $f,g\in\sw$  and let $\sigma=\sigma(x,\xi,\eta)$ be a  symbol in $C^\infty(\re^{3n})$ 
such that for some polynomial $P(\xi,\eta),$
\[
\abs{\sigma(x,\xi,\eta)}\lesssim P(\xi,\eta)\quad \forall x,\xi,\eta\in\rn.
\]
Suppose there exists $k_0\in\ent$  such that
$$
\supp(\sigma(x, \cdot, \cdot)) \subset \{(\xi, \eta) \in \re^{2n}: |\xi| + |\eta| \le 2^{k_0}\}\quad \forall x\in\rn
$$
and 
$$
\supp(\widehat{f}), \supp(\widehat{g}) \subset \{\xi \in \re^{n}: |\xi| \le 2^{k_0}\}.
$$
If $0< r\le1$ and $\norm{\sigma(x, 2^{k_0+1} \cdot, 2^{k_0+1} \cdot)}{W^{\lfloor{2n/r}\rfloor + 1,1}(\re^{2n})} \mathcal{M}_r(f\otimes g)(x,y) $ is locally integrable in $\rtn,$ it holds that
\begin{equation}\label{eq:biMarschall}
|T_{\sigma}(f,g)(x)| \lesssim  \norm{\sigma(x, 2^{k_0+1} \cdot, 2^{k_0+1} \cdot)}{W^{\lfloor{2n/r}\rfloor + 1,1}(\re^{2n})} \mathcal{M}_r(f)(x) \mathcal{M}_r(g)(x) \quad \forall x \in \rn,
\end{equation}
where the implicit constant is independent of $\sigma,$  $f,$ $g$ and $k_0.$
\end{lemma}
\begin{proof}
 We have that
$$
T_{\sigma}(f,g)(x) = \int_{\rtn} \sigma(x, \xi, \eta) \fhat(\xi) \ghat(\eta) \eixxe\dxi\deta
$$
can be regarded as the restriction to the diagonal in $\re^{2n}$ of the linear pseudodifferential operator
$$
T_\Sigma(F)(X) = \int_{\re^{2n}} \Sigma(X, \zeta) \widehat{F}(\zeta) e^{2\pi i X \cdot \zeta } \dzeta
$$
after setting $\zeta=(\xi, \eta)$ and defining, for $X = (x, y) \in \re^{2n}$, 
$$
\Sigma(X, \zeta) :=\sigma(x, \xi, \eta)\quad and \quad F(X):= (f\otimes g) (X)=f(x)g(y).
$$
Note that $\Sigma(X, \zeta)$ is in $C^\infty(\rtn\times \rtn),$  has polynomial growth in $\zeta$ uniformly in $X$ and  is supported in $\{\zeta\in\re^{2n}: \abs{\zeta}\le 2^{k_0}\}$ for each $X\in \rtn;$ moreover $\widehat{F}(\zeta) =  \widehat{f}(\xi) \widehat{g}(\eta)$ is supported in $\{\zeta \in \re^{2n}: |\zeta| \leq 2^{k_0+1} \}.$ Then, \eqref{eq:biMarschall} follows after applying Lemma~\ref{lem:Marschall}  and \eqref{eq:Marschall:N:Sob} to $T_{\Sigma}(F)$ and noticing that 
\[
\mathcal{M}_r(F)(x, x)\lesssim \mathcal{M}_r(f)(x)\mathcal{M}_r(g)(x)\quad \forall x\in\rn.
\]
\end{proof}

\begin{remark}
We note that $\norm{\sigma(x, 2^{k_0+1} \cdot, 2^{k_0+1} \cdot)}{W^{\lfloor{2n/r}\rfloor + 1,1}(\re^{2n})} \mathcal{M}_r(f\otimes g)(x,y) $ is locally integrable when  $\norm{\sigma(x, 2^{k_0+1} \cdot, 2^{k_0+1} \cdot)}{W^{\lfloor{2n/r}\rfloor + 1,1}(\re^{2n})}$ is a bounded function of $x$ since $\mathcal{M}_r(f\otimes g)(x,y)$ is locally integrable.
\end{remark}


\section{Decomposition of $T_\sigma$ and strategy for the proof of Theorem \ref{thm:main1}}\label{sec:decomp}
In the proofs that follow in this chapter we implicitly assume that the symbol $\sigma$ in the statements of Theorem \ref{thm:main1} and Corollary \ref{coro:main1} has compact support in $\mathbb{R}^{3n}$. A limiting argument then allows us to to prove these results for sumbols in the bilinear H\"ormander classes without compact support. Indeed, for $0\le \delta,\rho\le 1,$ $m\in\re,$ $\sigma\in BS^m_{\rho,\delta}$ and $0\le \varepsilon<1$ let $\sigma_\epsilon (x,\xi,\eta) = \Psi(\varepsilon x, \varepsilon \xi, \varepsilon \eta) \sigma( x, \xi,\eta)$ for a smooth  function $\Psi$ of compact support such that $\Psi(0,0,0) = 1$. It follows that $\sigma_\varepsilon \in BS^m_{\rho,\delta}$ with constants independent of $\varepsilon$ and that, as $\varepsilon \rightarrow 0$, $T_{\sigma_\varepsilon}(f,g)$ converges to $T_\sigma(f,g)$ in $\swp$ for $f,g \in \sw$. Indeed, using the smoothness and compact support of $\Psi$ and the Leibniz rule we have that 
\begin{align*}
|\partial^\alpha_\xi \partial^\beta_\eta \sigma_\varepsilon(x,\xi,\eta)| & 
\leq \sum_{\alpha_0 \leq \alpha, \beta_0 \leq \beta} \varepsilon^{|\alpha_0| + |\beta_0|}|\partial^{\alpha_0}_\xi \partial^{\beta_0}_\eta \Psi(\varepsilon x, \varepsilon \xi, \varepsilon \eta) \partial^{\alpha - \alpha_0}_\xi \partial^{\beta - \beta_0}_\eta \sigma(x,\xi,\eta) | \\
& \leq \sum_{\alpha_0 \leq \alpha, \beta_0 \leq \beta} C_{\alpha_0 , \beta_0} (1 + |\xi| + |\eta|)^{m + \delta|\alpha - \alpha_0| - \rho|\beta - \beta_0|} \\
& =(1 + |\xi| + |\eta|)^{m + \delta|\alpha| - \rho|\beta|} \sum_{\alpha_0 \leq \alpha, \beta_0 \leq \beta} C_{\alpha_0 , \beta_0} (1 + |\xi| + |\eta|)^{-\delta| \alpha_0| + \rho|\beta_0|} \\
& \leq C_{\alpha,\beta} (1 + |\xi| + |\eta|)^{m + \delta|\alpha| - \rho|\beta|},
\end{align*}
for $(x,\xi,\eta) \in \text{supp}(\Psi)$ and $\sigma_\varepsilon \in BS^m_{\rho,\delta}$. Additionally by the dominated convergence theorem and using that  $\Psi(0,0,0) = 1$ we get that $T_{\sigma_\eps}(f,g) \rightarrow T_\sigma(f,g)$ in $\swp$ as $\eps \rightarrow 0$. Finally by the Fatou property of Besov spaces the estimates for $T_\sigma$ follow from the estimates for $T_\sigma_\eps$ which are uniform in $\eps$. 

We know present the decomposition of $T_\sigma$ that will be used in the proof of Theorem \ref{thm:main1}. Let $\varphi,\varphi_0 \in \sw$ satisfy conditions \ref{TL_B_psi1} and \ref{TL_B_phi1} respectively, and be such that $\sumkNz \fk \equiv 1,$ where $\fk(\xi)=\f(2^{-k}\xi)$ for $\xi\in\rn$ and $k\in\na.$

Let $m \in \re,$ $0 \leq \delta\le \rho < 1$ and $\sigma \in BS^m_{\rho, \delta}.$  We perform a triple spectral decomposition of $T_\sigma(f,g)$ with $f,g\in\sw:$
\begin{align*}
T_\sigma(f,g)(x)&= \int_{\rtn} \sigma(x, \xi, \eta) \fhat(\xi) \ghat(\eta) \eixxe \dxi \deta \\
& =  \sumjkNz \int_{\rtn} \left( \int_{\rn} \widehat{\sigma}^1(\zeta, \xi, \eta) \eixzeta  d\zeta \right) \fk(\xi) \fj(\eta) \fhat(\xi) \ghat(\eta) \eixxe \dxi \deta\\
& =  \sum_{j,k,\ell\in\na_0} \left( \int_{\rn} \f_\ell(\zeta)\widehat{\sigma}^1(\zeta, \xi, \eta) \eixzeta  d\zeta \right) \fk(\xi) \fj(\eta) \fhat(\xi) \ghat(\eta) \eixxe \dxi \deta\\
&= \sum_{j,k,\ell\in\na_0}  T_{\sigma_{j,k,\ell}}(f,g)(x),
\end{align*}
where for $j,k, \ell \in \naz$ we define
\begin{equation*}
\sigma_{j,k,\ell}(x, \xi, \eta)= \fk(\xi) \fj(\eta) \int_{\rn}  \f_\ell(\zeta) \widehat{\sigma}^1(\zeta, \xi, \eta) \eixzeta  \dzeta.
\end{equation*}
We next introduce the symbols  

\begin{align*}
& \sigma^1:= \quad \sum\limits_{\ell=4}^\infty \sum\limits_{k=0}^{\ell - 4} \sum\limits_{j =0}^{k} \sigma_{j,k,\ell}, \quad \sigma^2:= \quad \sum\limits_{k=0}^\infty \sum\limits_{j=0}^k \sum\limits_{\ell =\max(0,k-3)}^{k+3} \sigma_{j,k,\ell},\quad\sigma^3:= \quad \sum\limits_{k=4}^\infty \sum\limits_{j=0}^k \sum\limits_{\ell =0}^{k-4} \sigma_{j,k,\ell},\\
&\sigma^4:= \quad \sum\limits_{\ell=5}^\infty \sum\limits_{j=1}^{\ell - 4} \sum\limits_{k =0}^{j-1} \sigma_{j,k,\ell}, \quad \sigma^5:= \quad  \sum\limits_{j=1}^\infty \sum\limits_{k=0}^{j-1} \sum\limits_{\ell =\max(0,j-4)}^{j+3} \sigma_{j,k,\ell},\quad \sigma^6:= \quad \sum\limits_{j=5}^\infty \sum\limits_{k=0}^{j-1} \sum\limits_{\ell =0}^{j-5} \sigma_{j,k,\ell},
\end{align*}
so that $\sigma = \sigma^1+\sigma^2+\sigma^3+\sigma^4+\sigma^5+\sigma^6.$ Notice that since $j\leq k$ in $\sigma_1$, $\sigma_2$, and $\sigma_3$ they are supported on the set $\{(x,\xi,\eta) \in \mathbb{R}^{3n} : |\eta| \leq 4|\xi|\}$. On the other hand $\sigma_4$, $\sigma_5$, and $\sigma_6$ are supported on $\{(x,\xi,\eta) \in \mathbb{R}^{3n} : |\xi| \leq 2|\eta|\}$ since in these sums $k \leq j-1$. By taking the Fourier transform with respect to the first variable we have that $\widehat{\sigma^1(\cdot,\xi,\eta)}$ is supported on $\{\zeta\in\rn: \abs{\xi}\lesssim \abs{\zeta}\},$
$\widehat{\sigma^2(\cdot,\xi,\eta)}$ is supported mostly on $\{\zeta\in\rn: \abs{\xi}\sim \abs{\zeta}\}$ and $\widehat{\sigma^3(\cdot,\xi,\eta)}$ is supported on $\{\zeta\in\rn: \abs{\zeta}\lesssim \abs{\xi}\}$. The supports of $\widehat{\sigma^4(\cdot,\xi,\eta)}$, $\widehat{\sigma^5(\cdot,\xi,\eta)}$, and $\widehat{\sigma^6(\cdot,\xi,\eta)}$ are in supported in similar sets with $|\xi|$ replaced by $|\eta|$. The proof of Theorem \ref{thm:main1} will follow from obtaining bounds for $T_{\sigma^j}$, $j=1,2,3,4,5,6$. In this dissertation we will show the proofs of the results for $T_{\sigma^1}$, $T_{\sigma^2}$, and $T_{\sigma^3}$. By symmetry the results for $T_{\sigma^4}$, $T_{\sigma^5}$, and $T_{\sigma^6}$ are obtained in an analogous way as $T_{\sigma^1}$, $T_{\sigma^2}$, and $T_{\sigma^3}$ respectively with the roles of $\xi$ and $\eta$ reversed.
\subsection{Estimates for $T_{\sigma^1}$}
In this section we prove the estimates for the operator $T_{\sigma_1}$

The precise bounds for $T_{\sigma^1}$ are stated in the following lemma.

\begin{lemma}\label{lem:T1} Let $m \in \re,$  $0 \le \delta\le  \rho < 1$ and $\sigma \in BS^m_{\rho, \delta}$. If   $0<p,p_1,p_2\le \infty$ are such that $\hcline,$ $0 < q ,\bar{q}\leq \infty,$ $s_1, s_2 \in \re$, it holds that
\begin{equation}\label{eq:estT1}
\norm{T_{\sigma^1}(f,g)}{\B{s_1}{p}{q}}  \lesssim \norm{f}{\B{s_2}{p_1}{\bar{q}}} \norm{g}{h^{p_2}} \quad \forall f, g \in \sw,
\end{equation}
where $\sigma^1$ is as in Section~\ref{sec:decomp} and $h^{p_2}$ must be replaced by $L^\infty$ if $p_2=\infty.$
\end{lemma}


We note that in Lemma \ref{lem:T1} there is no restriction on the order $m$ of the symbol and the regularity indices, $s_1$ and $s_2$, can be different. In particular Lemma \ref{lem:T1} implies that 
\norm{T_{\sigma^1}(f,g)}{\B{s}{p}{q}}  \lesssim \norm{f}{\B{s}{p_1}{q}} \norm{g}{h^{p_2}} \quad \forall f, g \in \sw,
\end{equation*}
for $s\in\re$ and $\sigma\in BS^{m(\rho,p_1,p_2)}_{\rho,\delta}$ as needed for the proof of Theorem~\ref{thm:main1}.



\begin{proof}  Let $m,$ $\rho,$ $p,$ $p_1,$ $p_2,$ $q,$ $\bar{q},$ $s_1,$  $s_2,$ $\sigma,$ $\sigma^1,$ $f$ and $g$ be as in the statement of the lemma. 

For $\ell \in \naz$  set
$$
\sigma^1_\ell:=  \sum\limits_{k=0}^{\ell -4} \sum\limits_{j =0}^{k} \sigma_{j,k,\ell},
$$
so that $T_{\sigma^1}(f,g)= \sum\limits_{\ell=4}^\infty T_{\sigma^1_\ell}(f,g)$. Recalling the definition of $\Phi_k$ and $\sigma_{j,k,\ell},$ we have
\begin{align*}
T_{\sigma^1_\ell}(f,g)(x)= \int_{\re^{3n}} \sum\limits_{k=0}^{\ell -4}   \fk(\xi) \Phi_k(\eta)  \f_\ell(\zeta) \widehat{\sigma}^1(\zeta, \xi, \eta) \widehat{f}(\xi) \widehat{g}(\eta) e^{2\pi i x \cdot (\xi + \eta + \zeta)}  \dzeta \dxi \deta,
\end{align*}
and changing variables, we get
\begin{align*}
T_{\sigma^1_\ell}(f,g)(x)= \int_{\rn} \left( \int_{\rtn}  \sum\limits_{k=0}^{\ell -4}   \fk(\xi) \Phi_k(\eta)  \f_\ell(\omega- \xi - \eta) \widehat{\sigma}^1(\omega- \xi - \eta, \xi, \eta) \widehat{f}(\xi) \widehat{g}(\eta) \dxi \deta \right)  e^{2\pi i x \cdot \omega} \, d\omega.
\end{align*}
This yields
\begin{align*}
\widehat{T_{\sigma^1_\ell}(f,g)}(\omega)=  \int_{\rtn}  \sum\limits_{k=0}^{\ell -4}   \fk(\xi) \Phi_k(\eta)  \f_\ell(\omega- \xi - \eta) \widehat{\sigma}^1(\omega- \xi - \eta, \xi, \eta) \widehat{f}(\xi) \widehat{g}(\eta) \dxi \deta,
\end{align*}
where the integral effectively takes place when $ 2^{\ell-1}\le |\omega- \xi - \eta| \leq 2^{\ell +1}$ (keep in mind that $\ell\ge 4$)  as well as $|\xi| \leq 2^{\ell-3}$ and $|\eta| \leq 2^{\ell-3}.$  Consequently,
$\widehat{T_{\sigma^1_\ell}(f,g)}$ is supported in $\{\omega\in\rn: 2^{\ell-2}\le |\omega| \leq 2^{\ell +2}\}$. Then, the fact that  $T_{\sigma^1}(f,g)= \sum\limits_{\ell=4}^\infty T_{\sigma^1_\ell}(f,g)$  and  Lemma~\ref{lem:nikol} give
\begin{equation}\label{eq:T:sigma:1:fg}
\norm{T_{\sigma^1}(f,g)}{\B{s_1}{p}{q}} \lesssim  \left( \sum\limits_{\ell=4}^\infty 2^{\ell s_1q} \norm{T_{\sigma^1_\ell}(f,g)}{{L^p}}^q \right)^\frac{1}{q}.
\end{equation}

Recalling the definitions of $\widetilde{\f}_k$ and $\widetilde{\Phi}_k,$  we have
\begin{align*}
T_{\sigma^1_\ell}(f,g)(x) = 
  \sum\limits_{k=0}^{\ell -4} T_{\sigma^1_{\ell,k}}(\widetilde{\f}_k(D)f,\widetilde{\Phi}_k(D)g)(x),
\end{align*}
where, for each $k$ between $0$ and $\ell -4$, we set 
$$
\sigma^1_{\ell, k}(x, \xi, \eta):=\left( \int_{\rn}  \f_\ell(\zeta) \widehat{\sigma}^1(\zeta, \xi, \eta) e^{2\pi i x \cdot \zeta}  \dzeta \right)  \fk(\xi) \Phi_k(\eta).
$$ 
Since $\sigma^1_{\ell,k}$ satisfies 
$$\abs{\sigma^1_{\ell,k}(x,\xi,\eta)}\lesssim (1+\abs{\xi}+\abs{\eta})^m\quad \forall x,\xi,\eta\in\rn,$$
$\sigma^1_{\ell, k}(x,\cdot,\cdot)$ is supported on $\{(\xi, \eta) \in \re^{2n} : |\xi| + |\eta| \leq 2^{k+2}\}$ for all $x\in\rn$ and $\widetilde{\f}_k(D)f$ and $\widetilde{\Phi}_k(D)g$ are Schwartz functions with  Fourier transforms  supported in $\{\xi \in \rn: |\xi| \le 2^{k+1}\},$  we can apply the bilinear Marschall inequality \eqref{eq:biMarschall} with $0<r\le 1$ to get 
\begin{align}\label{eq:bound:h1:ell:k}
&|T_{\sigma^1_{\ell,k}}(\widetilde{\f}_k(D)f,\widetilde{\Phi}_k(D)g)(x)| \\
&\hspace{0.5cm}\lesssim  \norm{\sigma^1_{\ell, k}(x, 2^{k+3} \cdot, 2^{k+3} \cdot)}{W^{\lfloor{2n/r}\rfloor + 1,1}(\re^{2n})} \mathcal{M}_r(\widetilde{\f}_k(D)f)(x)\mathcal{M}_r(
\widetilde{\Phi}_k(D)g)(x)\quad \forall x\in\rn.\nonumber
\end{align}
(See Remark~\ref{remark:locint} along with \eqref{eq:estsigkl} below.)

We next estimate $\norm{\sigma^1_{\ell, k}(x, 2^{k+3} \cdot, 2^{k+3} \cdot)}{W^{\lfloor{2n/r}\rfloor + 1,1}(\re^{2n})}.$ For ease of notation we just work with $2^k$ instead of $2^{k+3}.$ Notice that 
\begin{equation*}
\sigma^1_{\ell, k}(x, 2^k \xi, 2^k \eta) = \f(\xi) \f_0(\eta) \int_{\rn} \widecheck{\f_\ell}(y) \sigma(x-y, 2^k \xi, 2^k \eta)\dy\quad \forall k\in\na,
\end{equation*}
with a similar expression for $\sigma^1_{\ell, 0}$ obtained by replacing $\f$ with $\f_0$ in the formula above. 
 For $\ell \geq 4$ the function $\widecheck{\f_\ell}$ has vanishing moments of every order; if $N \in \na$, we can then write
\begin{align*}
 I_{\ell, k}(\xi, \eta)&:= \int_{\rn} \widecheck{\f_\ell}(y) \sigma(x-y, 2^k \xi, 2^k \eta)\dy=2^{n \ell}\int_{\rn} \widecheck{\f}(2^{\ell}y) \sigma(x-y, 2^k \xi, 2^k \eta)\dy\\
& = 2^{n \ell} \int_{\rn} \widecheck{\f}(2^{\ell}y) \left( \sigma(x-y, 2^k \xi, 2^k \eta) - \sum\limits_{|\alpha| < N} \frac{1}{\alpha!} (-y)^\alpha \partial_x^\alpha \sigma(x, 2^k \xi, 2^k \eta) \right)\dy\\
& = 2^{n \ell} \int_{\rn} \widecheck{\f}(2^{\ell}y) \sum\limits_{|\alpha| = N} \frac{N}{\alpha!} (-y)^\alpha \int_0^1 (1-t)^{N-1} \partial_x^\alpha \sigma(x - t y, 2^k \xi, 2^k \eta) \dy.
\end{align*}
Given multiindices $\beta, \gamma \in \naz^n$ and using that $\sigma \in BS^m_{\rho, \rho}$, it follows that
\begin{align*}
& |\partial_\xi^{\beta} \partial_\eta^{\gamma}  I_{\ell, k}(\xi, \eta)|\\
& = 2^{n \ell} 2^{k (|\beta+ \gamma|)}  \left|  \int_{\rn} \widecheck{\f}(2^{\ell}y) \sum\limits_{|\alpha| = N} \frac{N}{\alpha!} (-y)^\alpha \int_0^1 (1-t)^{N-1} \partial_x^\alpha \partial_\xi^{\beta} \partial_\eta^{\gamma}\sigma(x - t y, 2^k \xi, 2^k \eta) \,dt\dy \right|\\
& \lesssim  2^{k (|\beta+ \gamma|)}  (1+ |2^k \xi| + |2^k \eta|)^{m + \rho N -  \rho|\beta+ \gamma|} \int_{\rn} 2^{n \ell} |\widecheck{\f}(2^{\ell}y)| |y|^N \dy \\
&\lesssim 2^{- N \ell} 2^{k (|\beta+ \gamma|)}  (1+ |2^k \xi| + |2^k \eta|)^{m + \rho N -  \rho |\beta+ \gamma|}.
\end{align*}
Then, for $(\xi,\eta)$ in the support of $\sigma^1_{\ell,k}(x, 2^k\cdot,2^k\cdot)$  we get
\begin{equation*}
 |\partial_\xi^{\beta} \partial_\eta^{\gamma}  I_{\ell, k}(\xi, \eta)| \lesssim 2^{- N \ell} 2^{k (1-\rho)|\beta+ \gamma|} 2^{k (m+\rho N)}.
\end{equation*}
Given $0< r \le 1$, taking  derivatives up to order $\lfloor{2n/r}\rfloor + 1$ in $(\xi,\eta)$ of
$\sigma^1_{\ell, k}(x, 2^k \xi, 2^k \eta) = \f(\xi) \f_0(\eta)  I_{\ell, k}(\xi, \eta),$
 we obtain
\begin{align}\label{eq:estsigkl}
  \norm{\sigma^1_{\ell, k}(x, 2^{k} \cdot, 2^{k} \cdot)}{W^{\lfloor{2n/r}\rfloor + 1,1}(\re^{2n})} \lesssim 2^{- N \ell} 2^{k (1-\rho)( \lfloor{2n/r}\rfloor + 1)} 2^{k (m+\rho N)}.
\end{align}
From \eqref{eq:bound:h1:ell:k}, it then follows that for all $x\in\rn,$ we have
\begin{equation*}
|T_{\sigma^1_{\ell,k}}(\widetilde{\f}_k(D)f,\widetilde{\Phi}_k(D)g)(x)|  \lesssim 2^{- N \ell} 2^{k [(1-\rho)( \lfloor{2n/r}\rfloor + 1)+m+\rho N]}    \mathcal{M}_r(\widetilde{\f}_k(D)f)(x)\mathcal{M}_r(\widetilde{\Phi}_k(D)g)(x).
\end{equation*}

  Define $\pst=\min(1,p).$  For the sake of notation, we will next work with $q$ finite; the case $q=\infty$ can be treated analogously.
  Recalling that $T_{\sigma^1_\ell}(f,g)= \sum\limits_{k=0}^{\ell -4} T_{\sigma^1_{\ell,k}}(\widetilde{\f}_k(D)f,\widetilde{\Phi}_k(D)g)$,  \eqref{eq:T:sigma:1:fg} and the last  estimate give
\begin{align}
& \norm{T_{\sigma^1}(f,g)}{\B{s_1}{p}{q}}  \lesssim  \left( \sum\limits_{\ell=4}^\infty 2^{\ell s_1q} \norm{T_{\sigma^1_\ell}(f,g)}{L^p}^q \right)^\frac{1}{q}\label{eq:T1:fg:Fqp}\\
& \lesssim  \left[  \sum\limits_{\ell=4}^\infty 2^{\ell s_1 q} \left(  \sum\limits_{k=0}^{\ell -4} 2^{- N\pst \ell} 2^{k\pst [(1-\rho)( \lfloor{2n/r}\rfloor + 1)+m+\rho N]}   \norm{\mathcal{M}_r(\widetilde{\f}_k(D)f)\mathcal{M}_r(\widetilde{\Phi}_k(D)g)}{L^p}^{\pst}\right)^\frac{q}{\pst} \right]^\frac{1}{q}. \nonumber
\end{align}
Next, let us see that, for  $N>s_1$ and $0<\epsilon< N-s_1,$  it holds that
\begin{align}\nonumber
& \sum\limits_{\ell=4}^\infty 2^{\ell  s_1 q} \left(  \sum\limits_{k=0}^{\ell -4} 2^{- N \pst\ell} 2^{k \pst[(1-\rho)( \lfloor{2n/r}\rfloor + 1)+m+\rho N]}    \norm{\mathcal{M}_r(\widetilde{\f}_k(D)f)\mathcal{M}_r(\widetilde{\Phi}_k(D)g)}{L^p}^{\pst}\right)^{\frac{q}{\pst}} \\ \label{eq:moving:the:sums}
&\hspace{2cm}\lesssim \sum\limits_{k=0}^\infty 2^{k q[ (1-\rho)( \lfloor{2n/r}\rfloor + 1-N) + m+ s_1+\eps] }   \norm{\mathcal{M}_r(\widetilde{\f}_k(D)f)\mathcal{M}_r(\widetilde{\Phi}_k(D)g)}{L^p}^q.
\end{align}
 Indeed, if $0 < q \leq \pst$, we have
\begin{align*}
& \sum\limits_{\ell=4}^\infty  2^{\ell  s_1 q} \left(  \sum\limits_{k=0}^{\ell -4} 2^{- N \pst \ell} 2^{k \pst [(1-\rho)( \lfloor{2n/r}\rfloor + 1)+m+\rho N]}   \norm{\mathcal{M}_r(\widetilde{\f}_k(D)f)\mathcal{M}_r(\widetilde{\Phi}_k(D)g)}{L^p}^{\pst}\right)^\frac{q}{\pst} \\
& \leq \sum\limits_{\ell=4}^\infty 2^{-(N-s_1) q \ell} \sum\limits_{k=0}^{\ell -4}  2^{k q [(1-\rho)( \lfloor{2n/r}\rfloor + 1) + m+\rho N]}   \norm{\mathcal{M}_r(\widetilde{\f}_k(D)f)\mathcal{M}_r\widetilde{\Phi}_k(D)g)}{L^p}^q\\
&= \sum_{k=0}^{\infty} \left(\sum_{\ell=k+4}^\infty 2^{-(N-s_1) q \ell} \right)  2^{k q [(1-\rho)( \lfloor{2n/r}\rfloor + 1) + m+\rho N]}   \norm{\mathcal{M}_r(\widetilde{\f}_k(D)f)\mathcal{M}_r(\widetilde{\Phi}_k(D)g)}{L^p}^q \\
& \lesssim \sum\limits_{k=0}^\infty  2^{k q [(1-\rho)( \lfloor{2n/r}\rfloor + 1) + m+\rho N]} 2^{-kq(N-s_1)}   \norm{\mathcal{M}_r(\widetilde{\f}_k(D)f)\mathcal{M}_r(\widetilde{\Phi}_k(D)g)}{L^p}^q,
\end{align*}
and \eqref{eq:moving:the:sums} follows for any $\eps>0.$ Now, if $\pst<q <\infty$ and $0<\eps < N - s_1,$ we have
\begin{align*}
& \sum\limits_{\ell=4}^\infty   2^{\ell  s_1 q} \left(  \sum\limits_{k=0}^{\ell -4} 2^{- N \pst \ell} 2^{k \pst [(1-\rho)( \lfloor{2n/r}\rfloor + 1) +m+\rho N]}   \norm{\mathcal{M}_r(\widetilde{\f}_k(D)f)\mathcal{M}_r(\widetilde{\Phi}_k(D)g)}{L^p}^{\pst}\right)^\frac{q}{\pst} \\
& \lesssim \sum\limits_{\ell=4}^\infty 2^{-  (N- s_1 - \eps) q \ell}  \left(  \sum\limits_{k=0}^{\ell -4} 2^{-\eps \pst k} 2^{k \pst [(1-\rho)( \lfloor{2n/r}\rfloor + 1) + m+\rho N]}   \norm{\mathcal{M}_r(\widetilde{\f}_k(D)f)\mathcal{M}_r(\widetilde{\Phi}_k(D)g)}{L^p}^{\pst}\right)^\frac{q}{\pst}\\
& \lesssim  \sum\limits_{\ell=4}^\infty 2^{-  (N- s_1 - \eps) q \ell} \sum\limits_{k=0}^{\ell -4} 2^{k q [(1-\rho)( \lfloor{2n/r}\rfloor + 1) + m+\rho N]}   \norm{\mathcal{M}_r(\widetilde{\f}_k(D)f)\mathcal{M}_r(\widetilde{\Phi}_k(D)g)}{L^p}^q \\
& =  \sum\limits_{k=0}^\infty \left( \sum\limits_{\ell=k+4}^{\infty} 2^{- (N- s_1 - \eps) q \ell} \right) 2^{k q [(1-\rho)( \lfloor{2n/r}\rfloor + 1) + m+\rho N]}   \norm{\mathcal{M}_r(\widetilde{\f}_k(D)f)\mathcal{M}_r(\widetilde{\Phi}_k(D)g)}{L^p}^q \\
& \lesssim \sum\limits_{k=0}^\infty  2^{- (N- s_1 - \eps) kq } 2^{k q [(1-\rho)( \lfloor{2n/r}\rfloor + 1) + m+\rho N]}   \norm{\mathcal{M}_r(\widetilde{\f}_k(D)f)\mathcal{M}_r(\widetilde{\Phi}_k(D)g)}{L^p}^q 
\end{align*}
and \eqref{eq:moving:the:sums} follows.

Using  \eqref{eq:moving:the:sums},  the fact that  
\begin{equation}\label{def:g*}
|\widetilde{\Phi}_k(D)g(x)| \leq \sup\limits_{0 < t \le 1} |t^{-n}\mathcal{F}^{-1}(\widetilde{\f}_0)(t^{-1} \cdot)* g(x)| =: g^*(x) \quad \forall k \in \naz,  x \in \rn, 
\end{equation}
and \eqref{eq:Mtau:in:Lp} with $0<r < \min(1,p),$ we can now continue with the inequality \eqref{eq:T1:fg:Fqp} to get
\begin{align*}
& \norm{T_{\sigma^1}(f,g)}{\B{s_1}{p}{q}}\\
&  \lesssim   \left[  \sum\limits_{k=0}^\infty   2^{k q [(1-\rho)( \lfloor{2n/r}\rfloor + 1-N) + m+s_1+\eps]}  \norm{ \mathcal{M}_r(\widetilde{\f}_k(D)f)}{L^{p_1}}^q \right]^\frac{1}{q} \norm{\mathcal{M}_r(g^*)}{L^{p_2}}\\
&  \lesssim   \left[  \sum\limits_{k=0}^\infty   2^{k q [(1-\rho)( \lfloor{2n/r}\rfloor + 1-N) + m+s_1+\eps]}   \norm{\widetilde{\f}_k(D)f}{L^{p_1}}^q \right]^\frac{1}{q} \norm{g^*}{L^{p_2}}\\
& \lesssim \norm{f}{\B{(1-\rho)( \lfloor{2n/r}\rfloor + 1-N) + m + s_1+\eps }{p_1}{q}}  \norm{ g}{h^{p_2}},
\end{align*}
 where, if $p_2=\infty,$  $\norm{g}{h^{p_2}}$ should be replaced with $\norm{g}{L^\infty}.$ Since $\rho<1$, we can choose $N$ large enough so that
$$
(1-\rho)( \lfloor{2n/r}\rfloor + 1-N) + m +s_1 +\eps  < s_2,
$$
and obtain $ \norm{f}{\B{(1-\rho)( \lfloor{2n/r}\rfloor + 1-N) + m +s_1+ \eps }{p_1}{q}} \leq  \norm{f}{\B{s_2}{p_1}{\bar{q}}}$ (by the embedding properties of Besov spaces). The proof of Lemma \ref{lem:T1} is then complete.

\end{proof}

\subsection{Estimates for $T_{\sigma^2}$}

In this section we prove the bounds for $T_\sigma^2}$ which are stated in the following lemma.

\begin{lemma}\label{lem:T2} 
Let $m \in \re,$ $0 \le \delta\le \rho < 1$ and $\sigma \in BS^m_{\rho, \delta}$. If $0<p<\infty$ and $0< p_1, p_2 \le \infty$ are such that $\hcline,$ $0 < q,\bar{q} \leq \infty$, and $s_1, s_2 \in \re,$ it holds that
\begin{equation}\label{eq:T2}
\norm{T_{\sigma^2}(f,g)}{\B{s_1}{p}{q}}  \lesssim \norm{f}{\B{s_2}{p_1}{\bar{q}}} \norm{g}{h^{p_2}} \quad \forall f, g \in \sw,
\end{equation}
where $\sigma^2$ is as in Section~\ref{sec:decomp} and $h^{p_2}$ must be replaced by $L^\infty$ if $p_2=\infty.$
\end{lemma}

Like with Lemma \ref{lem:T1} there is no restriction on the order $m$ of the symbol and the regularity indices can be different on the left and right hand sides.


\begin{proof}   Let $p_1,$ $p_2,$ $p,$ $q,$ $\bar{q},$ $s_1,$ $s_2,$ $m,$ $\rho$ and $\sigma$ be as in the hypotheses of the lemma and consider $\fk,$ $\Phi_k,$ $\widetilde{\f}_k$ and $\widetilde{\Phi}_k$ as in Section~\ref{sec:T1}. We assume $q<\infty;$ the proof for the case $q=\infty$ is analogous.

Recall that
\[
\sigma^2= \quad \sum\limits_{k=0}^\infty  \sum\limits_{\ell =\max(0,k-3)}^{k+3} \sum\limits_{j=0}^k \sigma_{j,k,\ell}
\]
and write $\sigma^2=\sigma^{2,1}+\sigma^{2,2},$ where 
\[
\sigma^{2,1}= \quad \sum\limits_{k=0}^{3}  \sum\limits_{\ell =0}^{k+3} \sum\limits_{j=0}^k \sigma_{j,k,\ell}  
\quad \text{ and }\quad  \sigma^{2,2}= \sum\limits_{k=4}^\infty  \sum\limits_{\ell =k-3}^{k+3} \sum\limits_{j=0}^k \sigma_{j,k,\ell}.
\]

Notice that the symbol $\sigma^{2,1}$ is supported on $\{(x,\xi,\eta)\in\re^{3n}: \abs{\xi}\le 2^{4} \text{ and } \abs{\eta}\le2^{4}\}$ and belongs to any H\"ormander class; in particular $\sigma^{2,1}\in BS^{m(0,p_1,p_2)}_{0,0}$ and by Miyachi--Tomita~\cite[Theorem 1.1]{MR3179688}, $T_{\sigma^{2,1}}$ is bounded from $h^{p_1}(\rn)\times h^{p_2}(\rn)$ to $h^{p}(\rn)$ (with $h^{p_1}(\rn)$ and $h^{p_2}(\rn)$ replaced by $L^\infty(\rn)$ if $p_1=\infty$ or $p_2=\infty$). Moreover, the Fourier transform of $T_{\sigma^{2,1}}(f,g)$ is supported on $\{\zeta\in\rn: \abs{\zeta}\le2^{8}\}.$ Let $h\in\sw$  be compactly supported and  identically one on $\{\xi\in\rn:\abs{\xi}\le 2^{4}\}.$ We then obtain
\begin{align*}
\norm{T_{\sigma^{2,1}}(f,g)}{\B{s_1}{p}{q}}&=\left(\sum_{k=0}^{8}2^{s_1kq}\norm{\varphi_k(D) T_{\sigma^{2,1}}(f,g)}{L^p}^q\right)^{\frac{1}{q}}\\&\lesssim \norm{T_{\sigma^{2,1}}(f,g)}{h^p}=\norm{T_{\sigma^{2,1}}(h(D)f,g)}{h^p}\\
&\lesssim \norm{h(D)f}{h^{p_1}}\norm{g}{h^{p_2}}\lesssim \norm{f}{\B{s_2}{p_1}{q}}\norm{g}{h^{p_2}},
\end{align*}
with $h^{p_1}$ or $h^{p_2}$ replaced by $L^\infty$ if $p_1=\infty$ or $p_2=\infty,$ respectively.


We next analyze the operator with  symbol $\sigma^{2,2};$ for $k \geq 4$ set 
\begin{equation*}
\sigma^{2,2,k}  = \sum\limits_{\ell =k-3}^{k+3} \sum\limits_{j=0}^k \sigma_{j,k,\ell},
\end{equation*}
so that $\sigma^{2,2} = \sum\limits_{k=4}^\infty \sigma^{2,2,k}.$ 
 Let $\K_k$ denote the bilinear kernel of  $T_{\sigma^{2,2,k}}$, that is,
$$
\K_k(x,y,z):= \int_{\rtn} \sigma^{2,2,k}(x, \xi, \eta) e^{2\pi i \xi \cdot( x- y)} e^{2\pi i \cdot \eta (x-z)} \dxi \deta.
$$
Note that
\begin{align*}
| \f_\nu(D) T_{\sigma^{2,2,k}}(f,g)(x)| \leq \int_{\re^{3n}} |\widecheck{\f_\nu}(x- w)| |\K_k(w, y, z)| |\widetilde{\f}_k(D)f(y)| |\widetilde{\Phi}_k(D)g(z)| \, dw \dy \dz.
\end{align*}
Given $0<\tau <1,$  let $M \in \na$ be such that $M \geq n/\tau$, then
\begin{align*}
|\widetilde{\f}_k(D)f(y)|&  = \frac{|\widetilde{\f}_k(D)f(y)|}{(1 + 2^k |x-y|)^M} (1 + 2^k |x-y|)^M\\
& \leq  (1 + 2^k |x-y|)^M \sup_{y\in\rn} \frac{|\widetilde{\f}_k(D)f(y)|}{(1 + 2^k |x-y|)^M} \\
& \lesssim (1 + 2^k |x-y|)^M \Mtau(\widetilde{\f}_k(D)f)(x)\quad\forall x,y\in\rn,k\in\na_0,
\end{align*}
where for the last inequality we used Peetre's maximal inequality (see Peetre~\cite{MR0380394} or Triebel~\cite[p.16, Theorem 1.3.1]{MR3024598}). Similarly, and recalling \eqref{def:g*},
\begin{align*}
|\widetilde{\Phi}_k(D)g(z)| & \lesssim (1 + 2^k |x-z|)^M \Mtau(\widetilde{\Phi}_k(D)g)(x)  \leq (1 + 2^k |x-z|)^M \Mtau(g^*)(x)
\end{align*}
for all $ x,z\in\rn,$ $k\in\na_0.$
Given $J, N \in \na$ with $N>2(M+n),$  we use the estimates above, the fact that $\f\in\sw$ and Lemmas \ref{lem:estimateKk} and  \ref{lem:int:M:N:k}  in Appendix~\ref{sec:app} to obtain 
\begin{align*}
 |\f_\nu(D) T_{\sigma^{2,2,k}}(f,g)(x)| &\lesssim 2^{-Jk} \Mtau(\widetilde{\f}_k(D)f)(x) \Mtau(g^*)(x)\\
 &\quad\times \int_{\re^{3n}} \frac{2^{\nu n}}{(1+ 2^\nu |x- w|)^N} \frac{(1 + 2^k |x-y|)^M (1 + 2^k |x-z|)^M}{(1+ |w-y|)^{N/2} (1+ |w-z|)^{N/2}} \, dw \dy \dz\\
&\lesssim 2^{-Jk} \Mtau(\widetilde{\f}_k(D)f)(x) \Mtau(g^*)(x)\\
&\quad \times \int_{\rn}  \frac{2^{\nu n} 2^{2kM}}{(1+ 2^\nu |x- w|)^N}(1 +  |x-w|)^{2M} \, dw\quad \forall x\in\rn,k,\nu\in\na_0, k\ge 4.
\end{align*}
Since
\begin{align*}
\int_{\rn}  \frac{2^{\nu n} 2^{2kM}}{(1+ 2^\nu |x- w|)^N}(1 +  |x-w|)^{2M} \, dw\lesssim 2^{2kM}\quad \forall x\in\rn, k,\nu\in\na_0,
\end{align*}
we then get 
\begin{equation}\label{eq:fnuTbound}
|\f_\nu(D) T_{\sigma^{2,2,k}}(f,g)(x)|\lesssim  2^{-k (J-2M)} \Mtau(\widetilde{\f}_k(D)f)(x) \Mtau(g^*)(x)\quad \forall x\in\rn, k,\nu\in\na_0, k\ge 4.
\end{equation}
 Using that  the Fourier transform of  $T_{\sigma^{2,2,k}}(f,g)$ is supported in $\{\zeta \in \rn: |\zeta| \leq 2^{k+5} \},$ choosing $\varepsilon>\max(0,s_1)$ and applying   \eqref{eq:fnuTbound},  we have
\begin{align*}
 \norm{T_{\sigma^{2,2}}(f,g)}{\B{s_1}{p}{q}} &=  \left(\sum\limits_{\nu =0}^\infty 2^{\nu s_1q}\norm{\f_\nu(D) \left(\sum\limits_{k=4}^\infty T_{\sigma^{2,2,k}}(f,g)\right)}{L^p}^q \right)^\frac{1}{q}\\
&\le \left(\sum\limits_{\nu =0}^{\infty}  2^{\nu  s_1 q} \norm{\sum\limits_{k=\max(4,\nu -5)}^\infty \left| \f_\nu(D) T_{\sigma^{2,2,k}}(f,g)\right|}{L^p}^q \right)^\frac{1}{q}\\
& \lesssim \left(\sum\limits_{\nu =0}^{\infty}  2^{\nu  s_1q} \norm{\sum\limits_{k=\max(4,\nu -5)}^\infty 2^{-k (J-2M)}  \Mtau(\widetilde{\f}_k(D)f) \Mtau(g^*)}{L^p}^q \right)^\frac{1}{q}\\
& \lesssim \left(\sum\limits_{\nu =0}^{\infty}  2^{\nu  (s_1-\eps) q} \norm{\sum\limits_{k=\max(4,\nu -5)}^\infty 2^{-k (J-2M-\eps)}  \Mtau(\widetilde{\f}_k(D)f) \Mtau(g^*)}{L^p}^q \right)^\frac{1}{q}\\
&\lesssim\norm{ \Mtau(g^*) \sum\limits_{k=0}^\infty 2^{-k (J-2M-\eps)}  \Mtau(\widetilde{\f}_k(D)f) }{L^p}\\
&\lesssim \left(\sum\limits_{k=0}^\infty 2^{-k (J-2M-\eps)\widetilde{p}_1} \norm{\Mtau(\widetilde{\f}_k(D)f)}{L^{p_1}}^{\widetilde{p}_1} \right)^{\frac{1}{\widetilde{p}_1}}\norm{\Mtau(g^*)}{L^{p_2}},
\end{align*}
where $\widetilde{p_1}=\min(1,p_1).$ Using that
$$
\left(\sum\limits_{k=0}^\infty 2^{-k (J-2M-\eps)\widetilde{p}_1} \norm{\Mtau(\widetilde{\f}_k(D)f)}{L^{p_1}}^{\widetilde{p}_1} \right)^{\frac{1}{\widetilde{p}_1}}\lesssim \left( \sum\limits_{k=0}^\infty 2^{-k  (J-2M -2\eps)q} \norm{\Mtau(\widetilde{\f}_k(D)f)}{L^{p_1}}^q  \right)^\frac{1}{q},
$$
along with  the boundedness properties of $\Mtau$ with $0<\tau<\min(1,p_1,p_2),$ we obtain
\begin{align*}
\norm{T_{\sigma^{2,2}}(f,g)}{\B{s_1}{p}{q}} & \lesssim \left( \sum\limits_{k=0}^\infty 2^{-k  (J-2M -2\eps)q} \norm{\Mtau(\widetilde{\f}_k(D)f)}{L^{p_1}}^q  \right)^\frac{1}{q}\norm{\Mtau(g^*)}{L^{p_2}}\\
& \lesssim \left( \sum\limits_{k=0}^\infty 2^{-k  (J-2M -2\eps)q} \norm{\widetilde{\f}_k(D)f)}{L^{p_1}}^q  \right)^\frac{1}{q} \norm{g^*}{L^{p_2}}\\
& \lesssim \norm{f}{\B{-J +2M +2\eps}{p_1}{q}} \norm{g}{h^{p_2}} \lesssim  \norm{f}{\B{s_2}{p_1}{\bar{q}}} \norm{g}{h^{p_2}},
\end{align*}
provided that we take $J$ large enough so that $-J +2M +2\eps < s_2$ and where $h^{p_2}$ should be replaced by $L^\infty$ if $p_2=\infty.$
\end{proof}




\subsection{Estimates for $T_{\sigma^3}$}

In this section we prove the bounds for $T_{\sigma^3}$. Let $m,$ $p_1,$ $p_2,$ $p,$ $q,$   $\rho$ and $\delta$ be as in the hypotheses of Theorem~\ref{thm:main1} and $\sigma\in BS^{m(\rho,p_1,p_2)}_{\rho,\delta}.$ We decompose $\sigma^3,$ defined in Section~\ref{sec:decomp}, as $\sigma^3=\sigma^{3,1}+\sigma^{3,2}$ where
\begin{align*}
\sigma^{3,1}:=  \sum\limits_{k=4}^\infty \sum\limits_{j=0}^{k-4} \sum\limits_{\ell =0}^{k-4} \sigma_{j,k,\ell}\quad \text{and}\quad \sigma^{3,2}:= \sum\limits_{k=4}^\infty \sum\limits_{j=k-3}^k \sum\limits_{\ell =0}^{k-4} \sigma_{j,k,\ell}.
\end{align*}



We next state and prove a lemma that is useful in the proof of Lemma~\ref{lem:T3}, which is presented at the end of this section.

\begin{lemma}\label{lem:Tsk:bound} Let $0<p<\infty$ and $0<p_1,p_2\le \infty$  be such that $\hcline$ and $0\le \delta\le \rho<1;$ assume $\{\sigma_k\}_{k\in \na_0}$ is a sequence in $BS^{(\rho,p_1,p_2)}_{\rho,\delta}$  with constants uniform in $k$ and satisfies 
\begin{equation*}
\supp(\sigma_k)\subset \{(x,\xi,\eta)\in \re^{3n}: \abs{\xi}+\abs{\eta}\sim 2^k\},
\end{equation*}
with constants uniform in $k,$  where  $\abs{\xi}+\abs{\eta}\sim 2^k$ must be replaced by $\abs{\xi}+\abs{\eta}\lesssim 1$ if $k=0.$
Then 
\begin{equation*}
\norm{T_{\sigma_k}(f,g)}{L^p}\lesssim \norm{f}{h^{p_1}}\norm{g}{h^{p_2}}\quad \forall f,g\in\sw,k\in\na_0,
\end{equation*}
where the local Hardy spaces $h^{p_1}$ and $h^{p_2}$ must be replaced by $L^\infty$ if $p_1=\infty$ or $p_2=\infty.$
\end{lemma}

\begin{proof} Let $p,$ $p_1,$ $p_2,$  $\rho,$  $\sigma_k,$ $f$ and $g$ be as in the hypotheses of the lemma.  

Define
\[
\Sigma_k(x,\xi,\eta)=\sigma_k(2^{-\rho k}x, 2^{\rho k}\xi, 2^{\rho k} \eta);
\]
it easily follows that 
\begin{equation}\label{eq:sk:Sk}
T_{\sigma_k}(f,g)(x)=T_{\Sigma_k} (f_k,g_k)(2^{\rho k}x),
\end{equation}
where $f_k(x)=f(2^{-\rho k}x)$ and $g_k(x)=g(2^{-\rho k}x).$ 

We next check that $\Sigma_k\in BS^{m(0,p_1,p_2)}_{0,0}$ with constants uniform in $k.$ Note that $\abs{\xi}+\abs{\eta}\sim 2^{(1-\rho)k}$ for $(x,\xi,\eta)\in \supp(\Sigma_k)$ and $k\in\na,$ and $\abs{\xi}+\abs{\eta}\lesssim 1$ for $(x,\xi,\eta)\in \supp(\Sigma_0).$ Using that $\sigma_k\in BS^{m(\rho,p_1,p_2)}_{\rho,\rho}$ with constants uniform in $k$ and assuming $(x,\xi,\eta)\in \supp(\Sigma_k),$ we have  
\begin{align*}
|\partial_x^\alpha\partial^\beta_\xi\partial^\gamma_\eta \Sigma_k(x,\xi, \eta)|&\lesssim (1+\abs{2^{\rho k}\xi}+\abs{2^{\rho k}\eta})^{m(\rho,p_1,p_2)+\rho\abs{\alpha}-\rho\abs{\beta+\gamma}}2^{-\rho k(\abs{\alpha}-\abs{\beta+\gamma})}\\
&\lesssim 2^{k(m(\rho,p_1,p_2)+\rho\abs{\alpha}-\rho\abs{\beta+\gamma})}2^{-\rho k(\abs{\alpha}-\abs{\beta+\gamma})}=2^{k(1-\rho)m(0,p_1,p_2)}\\
&\sim (1+\abs{\xi}+\abs{\eta})^{m(0,p_1,p_2)}.
\end{align*}
For the sake of notation assume $p_1$ and $p_2$ are finite; the argument below works as well replacing $h^{p_1}$ or $h^{p_2}$ with $L^\infty$ if $p_1=\infty$ or $p_2=\infty,$ respectively.  By Miyachi--Tomita~\cite[Theorem 1.1]{MR3179688}, we have  
\begin{equation*}
\norm{T_{\Sigma_k}(f,g)}{h^p}\lesssim \norm{f}{h^{p_1}}\norm{g}{h^{p_2}}\quad \forall k\in\na_0.
\end{equation*}
Since $T_{\Sigma_k}(f,g)\in \sw$ for $f,g\in\sw,$ we can apply \eqref{eq:Lp:hp} to get 
\begin{equation*}
\norm{T_{\Sigma_k}(f,g)}{L^p}\lesssim \norm{f}{h^{p_1}}\norm{g}{h^{p_2}}\quad \forall k\in\na_0.
\end{equation*}
Recalling \eqref{eq:sk:Sk}, applying the estimate above and \eqref{eq:hp:scale}, we then obtain
\begin{align*}
\norm{T_{\sigma_k}(f,g)}{L^p}&=2^{-\rho k \frac{n}{p}}\norm{T_{\Sigma_k}(f_k,g_k)}{L^p}\lesssim 2^{-\rho k \frac{n}{p}} \norm{f_k}{h^{p_1}}\norm{g_k}{h^{p_2}}\\
&\le  2^{-\rho k \frac{n}{p}} 2^{\rho k \frac{n}{p_1}}\norm{f}{h^{p_1}} 2^{\rho k \frac{n}{p_2}}\norm{g}{h^{p_2}}=\norm{f}{h^{p_1}} \norm{g}{h^{p_2}} \quad \forall k\in\na_0. 
\end{align*}
\end{proof}




\begin{proof}[Proof of Lemma~\ref{lem:T3}]  Let $p,$ $p_1,$ $p_2,$ $q,$ $s,$  $\rho,$  $\sigma,$ $f$ and $g$ be as in the hypotheses of the lemma. 
For the sake of notation, assume $p_1$ and $p_2$ are finite; the argument given below works as well with $h^{p_1}$ or $h^{p_2}$ replaced with $L^\infty$ if $p_1=\infty$ or $p_2=\infty,$ respectively.

For $k\in\na_0,$ define $\sigma_k(x,\xi,\eta)=\sigma(x,\xi,\eta)\fk(\xi),$ where $\fk$ is as in Section~\ref{sec:decomp}; then $T_\sigma=\sum_{k=0}^\infty T_{\sigma_k}.$ Since  $\{\sigma_k\}_{k\in\na_0}$ satisfies the hypotheses of Lemma~\ref{lem:Tsk:bound}, we  have 
\begin{equation}\label{eq:Tk:bound}
\norm{T_{\sigma_k}(f,g)}{L^p}\lesssim \norm{f}{h^{p_1}}\norm{g}{h^{p_2}}\quad \forall k\in\na_0.
\end{equation}
 The conditions on the supports of $\sigma$ and $\widehat{\sigma}^1$ imply that 
\begin{align*}
&\supp(\widehat{T_{\sigma_k}(f,g)}) \subset \{\zeta\in\rn: \abs{\zeta}\lesssim 2^k\}\quad \text{if } A\ge\fr{1}{2},\\
&\supp(\widehat{T_{\sigma_k}(f,g)}) \subset \{\zeta\in\rn: \abs{\zeta}\sim 2^k\} \quad \text{if } 0<A<\fr{1}{2},
\end{align*}
with constants independent of  $k,$  $f$ and $g$ (in the second inclusion $\abs{\zeta}\sim 2^k$ must be replaced with $\abs{\zeta}\lesssim 1$ if $k=0$).  Indeed,
\begin{align*}
\widehat{T_{\sigma_k}(f,g)}(\zeta)&=\int_{\rn}\left(\int_{\rtn}\sigma_k(x,\xi,\eta)\fhat(\xi)\ghat(\eta)\eixxe\dxi\deta\right)e^{-2\pi i x\cdot\zeta}\dx\\
&=\int_{\rtn} \fhat(\xi)\ghat(\eta) \widehat{\sigma_k}^1(\zeta-\xi-\eta,\xi,\eta)\dxi\deta.
\end{align*}
If $\zeta\in \supp(\widehat{T_{\sigma_k}(f,g)}),$ in view of \eqref{eq:supp1}, \eqref{eq:supp2} and the definition of $\sigma_k,$ there exist $\xi,\eta\in\rn$ such that $2^{k-1}\le \abs{\xi}\le 2^{k+1}$ ($\abs{\xi}\le 2$ if $k=0$), $\abs{\eta}\le A\abs{\xi}$ and $\abs{\zeta-\xi-\eta}\le A \abs{\xi}.$  This leads to 
\[
\abs{\zeta}\le \abs{\zeta-\xi-\eta}+\abs{\xi}+\abs{\eta}\le (2A+1) \abs{\xi}\lesssim 2^k\quad\forall k\in\na_0.
\]
and
\begin{equation*}
\abs{\zeta}\ge \abs{\xi}-\abs{\eta}-\abs{\zeta-\xi-\eta}\ge(1-2A)\abs{\xi}\ge (1-2A) 2^{k-1} \quad\forall k\in\na, 0<A<\fr{1}{2}.
\end{equation*}
Applying Lemma~\ref{lem:nikol}, recalling the definition of $\widetilde{\f}_k$ given at the beginning of Section~\ref{sec:T1} and using \eqref{eq:Tk:bound} and \eqref{eq:besovlh}, we obtain
\begin{align*}
\norm{T_\sigma(f,g)}{\B{s}{p}{q}}&\lesssim \left(\sum_{k=0}^\infty 2^{ksq}\norm{T_{\sigma_k}(f,g)}{L^p}^q\right)^{\frac{1}{q}}=\left(\sum_{k=0}^\infty 2^{ksq}\norm{T_{\sigma_k}(\widetilde{\f}_k(D)f,g)}{L^p}^q\right)^{\frac{1}{q}}\\
&\lesssim\left(\sum_{k=0}^\infty 2^{ksq}\norm{\widetilde{\f}_k(D)f}{h^{p_1}}^q\right)^{\frac{1}{q}}\norm{g}{h^{p_2}}\sim \norm{f}{\B{s}{p_1}{q}}\norm{g}{h^{p_2}}.
\end{align*}
\end{proof}
\section{Proof of Theorem \ref{thm:main1} }

\section{Proof of Corollary \ref{}}


\section{•}


