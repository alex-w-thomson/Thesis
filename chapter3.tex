% +--------------------------------------------------------------------+
% | Sample Chapter 3
% +--------------------------------------------------------------------+

\cleardoublepage

% +--------------------------------------------------------------------+
% | Replace "This is Chapter 3" below with the title of your chapter.
% | LaTeX will automatically number the chapters.                      
% +--------------------------------------------------------------------+

\chapter{Bilinear H\"ormander Classes of Critical Order}
\label{makereference3}

\section{Introduction}

In this chapter we obtain Leibniz-type rules for bilinear multiplier operators associated to symbols in the H\"ormander classes of critical order in the setting of local Hardy spaces. First, we will discuss the bilinear H\"ormander classes $BS^m_{\rho,\delta}$ and what it means for these symbols to be of critical order. Given $0\leq \delta \leq \rho \leq 1$ and $m\in\re$, a complex-valued function $\sigma = \sigma(x,\xi,\eta)$, $x,\xi,\eta \in \rn$, belongs to the bilinear H\"ormander class $BS^m_{\rho,\delta}$ if for any multiindices $\alpha,\beta,\gamma \in \mathbb{N}^n_0$ there exists a positive constant $C_{\alpha,\beta,\gamma}$ such that 
\begin{equation}\label{def:Bmrd}
|\partial_x^\alpha \partial_\xi^\beta \partial_\eta^\gamma \sigma(x, \xi, \eta)| \leq C_{\alpha, \beta, \gamma} (1+|\xi|+|\eta|)^{m +\delta \abs{\alpha}-\rho(\abs{\beta+\gamma})} \quad \forall x, \xi, \eta \in \rn.
\end{equation}
Then for any $\sigma \in BS^m_{\rho,\delta},$ the bilinear pseudodifferential operator $T_\sigma$ associated to $\sigma$ is defined as in \ref{psydo}. There has been a considerable amount of effort devoted to studying bilinear pseudodifferential operators associated to symbols in the classes $BS^m_{\rho,\delta}$. One fundamental aspect of the study of such symbols is their symbolic calculus for the transposes of operators associated to them. This was established in the works B\'enyi-Torres \citep{MR1986065} and B\'enyi-Maldonado-Naibo-Torres \citep{MR2660466}. Another important aspect of the study of these operators is their boundedness properties in a variety of function spaces. Operators associated to symbols in $BS^0_{1,0}$ can be realized as Calder\'on-Zygmund operators. As a consequence operators associated to symbols in $BS^0_{1,0}$ are bounded from $L^{p_1}(\rn) \times L^{p_2}(\rn)$ to  $L^p(\rn)$ for  $1 < p_1, p_2 < \infty$ and $1/2<p <\infty$ related through $\hcline.$ These operators also satisfy the endpoint mappings $L^\infty(\rn) \times L^\infty(\rn) \rightarrow BMO(\rn)$ and $L^1(\rn) \times L^1(\rn) \rightarrow L^{1/2, \infty}(\rn)$. For the development of the Calder\'on-Zygmund theory see Coifman-Meyer \citep{MR518170}, Kenig-Stein \citep{MR1713146}, and Grafakos-Torres \citep{MR1880324}.

\section{Preliminaries}

\section{•}
