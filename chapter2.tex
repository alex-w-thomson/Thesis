\cleardoublepage

\chapter{Weighted Leibniz-type rules with applications to scattering properties of PDEs}
%\label{makereference2}



In this chapter we discuss bilinear multiplier operators associated to Coifman-Meyer multipliers and Leibniz-type rules in the settings of weighted Triebel-Lizorkin and Besov spaces.  Additionally we obtain applications of these results to scattering properties of certain systems of parial differential equations. One of the main results in this chapter is in the setting of Triebel-Lizorkin spaces based in weighted Lebesgue spacs and Hardy spaces. It is as follows. 

\begin{theorem}\label{thm:CM:TL:B}  For $m \in \re,$ let $\sigma(\xi,\eta),$ $\xi,\eta\in\rn,$ be a Coifman-Meyer multiplier of order $m.$ Consider  $0 < p, p_1, p_2  \le \infty$  such that $\hcline$ and  $0 < q \leq \infty;$ let  $w_1,w_2\in A_\infty$ and set $w=w_1^{{p}/{p_1}} w_2^{{p}/{p_2}}.$ 
If $0 < p,p_1,p_2 < \infty$ and  $s > \tau_{p,q}(w),$  it holds that
\begin{equation}\label{KP:CM:TL}
\norm{T_\sigma(f,g)}{\tlw{p}{s}{q}{w}} \lesssim \norm{f}{\tlw{p_1}{s+m}{q}{w_1} } \norm{g}{H^{p_2}(w_2)} +  \norm{f}{H^{p_1}(w_1)}   \norm{g}{\tlw{p_2}{s+m}{q}{w_2} } \quad \forall f, g \in \swz.
\end{equation}
If $0< p, p_1,p_2\leq \infty$ and $s > \tau_p(w)$, it holds that
\begin{equation}\label{KP:CM:B}
\norm{T_\sigma(f,g)}{\besw{p}{s}{q}{w}} \lesssim \norm{f}{\besw{p_1}{s+m}{q}{w_1} } \norm{g}{H^{p_2}(w_2)} +  \norm{f}{H^{p_1}(w_1)}   \norm{g}{\besw{p_2}{s+m}{q}{w_2} } \quad \forall f, g \in \swz,
\end{equation}
where the Hardy spaces $H^{p_1}(w_1)$ and $H^{p_2}(w_2)$ must be replaced by $L^\infty$ if $p_1=\infty$ or $p_2=\infty,$ respectively.

If $w_1=w_2$ then different pairs of $p_1, p_2$ can be used on the right-hand sides of \eqref{KP:CM:TL} and \eqref{KP:CM:B}; moreover, if $w\in A_\infty,$ then 
\begin{equation}\label{KP:CM:TL2}
\norm{T_\sigma(f,g)}{\tlw{p}{s}{q}{w}} \lesssim \norm{f}{\tlw{p}{s+m}{q}{w} } \norm{g}{L^\infty} +  \norm{f}{L^\infty}   \norm{g}{\tlw{p}{s+m}{q}{w}} \quad \forall f, g \in \swz,
\end{equation}
where $0<p<\infty,$ $0<q\le\infty$ and $s>\tau_{p,q}(w).$
\end{theorem}

In the following sections the function spaces and multipliers used in the hypotheses are defined and discussed. We will then remark on corollaries to Theorem (\ref{thm:CM:TL:B}) and their connection to the Leibniz rules in the previous chapter.  theorem and the proof of Theorem (\ref{thm:CM:TL:B}). The proof is quite flexible and can be readily adapted to Triebel-Lizorkin and Besov spaces based in other function spaces. 

\section{Definitions}

\subsection{Coifman-Meyer Multipliers}

The symbols used in Theorem (\ref{thm:CM:TL:B}) and results later in this chapter are Coifman-Meyer multipliers. Such multipliers are defined as follows.

\begin{dfn}\label{CM_def}
For $m\in\rn$, a smooth function $\sigma = \sigma(\xi,\eta)$, $\xi,\eta\in\rn$, is a \textit{Coifman-Meyer multiplier} of order $m$ if for all multi-indices $\alpha,\beta\in\mathbb{N}^n_0$ there exists a positive constant $C_{\alpha,\beta}$ such that 
\begin{equation}\label{eq:CMm}
|\partial_\xi^\alpha \partial_\eta^\beta \sigma(\xi, \eta)| \leq C_{\alpha, \beta} (|\xi|+|\eta|)^{m -(\abs{\alpha}+ \abs{\beta})} \quad \forall (\xi, \eta) \in \re^{2n} \setminus \{(0,0)\}.
\end{equation}
\end{dfn}
Operators associated to these multipliers have been widely studied. For instance in Grafakos-Torres \cite{MR1880324} operators associated to Coifman-Meyer multipliers were studied because of their connection to a larger class of operators called Calder\'on-Zygmund operators. In particular it holds that 
\[ \norm{T_\sigma(f,g)}{L^p}\lesssim \norm{f}{L^{p_1}}\norm{g}{L^{p_2}} \]
where $\sigma$ is a Coifman-Meyer multiplier, $\frac{1}{p} = \frac{1}{p_1} + \frac{1}{p_2}$, and $1<p_1,p_2<\infty$. 
We also note that in particular Coifman-Meyer multipliers of order $m$ belong to the bilinear H\"ormander class $\dot{BS}^m_{1,1}$. These symbols and operators associated to them will be discussed in the following chapter. 

For the proof of Theorem (\ref{thm:CM:TL:B}) we will use a the following decomposition of Coifman-Meyer multipliers. Let $\sigma$ be a Coifman--Meyer multiplier of order $m.$
Fix $\Psi \in \sw$ such that 
$$
\supp(\widehat{\Psi}) \subseteq \{ \xi \in \rn : \fr{1}{2} < |\xi| < 2 \} \quad\text{ and }\quad
\sum_{j\in\ent} \widehat{\Psi}(2^{-j}\xi) = 1 \,\,\forall \xi \in \rn \setminus \{ 0 \};
$$
define $\Phi \in \sw$ so that
$$
\widehat{\Phi}(0) \coloneqq 1,\quad \widehat{\Phi}(\xi) \coloneqq \sum_{j \le 0} \widehat{\Psi}(2^{-j}\xi)\quad \forall \xi\in\rn\setminus \{0\}.$$
For $a,b\in\rn,$ $\Do{\tau_a \Psi }{j} f$ and  $\So{ \tau_a \Phi }{j} f$ satisfy $\widehat{\Do{\tau_a \Psi }{j} f}(\xi)=\widehat{\tau_a\Psi}(2^{-j}\xi)\widehat{f}(\xi)=e^{2\pi i 2^{-j}\xi\cdot a} \widehat{\Psi}(2^{-j}\xi)\widehat{f}(\xi)$ and   $\widehat{\So{\tau_b \Phi }{j} f}(\xi)=\widehat{\tau_b\Phi}(2^{-j}\xi)\widehat{f}(\xi)=e^{2\pi i 2^{-j}\xi\cdot b} \widehat{\Phi}(2^{-j}\xi)\widehat{f}(\xi).$ 
By the work of Coifman and Meyer in \cite{MR518170},  given $N\in \na$ such that $N>n,$ it follows that $T_\sigma= T_\sigma^1 + T_\sigma^2$, where, for  $f\in \swz$ ($f\in \sw$ if $m\ge 0$) and $g\in\sw,$
\begin{align}\label{eq:decompT1}
T_\sigma^1(f,g)(x) &= \sum_{a,b\in \ent^n} \frac{1}{(1+|a|^2+|b|^2)^N} \sum_{j\in\ent} \C_j(a,b) \,(\Do{ \tau_a \Psi }{j} f)(x)\, (\So{ \tau_b \Phi }{j} g )(x)\quad \forall x\in \rn,
\end{align}
 the coefficients $\C_j(a,b)$   satisfy
\begin{equation}\label{eq:cjbound}
\abs{\C_j(a,b)}\lesssim 2^{jm}\quad \forall a,b\in\ent, j\in \ent,
\end{equation}
with the implicit constant depending on $\sigma,$ and an analogous expression holds for $T_\sigma^2$ with the roles of $f$ and $g$ interchanged. 

If $\sigma$ is an inhomogeneous Coifman--Meyer multiplier of order $m,$ a similar decomposition to \eqref{eq:decompT1} follows  with the summation in $j\in\naz$ rather than $j\in\ent,$  with  $\Delta_0^{\tau_a \Psi}$ replaced by $S_0^{\tau_a\Phi}$ and for $f,g\in \sw.$

\begin{remark}\label{re:numderiv1} For the formula \eqref{eq:decompT1}  and its corresponding counterpart for $T^2_\sigma$ to hold, the condition \eqref{eq:CMm} on the derivatives of $\sigma$ is only needed for multi-indices $\alpha$ and $\beta$ such that  $\abs{\alpha+\beta}\le 2N$
\end{remark}

\end{remark}

\subsection{Function spaces}

\subsubsection{Weighted spaces}

\begin{dfn}
A \textit{weight} $w(x)$ defined on $\rn$ is a nonnegative, locally integrable function such that $0<w(x)<\infty$ for almost every $x\in\rn$.
\end{dfn}

Given a weight $w(x)$ and $0<p<\infty$ we define the weighted Lebesgue space $L^p(w)$ as the space of all measurable functions satisfying 
\[ \norm{f}{L^p(w)} = \left(\int_\rn |f(x)|^p w(x)dx\right)^{\frac{1}{p}} < \infty. \]
In the case that $p=\infty$ we define $L^\infty (w) = L^\infty$. 

The specific classes of weights in the hypotheses of our results are Muckenhoupt weights. 
\begin{dfn}
For $1<p<\infty$ the \textit{Muckenhoupt class} $A_p$ consists of all weights $w$ on $\rn$ satisfying 
 \begin{equation}\label{weight_condition}
 \sup_B\left(\frac{1}{\abs{B}}\int_Bw(x)\dx\right)\left(\frac{1}{\abs{B}}\int_Bw(x)^{-\frac{1}{p-1}}\dx\right)^{p-1}<\infty,
 \end{equation}
where the supremum is taken over all Euclidean balls $B\subset\rn$ and $|B|$ is the Lebesgue measure of $B$. For $p=\infty$ we define $A_\infty = \cup_{1<p} A_p$. 
\end{dfn}

From this definition it follows that $A_p \subset A_q$ when $p\leq q$. For a $w\in A_\infty$ we set $\tau_w = \inf\{\tau \in (1,\infty]: w\in A_\tau \]. The condition (\ref{weight_condition}) is motivated by the following fact: for $f\in L^p(w)$ the Hardy-Littlewood maximal function $\mathcal{M}(f)(x)$ is bounded on $L^p(w)$ if and only if $w\in A_p$. That is for for $1<p<\infty$ $w\in A_p$ if and only if 
\[\norm{\mathcal{M}(f)}{L^p (w)} \lesssim \norm{f}{L^p (w)}. \]
Later in this chapter we will use the maximal function $\mathcal{M}_r (f) = \left(\mathcal{M}(|f|^r)\right)^{\frac{1}{r}}$. By the properties for the Hardy-Littlewood maximal function above it holds that for $0<r<p$ $\mathcal{M}_r$ is bounded on $L^p (w)$ for $w\in A_{p/r}$ and in this case $0<r<\frac{p}{\tau_w}$. The following theorem is a vector valued version of the previous statement called a weighted Fefferman-Stein inequality.

\begin{theorem}
If $0<p<\infty,$ $0<q\le \infty,$  $0<r <\min(p,q)$ and $w \in A_{p/r}$ (i.e. $0<r<\min(p/\tau_w,q)$), then for all sequences $\{f_{j}\}_{j\in\ent}$ of locally integrable functions defined on $\rn,$ we have
 \begin{equation*}
 \norm{\left(\sum_{j\in\ent}\abs{\M_r (f_j)}^q\right)^{\frac{1}{q}}}{\lebw{p}{w}}\lesssim
 \norm{\left(\sum_{j\in\ent}\abs{f_j}^q\right)^{\frac{1}{q}}}{\lebw{p}{w}},\label{eq:wFS}
 \end{equation*}
where the implicit constant depends on $r,$ $p,$ $q,$ and $w$ and the summation in $j$ should be replaced by the supremum in $j$ if $q=\infty.$
\end{theorem}
For more detail on the Muckenhoupt classes see Grafakos \cite{MR3243741}.

\subsubsection{Triebel-Lizorkin and Besov spaces}\label{TL_B_section}
Here we describe the function spaces in which (\ref{thm:CM:TL:B}) is based and some properties of such spaces. 

Let $\psi$ and $\varphi$ be functions in $\sw(\rn)$ satisfying the following conditions:
\begin{itemize}
\item supp$(\widehat{\psi})\subset\{\xi\in\rn : \frac{1}{2} < |\xi| <2\}$,
\item $|\widehat{\psi}(\xi)|>c$ for all $\xi$ such that $\frac{3}{5} < |\xi| < \frac{5}{3}$ for some $c>0$,
\item supp$(\widehat{\varphi}\subset \{\xi\in\rn : |\xi| < 2\}$,
\item $|\widehat{\varphi}(\xi)|>c$. 
\end{itemize}
For $\psi$ supported in an annulus and $j\in\ent$ we define the operator $\Delta^\psi_j(f)$ through its Fourier transform as \[\widehat{\Delta^\psi_j (f)}(\xi) = \psi(2^{-j}\xi)\widehat{f}(\xi)\]
For such $\psi$ we define the homogeneous Triebel-Lizorkin and Besov spaces as follows.

\begin{dfn}\label{TL_B_def}
Let $s\in\mathbb{R}$, $0<p<\infty$, and $0<q\leq\infty$.
\begin{itemize}
\item The weighted \textit{homogeneous Triebel-Lizorkin space} $\tlw{p}{s}{q}{w}$ consists of all $f\in \swp/\mathcal{P}(\rn)$ such that 
\begin{equation*}
\norm{f}{\tlw{p}{s}{q}{w}}=\norm{\left(\sum_{j\in\ent}(2^{sj}|\Delta^\psi_jf|)^q\right)^{\frac{1}{q}}}{\lebw{p}{w}}<\infty.
\end{equation*}
\item The weighted \textit{homogeneous Besov space} $\besw{p}{s}{q}{w}$ consists of all $f\in \swp/\mathcal{P}(\rn)$ such that 
\begin{equation}
\norm{f}{\besw{p}{s}{q}{w}} = \left(\sum_{j\in\int} (2^{js}\norm{\Delta^\psi_jf}{L^p(w)})^q\right)^{\frac{1}{q}} < \infty.
\end{equation}
\end{itemize}
\end{dfn}

Given $\varphi$, $\psi$, $S^\varphi_0$, and $\Delta^\psi_j$ as above the weighted inhomogeneous Triebel-Lizorkin and Besov spaces are defined as follows. 

\begin{dfn}\label{ITL_B_def}
Let $s\in\mathbb{R}$, $0<p<\infty$, and $0<q\leq\infty$.
\begin{itemize}
\item The weighted \textit{inhomogeneous Triebel-Lizorkin space} $\itlw{p}{s}{q}{w}$ is the class of all $f\in\sw'$ such that
\begin{equation*}
\norm{f}{\itlw{p}{s}{q}{w}}= \norm{S^\varphi_0 f}{L^p(w)} + \norm{\left(\sum_{j\in\naz}(2^{sj}|\Delta^\psi_jf|)^q\right)^{\frac{1}{q}}}{\lebw{p}{w}}<\infty.
\end{equation*}
\item The weighted \textit{inhomogeneous Besov space} $\besw{p}{s}{q}{w}$ is the class of all $f\in\sw'$ such that 
\begin{equation}
\norm{f}{\ibesw{p}{s}{q}{w}} = \norm{S^\varphi_0 f}{L^p(w)} + \left(\sum_{j\in\naz} (2^{js}\norm{\Delta^\psi_jf}{L^p(w)})^q\right)^{\frac{1}{q}} < \infty.
\end{equation}
\end{itemize}
\end{dfn}

The definitions above are independent of the choice of $\varphi$ and $\psi$. These spaces are generally quasi-Banach spaces and if $1\leq p,q <\infty$ they are Banach spaces. These spaces provide a framework to study a variety of other spaces such as Lebesgue, Hardy, Sobolev, and BMO spaces with a unified approach. For a detailed overview of the development of these spaces see Triebel [books 1-3]. 

For certain $s$, $p$, and $q$ as in the definitions (\ref{TL_B_def}) and (\ref{ITL_B_def}) Triebel-Lizorkin and Besov spaces coincide with other well known function spaces. For instance we have the following equivalences where the function spaces are equivalent in norm

\begin{align*}
& \itlw{p}{0}{2}{w} \equiv H^p (w) \text{ for } 0<p<\infty, \quad w\in A_\infty, \\
& \itlw{p}{0}{2}{w} \equiv L^p(w) \equiv H^p(w) \text{ for } 1<p<\infty, \quad w\in A_p, \\
& \itlw{p}{s}{2}{w} \equiv \dot{W}^{s,p}(w) \text{ for } 1<p<\infty, \quad w\in A_p.
\end{align*}

Additionally by the lifting property of Triebel-Lizorkin and Besov spaces Theorem (\ref{thm:CM:TL:B}) and results like it in this chapter can be seen as Leibniz-type rules as in Chapter 1. For weighted Triebel-Lizorkin spaces the lifting property is as follows: for $s$, $p$, and $q$ as in (\ref{ITL_B_def}) and (\ref{TL_B_def}) and $w\in A_\infty$ we have that 

 \begin{align*}
 & \norm{f}{\tlw{p}{s}{q}{w}}\simeq\norm{D^s f}{\tlw{p}{0}{q}{w}} \quad \text{ and } \quad \norm{f}{\itlw{p}{s}{q}{w}}\simeq\norm{J^s f}{\itlw{p}{0}{q}{w}}.
 \end{align*}
The corresponding statement for Besov spaces is: for $s$, $p$, and $q$ as in (\ref{ITL_B_def}) and (\ref{TL_B_def}) and $w\in A_\infty$ we have that 
 \begin{align*}
 & \norm{f}{\besw{p}{s}{q}{w}}\simeq\norm{D^s f}{\besw{p}{0}{q}{w}} \quad \text{ and } \quad \norm{f}{\ibesw{p}{s}{q}{w}}\simeq\norm{J^s f}{\ibesw{p}{0}{q}{w}}.
 \end{align*}
 

  

\subsubsection{Nikol'skij representations for weighted homogeneous and inhomogeneous Triebel-Lizorkin and Besov spaces}

An important tool for the proof of Theorem (\ref{thm:CM:TL:B}) is the Nikol'skij representation for weighted Triebel-Lizorking and Besov spaces. Here we state a weighted version of \cite[Theorem 3.7]{MR837335} (see also \cite[Section 2.5.2]{MR3024598}). For  completeness, a sketch of its proof is outlined in Appendix~\ref{sec:appendix}.

\begin{theorem}\label{thm:Nikolskij:weighted} For $\A> 0,$ let $\{u_j\}_{j \in \ent} \subset \mathcal{S}'(\rn)$ be a sequence of tempered distributions such that
\begin{equation*}
\supp(\widehat{u_j}) \subset B(0, \A\, 2^j ) \quad \forall j \in \ent.
\end{equation*}
If $w\in A_\infty,$ then the following holds:  
\begin{enumerate}[(i)]
\item\label{item:thh:Nikolskij:TL} Let $0 < p < \infty$, $0 < q \leq \infty$ and $s > \tau_{p,q}(w)$. If $\norm{\{2^{js} u_j\}_{j\in\ent}}{L^p(w)(\ell^{q})} < \infty$, then the series $\sum_{j \in \ent} u_j$ converges in $\tlw{p}{s}{q}{w}$ (in $\mathcal{S}'_0(\rn)$ if $q=\infty$) and 
\begin{equation*}
\norm{\sum_{j \in \ent} u_j}{\tlw{p}{s}{q}{w}} \lesssim  \norm{\{2^{js} u_j\}_{j\in\ent}}{L^p(w)(\ell^{q})},
\end{equation*}
where the implicit constant depends only on $n,$ $\A,$ $s,$ $p$ and  $q.$  An analogous statement, with $j\in\naz,$ holds true for $\itlw{p}{s}{q}{w}$ (when $q=\infty,$  the convergence is in $\swp$).
\item\label{item:thh:Nikolskij:B} Let $0 < p, q \leq \infty$ and $s > \tau_p(w)$. If $\norm{\{2^{js} u_j\}_{j\in\ent}}{\ell^{q}(L^p(w))} < \infty$, then the series $\sum_{j \in \ent} u_j$ converges in  $\besw{p}{s}{q}{w}$ (in $\mathcal{S}'_0(\rn)$ if $q=\infty$) and 
\begin{equation*}
\norm{\sum_{j \in \ent} u_j}{\besw{p}{s}{q}{w}} \lesssim  \norm{\{2^{js} u_j\}_{j\in\ent}}{\ell^{q}(L^p(w))},
\end{equation*}
where the implicit constant depends only on $n,$ $\A,$ $s,$ $p$ and $q.$   An analogous statement, with $j\in\naz,$ holds true for $\ibesw{p}{s}{q}{w}$ (when $q=\infty,$  the convergence is in $\swp$).
\end{enumerate}
\end{theorem} 
 
 
 
 \section{Weighted Leibniz-type rules}
 
 \subsection{Homogeneous Leibniz-type rules}
 
 In the setting of weighted homogeneous Besov and Triebel-Lizorkin spaces we obtain the following Leibniz-type rule. As we will see in the corollaries to this result it improves the Leibniz-type rule (\ref{KPH:Lp}) and has extensions to weighted versions of (\ref{KPH:Lp}).
  
  \begin{theorem}\label{thm:CM:TL:B}  For $m \in \re,$ let $\sigma(\xi,\eta),$ $\xi,\eta\in\rn,$ be a Coifman-Meyer multiplier of order $m.$ Consider  $0 < p, p_1, p_2  \le \infty$  such that $\hcline$ and  $0 < q \leq \infty;$ let  $w_1,w_2\in A_\infty$ and set $w=w_1^{{p}/{p_1}} w_2^{{p}/{p_2}}.$ 
If $0 < p,p_1,p_2 < \infty$ and  $s > \tau_{p,q}(w),$  it holds that
\begin{equation}\label{KP:CM:TL}
\norm{T_\sigma(f,g)}{\tlw{p}{s}{q}{w}} \lesssim \norm{f}{\tlw{p_1}{s+m}{q}{w_1} } \norm{g}{H^{p_2}(w_2)} +  \norm{f}{H^{p_1}(w_1)}   \norm{g}{\tlw{p_2}{s+m}{q}{w_2} } \quad \forall f, g \in \swz.
\end{equation}
If $0< p, p_1,p_2\leq \infty$ and $s > \tau_p(w)$, it holds that
\begin{equation}\label{KP:CM:B}
\norm{T_\sigma(f,g)}{\besw{p}{s}{q}{w}} \lesssim \norm{f}{\besw{p_1}{s+m}{q}{w_1} } \norm{g}{H^{p_2}(w_2)} +  \norm{f}{H^{p_1}(w_1)}   \norm{g}{\besw{p_2}{s+m}{q}{w_2} } \quad \forall f, g \in \swz,
\end{equation}
where the Hardy spaces $H^{p_1}(w_1)$ and $H^{p_2}(w_2)$ must be replaced by $L^\infty$ if $p_1=\infty$ or $p_2=\infty,$ respectively.

If $w_1=w_2$ then different pairs of $p_1, p_2$ can be used on the right-hand sides of \eqref{KP:CM:TL} and \eqref{KP:CM:B}; moreover, if $w\in A_\infty,$ then 
\begin{equation}\label{KP:CM:TL2}
\norm{T_\sigma(f,g)}{\tlw{p}{s}{q}{w}} \lesssim \norm{f}{\tlw{p}{s+m}{q}{w} } \norm{g}{L^\infty} +  \norm{f}{L^\infty}   \norm{g}{\tlw{p}{s+m}{q}{w}} \quad \forall f, g \in \swz,
\end{equation}
where $0<p<\infty,$ $0<q\le\infty$ and $s>\tau_{p,q}(w).$
\end{theorem}
 
We note that if $m\geq 0$ then the above estimates hold for any $f,g\in\sw(\rn)$ when $\sw(\rn)$ is a subspace of the function spaces on the right-hand side. This is the case when $1<p_1,p_2<\infty$, $w_1\in A_{p_1}$, and $w_2\in A_{p_2}$ in (\ref{KP:CM:TL}) and (\ref{KP:CM:B}) and $w\in A_p$ for (\ref{KP:CM:TL2}). 

By the lifting property of weighted Besov and Triebel-Lizorkin spaces in section and their relation to weighted Hardy spacees in section (\ref{TL_B_section}) the estimates (\ref{KP:CM:TL}) and (\ref{KP:CM:B}) imply the following Leibniz-type rule for Coifman-Meyer multipliers of order zero.

\begin{corollary}\label{coro:KP:CM:Hardy}  Let $\sigma(\xi,\eta),$ $\xi,\eta\in\rn,$ be a Coifman-Meyer multiplier of order $0.$ 
Consider  $0 < p, p_1, p_2  < \infty$  such that $\hcline;$ let  $w_1,w_2\in A_\infty$ and set $w=w_1^{{p}/{p_1}} w_2^{{p}/{p_2}}.$ 
If  $s > \tau_p(w),$ it holds that
\begin{equation}\label{KP:CM:Hardy}
\norm{D^s(T_\sigma(f,g))}{H^p(w)} \lesssim \norm{D^s f}{H^{p_1}(w_1)} \norm{g}{H^{p_2}(w_2)} +  \norm{f}{H^{p_1}(w_1)}   \norm{D^s g}{H^{p_2}(w_2)} \quad \forall f, g \in \swz.
\end{equation}
If $w_1=w_2$ then different pairs of $p_1, p_2$ can be used on the right-hand side of \eqref{KP:CM:Hardy}; moreover, if $w\in A_\infty,$ then 
\begin{equation}\label{Kp:CM:Hardy2}
\norm{D^s(T_\sigma(f,g))}{H^p(w)} \lesssim \norm{D^s f}{H^{p}(w)} \norm{g}{L^\infty} +  \norm{f}{L^\infty}   \norm{D^s g}{H^{p}(w)} \quad \forall f, g \in \swz,
\end{equation}
where $0<p<\infty$ and $s>\tau_{p}(w).$

\end{corollary}

Corollary (\ref{coro:KP:CM:Hardy}) gives estimates related to those in Brummer-Naibo \citep{BrNa2017} where using different methods the following result was proven
\begin{theorem}\label{BrNa2017_THM}
if $\sigma$ is a Coifman-Meyer multiplier of order 0, $1<p_1,p_2\le \infty,$ $\frac{1}{2}<p<\infty,$ $\hcline,$  $w_1\in A_{p_1},$ $w_2\in A_{p_2},$ $w=w_1^{{p}/{p_1}} w_2^{{p}/{p_2}}$ and $s>\tau_p,$ then for all $f,g\in \sw$ it holds that
\begin{equation}\label{KP:CM:Lebesgue}
\norm{D^s(T_\sigma(f,g))}{L^p(w)} \lesssim \norm{D^s f}{L^{p_1}(w_1)} \norm{g}{L^{p_2}(w_2)} +  \norm{f}{L^{p_1}(w_1)}   \norm{D^s g}{L^{p_2}(w_2)}. 
\end{equation}
Moreover, if $w_1=w_2$ then different pairs of $p_1, p_2$ can be used on the right-hand side of \eqref{KP:CM:Lebesgue}.
\end{theorem} 

Corollary (\ref{coro:KP:CM:Hardy}) and theorem \label{BrNa2017_THM} overlap in the following ways. 

\begin{itemize}
\item The estimate (\ref{KP:CM:Hardy}) allows for $0<p,p_1,p_2 < \infty$, $w_1,w_2 \in A_\infty$, and the $H^P(w)$ on the left-hand side if $s>\tau_p(w)$. However (\ref{KP:CM:Lebesgue}) requires $1<p_1,p_2\le \infty,$ $w_1\in A_{p_1},$ and $w_2\in A_{p_2}$ but allows for the Lebesgue norm on left-hand side when $s>\tau_p$. So (\ref{KP:CM:Hardy}) is less restrictive than (\ref{KP:CM:Lebesgue}) in terms of the indices $p$, $p_1$, and $p_2$ and the classes that the weights $w_1$ and $w_2$ belong to. However because $\tau_p \leq \tau_p(w)$ (\ref{KP:CM:Hardy}) is more restrictive in terms of the range of the regularity $s$ than (\ref{KP:CM:Lebesgue}).
\item If $s>\tau_{p}(w),$   $1/2<p<\infty,$ $1<p_1,p_2<\infty$ such that $\hcline,$  $w_1\in A_{p_1}$ and $w_2\in A_{p_2}$ then \eqref{KP:CM:Hardy} implies \eqref{KP:CM:Lebesgue}. However if $\tau_p < \tau_p(w)$ then \eqref{KP:CM:Hardy} does not imply \eqref{KP:CM:Lebesgue} for $\tau_p < s < \tau_p(w)$. Here we give examples of weights $w_1$ and $w_2$ such that the corresponding weight $w$ satisfies $\tau_p < \tau_p(w)$.  Let $1<p_1\leq p_2 <\infty$ and $w_1(x) = w_2(x) = w(x) = |x|^a$ with $n(r-1)<a<n(p_1-1)$ for some $1<r<p_1$. Then $w(x) \in A_{p_1}$, $A_{p_1} \subset A_{p_2}$, and $w \notin A_r$. This implies that $1<\tau_w$ which implies that $\tau_p < \tau_p(w)$. 
\item  The estimate (\ref{KP:CM:Lebesgue}) implies (\ref{Kp:CM:Hardy2}) for $1<p<\infty$, $w\in A_p$, and $s>\tau_p$ and gives the endpoint estimate 
\[ \norm{D^s(T_\sigma(f,g))}{L^p(w)} \lesssim \norm{D^s f}{L^{\infty}(w)} \norm{g}{L^{p}(w)} +  \norm{f}{L^{p}(w)}\norm{D^s g}{L^\infty}. \]
However (\ref{Kp:CM:Hardy2}) allows $0<p<\infty$ and $w\in A_\infty$ if $s>\tau_p (w)$.
\end{itemize}

\subsection{Connection to Kato-Ponce inequalities}
By setting $\sigma \equiv 1$ we obtain the following Kato-Ponce inequality as a corollary to Theorem (\ref{thm:CM:TL:B}). 

\begin{corollary}\label{coro:KP:TL:B}  Consider  $0 < p, p_1, p_2  \le \infty$  such that $\hcline$ and  $0 < q \leq \infty;$ let  $w_1,w_2\in A_\infty$ and set $w=w_1^{{p}/{p_1}} w_2^{{p}/{p_2}}.$ 
If $0 < p ,p_1,p_2< \infty$ and  $s > \tau_{p,q}(w),$ it holds that
\begin{equation}\label{KP:TL}
\norm{fg}{\tlw{p}{s}{q}{w}} \lesssim \norm{f}{\tlw{p_1}{s}{q}{w_1}} \norm{g}{H^{p_2}(w_2)} +  \norm{f}{H^{p_1}(w_1)}   \norm{g}{\tlw{p_2}{s}{q}{w_2}} \quad \forall f, g \in \swz.
\end{equation}
If $0 < p, p_1,p_2 \le \infty$ and $s > \tau_p(w)$, it holds that
\begin{equation}\label{KP:B}
\norm{fg}{\besw{p}{s}{q}{w}} \lesssim \norm{f}{\besw{p_1}{s}{q}{w_1}} \norm{g}{H^{p_2}(w_2)} +  \norm{f}{H^{p_1}(w_1)}   \norm{g}{\besw{p_2}{s}{q}{w_2}} \quad \forall f, g \in \swz,
\end{equation}
where the Hardy spaces $H^{p_1}(w_1)$ and $H^{p_2}(w_2)$ must be replaced by $L^\infty$ if $p_1=\infty$ or $p_2=\infty,$ respectively.

If $w_1=w_2$ then different pairs of $p_1, p_2$ can be used on the right-hand sides of \eqref{KP:TL} and \eqref{KP:B}; moreover, if $w\in A_\infty,$ then 
\begin{equation}\label{KP:TL2}
\norm{fg}{\tlw{p}{s}{q}{w}} \lesssim \norm{f}{\tlw{p}{s}{q}{w}} \norm{g}{L^\infty} +  \norm{f}{L^\infty}   \norm{g}{\tlw{p}{s}{q}{w}} \quad \forall f, g \in \swz,
\end{equation}
where $0<p<\infty,$ $0<q\le \infty$ and $s>\tau_{p,q}(w).$
\end{corollary}

In particular of we set $q=2$ and use the connection between weighted Hardy spaces and weighted Triebel-Lizorkin spaces we obtain the following corollary. 

\begin{corollary}\label{coro:KP:Hardy} 
Consider  $0 < p, p_1, p_2  < \infty$  such that $\hcline;$ let  $w_1,w_2\in A_\infty$ and set $w=w_1^{{p}/{p_1}} w_2^{{p}/{p_2}}.$ 
If  $s > \tau_{p}(w),$ it holds that
\begin{equation}\label{KP:Hardy}
\norm{D^s(fg)}{H^p(w)} \lesssim \norm{D^s f}{H^{p_1}(w_1)} \norm{g}{H^{p_2}(w_2)} +  \norm{f}{H^{p_1}(w_1)}   \norm{D^s g}{H^{p_2}(w_2)} \quad \forall f, g \in \swz.
\end{equation}
If $w_1=w_2$ then different pairs of $p_1, p_2$ can be used on the right-hand side of \eqref{KP:Hardy}; moreover, if $w\in A_\infty,$ then 
\begin{equation*}
\norm{D^s(fg)}{H^p(w)} \lesssim \norm{D^s f}{H^{p}(w)} \norm{g}{L^\infty} +  \norm{f}{L^\infty}   \norm{D^s g}{H^{p}(w)} \quad \forall f, g \in \swz,
\end{equation*}
where $0<p<\infty$ and $s>\tau_{p}(w).$
\end{corollary}

We note that (\ref{KP:Hardy}) extends and improves the inequality \eqref{KPH:Lp}. The inequality \eqref{KP:Hardy} extends the range of $p$, $p_1$, and $p_2$ by allowing $0<p,p_1,p_2<\infty$ while \eqref{KPH:Lp} requires that $1<p_1,p_2<\infty$. Additionally (\ref{KP:Hardy}) allows for the $H^p$ norm on the left-hand side which is generally larger than $||\cdot||_{L^p}$. 
 
 
 \subsection{Proof of Theorem \ref{thm:CM:TL:B}}
 
 \begin{proof}[Proof of Theorem~\ref{thm:CM:TL:B}] Consider $\Phi,$ $\Psi,$ $T_\sigma^1,$ $T_\sigma^2,$ $\{\C_j(a,b)\}_{j\in\ent,a,b\in\ent^n}$ as in Section~\ref{sec:decomp}. Let $m,$ $\sigma,$ $p,$ $p_1,$ $p_2,$ $q,$ $s,$ $w_1,$ $w_2$ and $w$ be as in the hypotheses.  
For ease of notation, $p_1$ and $p_2$ will be assumed to be finite; the same proof applies for \eqref{KP:CM:B} if that is not the case, and for \eqref{KP:CM:TL2}.


We next prove \eqref{KP:CM:TL} and \eqref{KP:CM:B}. 
 By symmetry,  it is enough to work with $T_\sigma^1$ and prove that 
 \begin{align*}
 \norm{T^{1}_\sigma(f,g)}{\dot{F}^s_{p,q}(w)} \lesssim  \norm{f}{\dot{F}^{s+m}_{p_1, q}(w_1)} \norm{g}{H^{p_2}(w_2)}\quad  \text{ and }\quad 
 \norm{T^1_\sigma(f,g)}{\dot{B}^s_{p,q}(w)} \lesssim  \norm{f}{\dot{B}^{s+m}_{p_1, q}(w_1)} \norm{g}{H^{p_2}(w_2)}.
\end{align*}
  Moreover, since $\norm{\sum f_j}{\tlw{p}{s}{q}{w}}^{\min(p,q,1)}\lesssim \sum\norm{f_j}{\tlw{p}{s}{q}{w}}^{\min(p,q,1)}$  and similarly for $\besw{p}{s}{q}{w}$, it suffices to prove that, given $\varepsilon>0$ there exist $0<r_1,r_2\le 1$  such that for all $g\in \sw$ and  $f\in \swz$ ($f\in \sw\cap \tlw{p}{s}{q}{w}$ or  $f\in \sw\cap \besw{p}{s}{q}{w}$ if $m\ge 0$),  it holds that
\begin{align}
 \norm{T^{a,b}(f,g)}{\dot{F}^s_{p,q}(w)} \lesssim (1+\abs{a})^{\varepsilon+\frac{n}{r_1}}  (1+\abs{b})^{\varepsilon+\frac{n}{r_2}} \norm{f}{\dot{F}^{s+m}_{p_1, q}(w_1)} \norm{g}{H^{p_2}(w_2)}\label{eq:estbbTL},\\
 \norm{T^{a,b}(f,g)}{\dot{B}^s_{p,q}(w)} \lesssim (1+\abs{a})^{\varepsilon+\frac{n}{r_1}}  (1+\abs{b})^{\varepsilon+\frac{n}{r_2}} \norm{f}{\dot{B}^{s+m}_{p_1, q}(w_1)} \norm{g}{H^{p_2}(w_2)}\label{eq:estbbB},
\end{align}
where
\[
T^{a,b}(f,g):=\sum_{j\in\ent} \C_j(a,b) \,(\Do{ \tau_a \Psi }{j} f)\, (\So{ \tau_b \Phi }{j} g )
\]
and the implicit constants are independent of $a$ and $b.$  We will assume $q$ finite; obvious changes apply if that is not the case.

In view of the supports of $\Psi$ and $\Phi$ we have that 
\begin{equation*}
\supp (\mathcal{F}[\C_j(a,b) \,(\Do{ \tau_a \Psi }{j} f ) \, ( \So{ \tau_b \Phi }{j} g )])  \subset \{\xi \in \rn: |\xi| \lesssim 2^j \} \quad \forall j \in \ent,\,a,b\in \ent^n.
\end{equation*}

For \eqref{eq:estbbTL},
Theorem \ref{thm:Nikolskij:weighted}\eqref{item:thh:Nikolskij:TL}, the bound \eqref{eq:cjbound} for $\C_j(a,b)$, and H\"older's inequality  imply
\begin{align*}
\norm{T^{a,b}(f,g)}{\dot{F}^s_{p,q}(w)} & \lesssim \norm{\{2^{sj} \C_j(a,b) \,(\Do{ \tau_a \Psi }{j} f ) \, (\So{ \tau_b \Phi }{j} g)\}_{j\in\ent} }{L^p(w)(\ell^q)}\\
& \lesssim \norm{\left(\sum\limits_{j \in \ent}  2^{(s +m) q j}  |(\Do{ \tau_a \Psi }{j} f )(x) \, (\So{ \tau_b \Phi }{j} g)|^q   \right)^\frac{1}{q}}{L^p(w)}\\
& \le\norm{\sup\limits_{j \in \ent} |(\So{ \tau_b \Phi }{j} g)| \left(\sum\limits_{j \in \ent}  2^{(s +m) q j}  |(\Do{ \tau_a \Psi }{j} f )|^q   \right)^\frac{1}{q}}{L^p(w)}\\
& \le \norm{\left(\sum_{j \in \ent}  2^{(s +m) q j}  |\Do{ \tau_a \Psi }{j} f |^q   \right)^\frac{1}{q}}{L^{p_1}(w_1)} \norm{\sup\limits_{j \in \ent} |\So{ \tau_b \Phi }{j} g|}{L^{p_2}(w_2)}.
\end{align*}
Consider $\varphi,\psi\in\sw$ as in Section~\ref{sec:spaces} such that   $\widehat{\varphi}\equiv 1$ on $\supp(\widehat{\Phi})$ and  $\widehat{\psi}\equiv 1$ on $\supp(\widehat{\Psi}).$  Let   $0<r_1<\min(1, p_1/\tau_{w_1},q)$; by Lemma~\ref{lem:pointineq} and the weighted Fefferman-Stein inequality  we have that  
\begin{align*}
\norm{\left(\sum_{j \in \ent}  2^{(s +m) q j}  |(\Do{ \tau_a \Psi }{j} f )|^q   \right)^\frac{1}{q}}{L^{p_1}(w_1)}&\lesssim (1+\abs{a})^{\varepsilon+\frac{n}{r_1}}
\norm{\left(\sum_{j \in \ent}  2^{(s +m) q j}  |\M_{r_1}(\Do{\psi }{j} f) |^q   \right)^\frac{1}{q}}{L^{p_1}(w_1)}\\
&\lesssim (1+\abs{a})^{\varepsilon+\frac{n}{r_1}} \norm{\left(\sum_{j \in \ent}  2^{(s +m) q j}  |\Do{\psi }{j} f|^q   \right)^\frac{1}{q}}{L^{p_1}(w_1)}\\
&\sim (1+\abs{a})^{\varepsilon+\frac{n}{r_1}}  \norm{f}{\dot{F}^{s+m}_{p,q}(w_1)},
\end{align*}
where the implicit constants are independent of $a$ and $f.$ Next, let  $0<r_2<\min(1,p_2/\tau_{w_2})$; by Lemma~\ref{lem:pointineq} and the boundedness properties of the Hardy-Littlewood maximal operator on weighted Lebesgue space  we have that  
\begin{align*}
\norm{\sup_{j\in\ent}|\So{\tau_b\Phi}{j}g|}{L^{p_2}(w_2)}&\lesssim (1+\abs{b})^{\varepsilon+\frac{n}{r_2}} \norm{\M_{r_2}(\sup_{j\in\ent}|\So{\varphi}{j}g|)}{L^{p_2}(w_2)}\\
&\lesssim (1+\abs{b})^{\varepsilon+\frac{n}{r_2}} \norm{\sup_{j\in\ent}|\So{\varphi}{j}g|}{L^{p_2}(w_2)}\\
&\sim (1+\abs{b})^{\varepsilon+\frac{n}{r_2}} \norm{g}{H^{p_2}(w_2)},
\end{align*}
where the implicit constants are independent of $b$ and $g.$ Putting all together we obtain \eqref{eq:estbbTL}.


For \eqref{eq:estbbB},  Theorem \ref{thm:Nikolskij:weighted}\eqref{item:thh:Nikolskij:B}, the bound \eqref{eq:cjbound} for $\C_j(a,b)$ and H\"older's inequality  give
\begin{align*}
\norm{T^{a,b}(f,g)}{\dot{B}^s_{p,q}(w)} & \lesssim \norm{\{2^{sj} \C_j(a,b) \,(\Do{ \tau_a \Psi }{j} f ) \, (\So{ \tau_b \Phi }{j} g)\}_{j\in\ent} }{\ell^q(L^p(w))}\\
& \lesssim \left(\sum\limits_{j \in \ent}  2^{(s +m) q j}  \norm{(\Do{ \tau_a \Psi }{j} f ) \, (\So{ \tau_b \Phi }{j} g)}{L^p(w)}^q   \right)^\frac{1}{q}  \\
&\le  \left(\sum\limits_{j \in \ent}  2^{(s +m) q j}  \norm{(\Do{ \tau_a \Psi }{j} f ) }{L^{p_1}(w_1)}^q   \right)^\frac{1}{q}  \norm{\sup_{k\in \ent}|\So{ \tau_b \Phi }{k} g|}{L^{p_2}(w_2)}\\
& \lesssim  (1+\abs{a})^{\varepsilon+\frac{n}{r_1}}  (1+\abs{b})^{\varepsilon+\frac{n}{r_2}}  \norm{f}{\dot{B}^{s+m}_{p_1, q}(w_1)} \norm{g}{H^{p_2}(w_2)},
\end{align*}
where in the last inequality we have used Lemma~\ref{lem:pointineq} and the boundedness properties of $\M$ with  $0<r_j<\min(1,p_j/\tau_{w_j})$ for $j=1,2$ .


It is clear from the proof above that if $w_1=w_2,$ then  different pairs of $p_1, p_2$ related to $p$ through the H\"older condition can be used on the right-hand sides of \eqref{KP:CM:TL} and \eqref{KP:CM:B}; in such case $w=w_1=w_2.$  
\end{proof}

 \subsection{Inhomogeneous Leibniz-type rules}
 
 \section{Leibniz rules in other functions spaces}
 \subsection{Lorrentz spaces}
 \subsection{Morrey spaces}
 \subsection{Variable Lebesgue spaces}
 
 \section{Applications to scattering properties of PDEs}