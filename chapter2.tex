\cleardoublepage

\chapter{Weighted Leibniz-type rules with applications to scattering properties of PDEs}
%\label{makereference2}

In this chapter we obtain Leibniz-type rules for bilinear multiplier operators associated to Coifman-Meyer multipliers in the settings of Triebel-Lizorkin and Besov spaces based in weighted quasi-Banach spaces. These results extend and improve the Leibniz-type rules (\ref{def:leibniz}) and (\ref{def:ileibniz}).  Additionally, we apply these results to obtain scattering properties of systems of partial differential equations involving fractional powers of the Laplacian. In the following sections Triebel-Lizorkin and Besov spaces based in weighted Lebesgue spaces, Muckenhoupt weights, and Coifman-Meyer multipliers defined and discussed; then we will prove Theorem \ref{thm:CM:TL:B} and the corresponding result for inhomogeneous Triebel-Lizorkin and Besov spaces based in weighted Lebesgue spaces. The method used to prove the estimates in Theorem \ref{thm:CM:TL:B} can be adapted to obtain related results for Triebel-Lizorkin and Besov spaces based in other quasi-Banach spaces such as Morrey, Lorentz, and variable-Lebesgue spaces. Last applications to systems of partial differential equations with fractional powers of the Laplacian are obtained as the solutions of these systems can be viewed as bilinear multiplier operators associated to multipliers which satisfy the hypotheses of Theorem \ref{thm:CM:TL:B}.

\section{Preliminaries}
In this section we discuss the operators, function spaces, and their properties that are used in Theorem \ref{thm:ICM:TL:B} and its proof. We first define Coifman-Meyer multipliers and see boundedness properties of operators associated to these multipliers in Lebesgue spaces. Next the weighted Triebel-Lizorkin and Besov spaces used in Theorem \ref{thm:ICM:TL:B} are defined and their connection other well known function spaces are discussed. In particular the Nikol'skij representation of these function spaces, which is an important tool for the proof of Theorem \ref{thm:ICM:TL:B}, will be stated with a detailed proof.
\subsection{Coifman-Meyer Multipliers}\label{Coifman-Meyer Multipliers}

The symbols used in Theorem \ref{thm:CM:TL:B} and results later in this chapter are Coifman-Meyer multipliers. Such multipliers are defined as follows.

\begin{dfn}\label{CM_def}
Given $m\in\mathbb{R}$, a smooth function $\sigma = \sigma(\xi,\eta)$, $\xi,\eta\in\rn$, is a \textit{Coifman-Meyer multiplier} of order $m$ if for all multi-indices $\alpha,\beta\in\mathbb{N}^n_0$ there exists a positive constant $C_{\alpha,\beta}$ such that 
\begin{equation}\label{eq:CMm}
|\partial_\xi^\alpha \partial_\eta^\beta \sigma(\xi, \eta)| \leq C_{\alpha, \beta} (|\xi|+|\eta|)^{m -(\abs{\alpha}+ \abs{\beta})} \quad \forall (\xi, \eta) \in \re^{2n} \setminus \{(0,0)\}.
\end{equation}
We say $\sigma = \sigma(\xi,\eta)$, $\xi,\eta\in\rn$, is an \textit{inhomogeneous Coifman-Meyer multiplier} of order $m$ if for all multi-indices $\alpha,\beta\in\mathbb{N}^n_0$ there exists a positive constant $C_{\alpha,\beta}$ such that 
\begin{equation}\label{eq:CMm}
|\partial_\xi^\alpha \partial_\eta^\beta \sigma(\xi, \eta)| \leq C_{\alpha, \beta} (1+|\xi|+|\eta|)^{m -(\abs{\alpha}+ \abs{\beta})} \quad \forall (\xi, \eta) \in \re^{2n}.
\end{equation}
\end{dfn}
Bilinear multiplier operators associated to Coifman-Meyer multipliers have been well studied. In particular such operators are examples of Calder\'on-Zygmund operators. As a consequence the following estimate holds 
\[ \norm{T_\sigma(f,g)}{L^p}\lesssim \norm{f}{L^{p_1}}\norm{g}{L^{p_2}} \]
holds where $\sigma$ is a Coifman-Meyer multiplier, $\frac{1}{p} = \frac{1}{p_1} + \frac{1}{p_2}$, and $1<p_1,p_2<\infty$ was established in \citep{MR1880324}. For more on these multipliers see Coifman-Meyer \citep{MR518170} for $L^2(\rn)$ estimates and background information. Additionally in David-Journ\'{e} \cite{MR763911} and Grafakos-Torres \citep{MR1880324} established estimates for Coifman-Meyer multipliers in their work on Calder\'on-Zygmund operators. Additionally estimates for operators associated to Coifman-Meyer multipliers in weighted-Lebesgue spaces and were proven by Grafakos-Torres \citep{MR1947875} and Lerner et al. \citep{MR2483720} in connection to a Calder\'{o}n-Zygmund operators. 	

In the proof of Theorem \ref{thm:CM:TL:B} we will use a decomposition of Coifman-Meyer operators. Before describing this decomposition we first define two linear multiplier operators and the translation operator.
For $j\in\mathbb{Z}$ and $\varphi \in \sw$ such that $\text{supp}(\varphi) \subset \{\xi : a < |\xi| < b \}$ the operator $\Delta^\varphi_j$ is defined so that $\widehat{\Delta^\varphi_j f}(\xi) = \varphi(2^{-j})\widehat{f}(\xi)$. 
If $\varphi \in \sw$ and $\text{\supp}(\varphi) \subset \{ \xi : |\xi| < b \}$ then the operator $S^\varphi_j$ is defined so that $\widehat{S^\varphi_j f}(\xi) = \varphi(2^{-j})\widehat{f}(\xi)$.
We then have that $\widehat{\Delta^\varphi_j f}(\xi)$ isolates $\widehat{f}$ to frequencies of order $2^j$ and $\widehat{S^\varphi_j f}(\xi)$ isolates $\widehat{f}$ to frequencies of order \textit{up to} $2^j$. 
We define the translation operator $\tau_a$ as $\tau_a f(x) := f(x+a)$ so by properties of the Fourier transform $\widehat{\tau_a f}(\xi) = e^{2\pi i \xi \cdot a} \widehat{f}(\xi)$.

Let $\sigma$ be a Coifman--Meyer multiplier of order $m.$
Fix $\Psi \in \sw$ such that 
$$
\supp(\widehat{\Psi}) \subseteq \{ \xi \in \rn : \fr{1}{2} < |\xi| < 2 \} \quad\text{ and }\quad
\sum_{j\in\ent} \widehat{\Psi}(2^{-j}\xi) = 1 \,\,\forall \xi \in \rn \setminus \{ 0 \};
$$
define $\Phi \in \sw$ so that
$$
\widehat{\Phi}(0) := 1,\quad \widehat{\Phi}(\xi) := \sum_{j \le 0} \widehat{\Psi}(2^{-j}\xi)\quad \forall \xi \in \rn \setminus \{0\}.$$
For $a,b\in\rn,$ $\Do{\tau_a \Psi }{j} f$ and  $\So{ \tau_a \Phi }{j} f$ satisfy 
$$\widehat{\Do{\tau_a \Psi }{j} f}(\xi)=\widehat{\tau_a\Psi}(2^{-j}\xi)\widehat{f}(\xi)=e^{2\pi i 2^{-j}\xi\cdot a} \widehat{\Psi}(2^{-j}\xi)\widehat{f}(\xi)$$
 and   
 $$\widehat{\So{\tau_b \Phi }{j} f}(\xi)=\widehat{\tau_b\Phi}(2^{-j}\xi)\widehat{f}(\xi)=e^{2\pi i 2^{-j}\xi\cdot b} \widehat{\Phi}(2^{-j}\xi)\widehat{f}(\xi).$$
By the work of Coifman and Meyer in \cite{MR518170},  given $N\in \na$ such that $N>n,$ it follows that $T_\sigma= T_\sigma^1 + T_\sigma^2$, where, for  $f\in \swz$ ($f\in \sw$ if $m\ge 0$) and $g\in\sw,$
\begin{align}\label{eq:decompT1}
T_\sigma^1(f,g)(x) &= \sum_{a,b\in \ent^n} \frac{1}{(1+|a|^2+|b|^2)^N} \sum_{j\in\ent} \C_j(a,b) \,(\Do{ \tau_a \Psi }{j} f)(x)\, (\So{ \tau_b \Phi }{j} g )(x)\quad \forall x\in \rn,
\end{align}
 the coefficients $\C_j(a,b)$   satisfy
\begin{equation}\label{eq:cjbound}
\abs{\C_j(a,b)}\lesssim 2^{jm}\quad \forall a,b\in\ent, j\in \ent,
\end{equation}
with the implicit constant depending on $\sigma,$ and an analogous expression holds for $T_\sigma^2$ with the roles of $f$ and $g$ interchanged. 

We note that for the formula \eqref{eq:decompT1} and its corresponding counterpart for $T^2_\sigma$ to hold, the condition \eqref{eq:CMm} on the derivatives of $\sigma$ is only needed for multi-indices $\alpha$ and $\beta$ such that  $\abs{\alpha+\beta}\le 2N$

If $\sigma$ is an inhomogeneous Coifman--Meyer multiplier of order $m,$ a similar decomposition to \eqref{eq:decompT1} follows  with the summation in $j\in\naz$ rather than $j\in\ent,$  with  $\Delta_0^{\tau_a \Psi}$ replaced by $S_0^{\tau_a\Phi}$ and for $f,g\in \sw.$




\subsection{Function spaces}

\subsubsection{Weighted spaces}

\begin{dfn}
A \textit{weight} $w$ defined on $\rn$ is a locally integrable function such that $0<w(x)<\infty$ for almost every $x\in\rn$.
\end{dfn}

Given a weight $w$ and $0<p<\infty$ we define the weighted Lebesgue space $L^p(w)$ as the space of all measurable functions satisfying 
\[ \norm{f}{L^p(w)} = \left(\int_\rn |f(x)|^p w(x)dx\right)^{\frac{1}{p}} < \infty. \]
In the case that $p=\infty$ we define $L^\infty (w) = L^\infty$. 

The specific classes of weights in the hypotheses of the results of this chapter are Muckenhoupt weights, which we next define. 
\begin{dfn}
For $1<p<\infty$ the \textit{Muckenhoupt class} $A_p$ consists of all weights $w$ on $\rn$ satisfying 
 \begin{equation}\label{weight_condition}
 \sup_B\left(\frac{1}{\abs{B}}\int_Bw(x)\dx\right)\left(\frac{1}{\abs{B}}\int_Bw(x)^{-\frac{1}{p-1}}\dx\right)^{p-1}<\infty,
 \end{equation}
where the supremum is taken over all Euclidean balls $B\subset\rn$ and $|B|$ is the Lebesgue measure of $B$. For $p=\infty$ we define $A_\infty := \cup_{1<p} A_p$. 
\end{dfn}

From this definition it follows that $A_p \subset A_q$ when $p\leq q$. 
For $w\in A_\infty$ we set 
$\tau_w = \inf\{\tau \in (1,\infty]: w\in A_\tau \}$. 
The condition (\ref{weight_condition}) is motivated by a connection to the Hardy-Littlewood maximal operator, which we now define.
\begin{dfn}
For a locally integrable function $f$ the \textit{Hardy-Littlewood maximal operator} $\mathcal{M}$ is defined as 
\[ \mathcal{M}(f)(x)  = \sup_{B \ni x} \frac{1}{|B|}\int_B |f(y)|dy \quad \forall x\in\mathbb{R}^n. \]
\end{dfn}
It turns out that the Hardy-Littlewood maximal operator is bounded on $L^p(w)$ if and only if $w\in A_p$. That is, for $1<p<\infty$, $w\in A_p$ if and only if 
\[\norm{\mathcal{M}(f)}{L^p (w)} \lesssim \norm{f}{L^p (w)}, \quad \forall f\in L^p(\rn). \]
Later in this chapter we will use the maximal function $\mathcal{M}_r (f) = \left(\mathcal{M}(|f|^r)\right)^{\frac{1}{r}}$. Since $0<r<p/\tau_w$ if and only if $0<r<p$ and $w\in A_{p/r}$ by the properties for the Hardy-Littlewood maximal operator stated above it holds that $\mathcal{M}_r$ is bounded on $L^p (w)$ when $w\in A_{p/r}$. This fact is a particular case of the following theorem which we refer to as the Fefferman-Stein inequality.

\begin{theorem}
If $0<p<\infty,$ $0<q\le \infty,$  $0<r <\min(p,q)$ and $w \in A_{p/r}$ (i.e. $0<r<\min(p/\tau_w,q)$), then for all sequences $\{f_{j}\}_{j\in\ent}$ of locally integrable functions defined on $\rn,$ we have
 \begin{equation*}
 \norm{\left(\sum_{j\in\ent}\abs{\M_r (f_j)}^q\right)^{\frac{1}{q}}}{\lebw{p}{w}}\lesssim
 \norm{\left(\sum_{j\in\ent}\abs{f_j}^q\right)^{\frac{1}{q}}}{\lebw{p}{w}},\label{eq:wFS}
 \end{equation*}
where the implicit constant depends on $r,$ $p,$ $q,$ and $w$ and the summation in $j$ should be replaced by the supremum in $j$ if $q=\infty.$
\end{theorem}
For more details on the Muckenhoupt classes see Muckenhoupt \cite{MR0293384} Grafakos \cite{MR3243741}.

\subsubsection{Triebel-Lizorkin and Besov spaces}\label{TL_B_section}
Here we describe the function spaces in which (\ref{thm:CM:TL:B}) is based and some properties of such spaces. 

Let $\psi$ and $\varphi$ be functions in $\sw$ satisfying the following conditions:
\begin{itemize}
\item supp$(\widehat{\psi})\subset\{\xi\in\rn : \frac{1}{2} < |\xi| <2\}$,
\item $|\widehat{\psi}(\xi)|>c$ for all $\xi$ such that $\frac{3}{5} < |\xi| < \frac{5}{3}$ for some $c>0$,
\item supp$(\widehat{\varphi}) \subset \{\xi\in\rn : |\xi| < 2\}$,
\item $|\widehat{\varphi}(\xi)|>c$ for $|\xi| < \frac{5}{3}$ and some $c>0$. 
\end{itemize}
For $\psi$ supported in an annulus and $j\in\ent$ we define the operator $\Delta^\psi_j(f)$ through its Fourier transform as \[\widehat{\Delta^\psi_j (f)}(\xi) = \psi(2^{-j}\xi)\widehat{f}(\xi).\]
Similarly we define the operator $S^\varphi_0$ through the Fourier transform as
\[ \widehat{S^\varphi_0 (f)}(\xi) = \widehat{\varphi}(\xi)\widehat{f}(\xi).\]
For such $\psi$ the homogeneous Triebel-Lizorkin and Besov spaces are sets of tempered distributions modulo polynomials. We denote the set of polynomials over $\rn$ by $\mathcal{P}(\rn)$. The space of \textit{tempered distributions} $\mathcal{S}'$ is the space of all continuous linear functionals acting on the Schwartz class.

\begin{dfn}\label{TL_B_def}
Let $s\in\mathbb{R}$, $0<p<\infty$, and $0<q\leq\infty$.
\begin{itemize}
\item The weighted \textit{homogeneous Triebel-Lizorkin space} $\tlw{p}{s}{q}{w}$ consists of all $f\in \swp/\mathcal{P}(\rn)$ such that 
\begin{equation*}
\norm{f}{\tlw{p}{s}{q}{w}}=\norm{\left(\sum_{j\in\ent}(2^{sj}|\Delta^\psi_jf|)^q\right)^{\frac{1}{q}}}{\lebw{p}{w}}<\infty.
\end{equation*}
\item The weighted \textit{homogeneous Besov space} $\besw{p}{s}{q}{w}$ consists of all $f\in \swp/\mathcal{P}(\rn)$ such that 
\begin{equation}
\norm{f}{\besw{p}{s}{q}{w}} = \left(\sum_{j\in\ent} (2^{js}\norm{\Delta^\psi_jf}{L^p(w)})^q\right)^{\frac{1}{q}} < \infty.
\end{equation}
\end{itemize}
\end{dfn}

Given $\varphi$, $\psi$, $S^\varphi_0$, and $\Delta^\psi_j$ as above the weighted inhomogeneous Triebel-Lizorkin and Besov spaces are defined as follows. 

\begin{dfn}\label{ITL_B_def}
Let $s\in\mathbb{R}$ and $0<q\leq\infty$.
\begin{itemize}
\item For $0<p<\infty$ the weighted \textit{inhomogeneous Triebel-Lizorkin space} $\itlw{p}{s}{q}{w}$ is the class of all $f\in\mathcal{S}'(\rn)$ such that
\begin{equation*}
\norm{f}{\itlw{p}{s}{q}{w}}= \norm{S^\varphi_0 f}{L^p(w)} + \norm{\left(\sum_{j\in\naz}(2^{sj}|\Delta^\psi_jf|)^q\right)^{\frac{1}{q}}}{\lebw{p}{w}}<\infty.
\end{equation*}
\item For $0<p\leq\infty$ weighted \textit{inhomogeneous Besov space} $\besw{p}{s}{q}{w}$ is the class of all $f\in\mathcal{S}'(\rn)$ such that 
\begin{equation*}
\norm{f}{\ibesw{p}{s}{q}{w}} = \norm{S^\varphi_0 f}{L^p(w)} + \left(\sum_{j\in\naz} (2^{js}\norm{\Delta^\psi_jf}{L^p(w)})^q\right)^{\frac{1}{q}} < \infty.
\end{equation*}
\end{itemize}
\end{dfn}

The definitions above are independent of the choice of $\varphi$ and $\psi$. The Triebel-Lizorkin and Besov defined above are generally quasi-Banach spaces and if $1\leq p,q <\infty$ they are Banach spaces. These spaces provide a framework to study a variety of other spaces such as Lebesgue, Hardy, and Sobolev spaces with a unified approach.  For instance the following equivalences hold where the corresponding function spaces have equivalent norms 
\begin{equation}\label{TL&Hardy}
 \itlw{p}{0}{2}{w} \simeq H^p (w) \text{ for } 0<p<\infty, \quad w\in A_\infty, 
\end{equation}
\begin{equation}
 \itlw{p}{0}{2}{w} \simeq L^p(w) \simeq H^p(w) \text{ for } 1<p<\infty, \quad w\in A_p, 
\end{equation}
\begin{equation}
 \itlw{p}{s}{2}{w} \simeq \dot{W}^{s,p}(w) \text{ for } 1<p<\infty, \quad w\in A_p.
\end{equation}

For a detailed overview of the development of Besov and Triebel-Lizorkin spaces see Triebel \cite{MR3024598} and Qui \cite{MR676560} for the unweighted and weighted settings respectively.
%Additionally by the lifting property of Triebel-Lizorkin and Besov spaces Theorem (\ref{thm:CM:TL:B}) and other results in this manuscript can be seen as Leibniz-type rules as in Chapter 1. 
For weighted Triebel-Lizorkin spaces the so called lifting property is as follows: for $s$, $p$, and $q$ as in (\ref{ITL_B_def}) and (\ref{TL_B_def}) and $w\in A_\infty$ we have that 

 \begin{align*}
 & \norm{f}{\tlw{p}{s}{q}{w}}\simeq\norm{D^s f}{\tlw{p}{0}{q}{w}} \quad \text{ and } \quad \norm{f}{\itlw{p}{s}{q}{w}}\simeq\norm{J^s f}{\itlw{p}{0}{q}{w}}.
 \end{align*}
The corresponding statement for Besov spaces is: for $s$, $p$, and $q$ as in (\ref{ITL_B_def}) and (\ref{TL_B_def}) and $w\in A_\infty$ we have that 
 \begin{align*}
 & \norm{f}{\besw{p}{s}{q}{w}}\simeq\norm{D^s f}{\besw{p}{0}{q}{w}} \quad \text{ and } \quad \norm{f}{\ibesw{p}{s}{q}{w}}\simeq\norm{J^s f}{\ibesw{p}{0}{q}{w}}.
 \end{align*}
 

  

\subsubsection{Nikol'skij representations for weighted homogeneous and inhomogeneous Triebel-Lizorkin and Besov spaces}

An important tool for the proof of Theorem \ref{thm:CM:TL:B} is the Nikol'skij representation for weighted Triebel-Lizorkin and Besov spaces. Here we state a weighted version of \cite[Theorem 3.7]{MR837335} (see also \cite[Section 2.5.2]{MR3024598}). 
%For  completeness, a sketch of its proof is outlined in Appendix~\ref{sec:appendix}.

\begin{theorem}\label{thm:Nikolskij:weighted} For $\A> 0,$ let $\{u_j\}_{j \in \ent} \subset \mathcal{S}'(\rn)$ be a sequence of tempered distributions such that
\begin{equation*}
\supp(\widehat{u_j}) \subset B(0, \A\, 2^j ) \quad \forall j \in \ent.
\end{equation*}
If $w\in A_\infty,$ then the following holds:  
\begin{enumerate}
\item[(i)]\label{item:thh:Nikolskij:TL} Let $0 < p < \infty$, $0 < q \leq \infty$ and $s > \tau_{p,q}(w)$. If $\norm{\{2^{js} u_j\}_{j\in\ent}}{L^p(w)(\ell^{q})} < \infty$, then the series $\sum_{j \in \ent} u_j$ converges in $\tlw{p}{s}{q}{w}$ (in $\mathcal{S}'_0(\rn)$ if $q=\infty$) and 
\begin{equation*}
\norm{\sum_{j \in \ent} u_j}{\tlw{p}{s}{q}{w}} \lesssim  \norm{\{2^{js} u_j\}_{j\in\ent}}{L^p(w)(\ell^{q})},
\end{equation*}
where the implicit constant depends only on $n,$ $\A,$ $s,$ $p$ and  $q.$  An analogous statement, with $j\in\naz,$ holds true for $\itlw{p}{s}{q}{w}$ (when $q=\infty,$  the convergence is in $\swp$).
\item[(ii)]\label{item:thh:Nikolskij:B} Let $0 < p, q \leq \infty$ and $s > \tau_p(w)$. If $\norm{\{2^{js} u_j\}_{j\in\ent}}{\ell^{q}(L^p(w))} < \infty$, then the series $\sum_{j \in \ent} u_j$ converges in  $\besw{p}{s}{q}{w}$ (in $\mathcal{S}'_0(\rn)$ if $q=\infty$) and 
\begin{equation*}
\norm{\sum_{j \in \ent} u_j}{\besw{p}{s}{q}{w}} \lesssim  \norm{\{2^{js} u_j\}_{j\in\ent}}{\ell^{q}(L^p(w))},
\end{equation*}
where the implicit constant depends only on $n,$ $\A,$ $s,$ $p$ and $q.$   An analogous statement, with $j\in\naz,$ holds true for $\ibesw{p}{s}{q}{w}$ (when $q=\infty,$  the convergence is in $\swp$).
\end{enumerate}
\end{theorem} 
 
 
 
 \section{Leibniz-type rules in weighted Triebel-Lizorkin and Besov spaces}
 
 \subsection{Homogeneous Leibniz-type rules}
 
 In the setting of weighted homogeneous Besov and Triebel-Lizorkin spaces we obtain Leibniz-type rules in Theorem \ref{thm:CM:TL:B}. As we will see in the corollaries to this result it improves the Leibniz-type rule (\ref{def:ileibniz}) and has extensions to weighted versions of (\ref{def:ileibniz}).
  
%  \begin{theorem}\label{thm:CM:TL:B}  For $m \in \re,$ let $\sigma(\xi,\eta),$ $\xi,\eta\in\rn,$ be a Coifman-Meyer multiplier of order $m.$ Consider  $0 < p, p_1, p_2  \le \infty$  such that $\hcline$ and  $0 < q \leq \infty;$ let  $w_1,w_2\in A_\infty$ and set $w=w_1^{{p}/{p_1}} w_2^{{p}/{p_2}}.$ 
%If $0 < p,p_1,p_2 < \infty$ and  $s > \tau_{p,q}(w),$  it holds that
%\begin{equation}\label{KP:CM:TL}
%\norm{T_\sigma(f,g)}{\tlw{p}{s}{q}{w}} \lesssim \norm{f}{\tlw{p_1}{s+m}{q}{w_1} } \norm{g}{H^{p_2}(w_2)} +  \norm{f}{H^{p_1}(w_1)}   \norm{g}{\tlw{p_2}{s+m}{q}{w_2} } \quad \forall f, g \in \swz.
%\end{equation}
%If $0< p, p_1,p_2\leq \infty$ and $s > \tau_p(w)$, it holds that
%\begin{equation}\label{KP:CM:B}
%\norm{T_\sigma(f,g)}{\besw{p}{s}{q}{w}} \lesssim \norm{f}{\besw{p_1}{s+m}{q}{w_1} } \norm{g}{H^{p_2}(w_2)} +  \norm{f}{H^{p_1}(w_1)}   \norm{g}{\besw{p_2}{s+m}{q}{w_2} } \quad \forall f, g \in \swz,
%\end{equation}
%where the Hardy spaces $H^{p_1}(w_1)$ and $H^{p_2}(w_2)$ must be replaced by $L^\infty$ if $p_1=\infty$ or $p_2=\infty,$ respectively.
%
%If $w_1=w_2$ then different pairs of $p_1, p_2$ can be used on the right-hand sides of \eqref{KP:CM:TL} and \eqref{KP:CM:B}; moreover, if $w\in A_\infty,$ then 
%\begin{equation}\label{KP:CM:TL2}
%\norm{T_\sigma(f,g)}{\tlw{p}{s}{q}{w}} \lesssim \norm{f}{\tlw{p}{s+m}{q}{w} } \norm{g}{L^\infty} +  \norm{f}{L^\infty}   \norm{g}{\tlw{p}{s+m}{q}{w}} \quad \forall f, g \in \swz,
%\end{equation}
%where $0<p<\infty,$ $0<q\le\infty$ and $s>\tau_{p,q}(w).$
%\end{theorem}

We note that if $m\geq 0$ then the above estimates hold for any $f,g\in\sw$ when $\sw$ is a subspace of the function spaces on the right-hand side. This is the case when $1<p_1,p_2<\infty$, $w_1\in A_{p_1}$, and $w_2\in A_{p_2}$ in (\ref{KP:CM:TL}) and (\ref{KP:CM:B}) and $w\in A_p$ for (\ref{KP:CM:TL2}). 

For $w \in A_\infty$ and $0<p\leq\infty$ we denote 
$\tau_w = \inf\{\tau\in(1,\infty]: w\in A_p\}$ and 
 $\tau_p(w) := n\left( \frac{1}{min(p/\tau_w,1)} - 1 \right).$  By the lifting property of weighted Besov and Triebel-Lizorkin spaces in Section \ref{TL_B_section} and their relation to weighted Hardy spaces (\ref{TL&Hardy}), the estimates (\ref{KP:CM:TL}) and (\ref{KP:CM:B}) imply the following Leibniz-type rule for Coifman-Meyer multipliers of order zero.

\begin{corollary}\label{coro:KP:CM:Hardy}  Let $\sigma(\xi,\eta),$ $\xi,\eta\in\rn,$ be a Coifman-Meyer multiplier of order $0.$ 
Consider  $0 < p, p_1, p_2  < \infty$  such that $\hcline;$ let  $w_1,w_2\in A_\infty$ and set $w=w_1^{{p}/{p_1}} w_2^{{p}/{p_2}}.$ 
If  $s > \tau_p(w),$ it holds that
\begin{equation}\label{KP:CM:Hardy}
\norm{D^s(T_\sigma(f,g))}{H^p(w)} \lesssim \norm{D^s f}{H^{p_1}(w_1)} \norm{g}{H^{p_2}(w_2)} +  \norm{f}{H^{p_1}(w_1)}   \norm{D^s g}{H^{p_2}(w_2)} \quad \forall f, g \in \swz.
\end{equation}
If $w_1=w_2$ then different pairs of $p_1, p_2$ can be used on the right-hand side of \eqref{KP:CM:Hardy}; moreover, if $w\in A_\infty,$ then 
\begin{equation}\label{Kp:CM:Hardy2}
\norm{D^s(T_\sigma(f,g))}{H^p(w)} \lesssim \norm{D^s f}{H^{p}(w)} \norm{g}{L^\infty} +  \norm{f}{L^\infty}   \norm{D^s g}{H^{p}(w)} \quad \forall f, g \in \swz,
\end{equation}
where $0<p<\infty$ and $s>\tau_{p}(w).$

\end{corollary}

Corollary \ref{coro:KP:CM:Hardy} gives estimates related to those in Brummer-Naibo \cite{BrNa2017}, where, using different methods, the following result was proven:

\textit{$\sigma$ is a Coifman-Meyer multiplier of order 0, $1<p_1,p_2\le \infty,$ $\frac{1}{2}<p<\infty,$ $\hcline,$  $w_1\in A_{p_1},$ $w_2\in A_{p_2},$ $w=w_1^{{p}/{p_1}} w_2^{{p}/{p_2}}$ and $s>\tau_{p,2}(w),$ then for all $f,g\in \sw$ it holds that
\begin{equation}\label{KP:CM:Lebesgue}
\norm{D^s(T_\sigma(f,g))}{L^p(w)} \lesssim \norm{D^s f}{L^{p_1}(w_1)} \norm{g}{L^{p_2}(w_2)} +  \norm{f}{L^{p_1}(w_1)}   \norm{D^s g}{L^{p_2}(w_2)}. 
\end{equation}
Moreover, if $w_1=w_2$ then different pairs of $p_1, p_2$ can be used on the right-hand side of \eqref{KP:CM:Lebesgue}.
}

Corollary \ref{coro:KP:CM:Hardy} and the result stated above from Bummer-Naibo compare as follows: 

\begin{itemize}
\item The estimate (\ref{KP:CM:Hardy}) allows for $0<p,p_1,p_2 < \infty$, $w_1,w_2 \in A_\infty$, and the $H^P(w)$ on the left-hand side if $s>\tau_p(w)$. On the other hand, (\ref{KP:CM:Lebesgue}) requires $1<p_1,p_2\le \infty,$ $w_1\in A_{p_1},$ $w_2\in A_{p_2}$ and the smaller $L^p(w)$ norm on the left-hand side when $s>\tau_{p,2}(w)$. Therefore, (\ref{KP:CM:Hardy}) is less restrictive than (\ref{KP:CM:Lebesgue}) in terms of the indices $p$, $p_1$, and $p_2$ and the classes that the weights $w_1$ and $w_2$ belong to. However, since $\tau_{p,2}(w) \leq \tau_p(w)$, (\ref{KP:CM:Hardy}) is more restrictive than (\ref{KP:CM:Lebesgue}) in terms of the range of the regularity $s$.
\item If $s>\tau_{p}(w),$   $1/2<p<\infty,$ $1<p_1,p_2<\infty$ such that $\hcline,$  $w_1\in A_{p_1}$ and $w_2\in A_{p_2}$, then \eqref{KP:CM:Hardy} implies \eqref{KP:CM:Lebesgue}. However if $\tau_p < \tau_p(w)$ then \eqref{KP:CM:Hardy} does not imply \eqref{KP:CM:Lebesgue} for $\tau_p < s < \tau_p(w)$. We next give examples of weights $w_1$ and $w_2$ such that the corresponding weight $w$ satisfies $\tau_p < \tau_p(w)$.  Let $1<p_1\leq p_2 <\infty$ and $w_1(x) = w_2(x) = w(x) = |x|^a$ with $n(r-1)<a<n(p_1-1)$ for some $1<r<p_1$. Then $w \in A_{p_1}$, $A_{p_1} \subset A_{p_2}$, and $w \notin A_r$. This implies that $1<\tau_w$ which leads to that $\tau_p < \tau_p(w)$. 
\item  The estimate (\ref{KP:CM:Lebesgue}) implies (\ref{Kp:CM:Hardy2}) for $1<p<\infty$, $w\in A_p$, and $s>\tau_{p,2}(w)$ and gives the endpoint estimate 
$$ \norm{D^s(T_\sigma(f,g))}{L^p(w)} \lesssim \norm{D^s f}{L^{\infty}} \norm{g}{L^{p}(w)} +  \norm{f}{L^{p}(w)}\norm{D^s g}{L^\infty}. $$
However, (\ref{Kp:CM:Hardy2}) allows $0<p<\infty$ and $w\in A_\infty$ if $s>\tau_p (w)$.
\end{itemize}

\subsection{Connection to Kato-Ponce inequalities}
Using the equivalences \ref{lifting} and following discussion we obtain the following corollary to Theorem \ref{thm:CM:TL:B} by setting $\sigma \equiv 1$. 

\begin{corollary}\label{coro:KP:TL:B}  Consider  $0 < p, p_1, p_2  \le \infty$  such that $\hcline$ and  $0 < q \leq \infty;$ let  $w_1,w_2\in A_\infty$ and set $w=w_1^{{p}/{p_1}} w_2^{{p}/{p_2}}.$ 
If $0 < p ,p_1,p_2< \infty$ and  $s > \tau_{p,q}(w),$ it holds that
\begin{equation}\label{KP:TL}
\norm{fg}{\tlw{p}{s}{q}{w}} \lesssim \norm{f}{\tlw{p_1}{s}{q}{w_1}} \norm{g}{H^{p_2}(w_2)} +  \norm{f}{H^{p_1}(w_1)}   \norm{g}{\tlw{p_2}{s}{q}{w_2}} \quad \forall f, g \in \swz.
\end{equation}
If $0 < p, p_1,p_2 \le \infty$ and $s > \tau_p(w)$, it holds that
\begin{equation}\label{KP:B}
\norm{fg}{\besw{p}{s}{q}{w}} \lesssim \norm{f}{\besw{p_1}{s}{q}{w_1}} \norm{g}{H^{p_2}(w_2)} +  \norm{f}{H^{p_1}(w_1)}   \norm{g}{\besw{p_2}{s}{q}{w_2}} \quad \forall f, g \in \swz,
\end{equation}
where the Hardy spaces $H^{p_1}(w_1)$ and $H^{p_2}(w_2)$ must be replaced by $L^\infty$ if $p_1=\infty$ or $p_2=\infty,$ respectively.

If $w_1=w_2$ then different pairs of $p_1, p_2$ can be used on the right-hand sides of \eqref{KP:TL} and \eqref{KP:B}; moreover, if $w\in A_\infty,$ then 
\begin{equation}\label{KP:TL2}
\norm{fg}{\tlw{p}{s}{q}{w}} \lesssim \norm{f}{\tlw{p}{s}{q}{w}} \norm{g}{L^\infty} +  \norm{f}{L^\infty}   \norm{g}{\tlw{p}{s}{q}{w}} \quad \forall f, g \in \swz,
\end{equation}
where $0<p<\infty,$ $0<q\le \infty$ and $s>\tau_{p,q}(w).$
\end{corollary}


In particular if we set $q=2$ and use the connection between weighted Hardy spaces and weighted Triebel-Lizorkin spaces (\ref{TL&Hardy}) we obtain the following corollary. 

\begin{corollary}\label{coro:KP:Hardy} 
Consider  $0 < p, p_1, p_2  < \infty$  such that $\hcline;$ let  $w_1,w_2\in A_\infty$ and set $w=w_1^{{p}/{p_1}} w_2^{{p}/{p_2}}.$ 
If  $s > \tau_{p}(w),$ it holds that
\begin{equation}\label{KP:Hardy}
\norm{D^s(fg)}{H^p(w)} \lesssim \norm{D^s f}{H^{p_1}(w_1)} \norm{g}{H^{p_2}(w_2)} +  \norm{f}{H^{p_1}(w_1)}   \norm{D^s g}{H^{p_2}(w_2)} \quad \forall f, g \in \swz.
\end{equation}
If $w_1=w_2$ then different pairs of $p_1, p_2$ can be used on the right-hand side of \eqref{KP:Hardy}; moreover, if $w\in A_\infty,$ then 
\begin{equation*}
\norm{D^s(fg)}{H^p(w)} \lesssim \norm{D^s f}{H^{p}(w)} \norm{g}{L^\infty} +  \norm{f}{L^\infty}   \norm{D^s g}{H^{p}(w)} \quad \forall f, g \in \swz,
\end{equation*}
where $0<p<\infty$ and $s>\tau_{p}(w).$
\end{corollary}
 
 
 \subsection{Proof of Theorem \ref{thm:CM:TL:B}}
 
The following lemma will be useful in the proofs of Theorems \ref{thm:CM:TL:B} and \ref{thm:ICM:TL:B}.

\begin{lemma}\label{lem:pointineq} Let $\phi_1,\phi_2\in \sw$ be  such that $\widehat{\phi_1}$ and $\widehat{\phi_2}$ have compact supports and  $\widehat{\phi_1}\widehat{\phi_2}=\widehat{\phi_1}.$  If $0<r\le 1$ and $\varepsilon>0,$ it holds that
\begin{align*}
\abs{P^{\tau_a\phi_1}_{j}f(x)}\lesssim (1+\abs{a})^{\varepsilon+\frac{n}{r}} \M_r(P^{\phi_2}_{j}f)(x)\quad \forall x,a\in\rn, j\in\ent, f\in\sw.
\end{align*}
\end{lemma}

\begin{proof} This estimate is a consequence of Lemma~\ref{coro:2:11}. In view of the supports of $\widehat{\phi_1}$ and $\widehat{\phi_2}$ we have $P^{\tau_a\phi_1}_{j}f=P^{\tau_a\phi_1}_{j}P^{\phi_2}_{j}f$ for  $j\in \ent $ and $f\in \sw.$ Applying Lemma~\ref{coro:2:11} with $\phi(x)=2^{nj}\tau_a \phi_1(2^j x),$ $A=2^j,$ $R\ge 1$ such that $\supp(\widehat{\phi_2})\subset \{\xi\in\rn:\abs{\xi}\le R\}$ and $d=\varepsilon+n/r,$ we get
\begin{align*}
\abs{P^{\tau_a\phi_1}_{j}f(x)}&\lesssim R^{n(\frac{1}{r}-1)} 2^{-jn}\norm{(1+\abs{2^j\cdot})^{\varepsilon+\frac{n}{r}}2^{nj}\tau_a \phi_1(2^j\cdot)}{L^\infty}\M_r(P^{\phi_2}_{j}f)(x)\\
&\sim \norm{(1+\abs{2^j\cdot})^{\varepsilon+\frac{n}{r}}\tau_a \phi_1(2^j\cdot)}{L^\infty}\M_r(P^{\phi_2}_{j}f)(x)\quad \forall x,a\in \rn, j\in \ent, f\in\sw.
\end{align*}
Since $\phi_1\in \sw,$ 
\[
\abs{\tau_a \phi_1(2^jx)}=\abs{\phi_1(2^j x +a)}\lesssim \frac{(1+\abs{a})^{\varepsilon+\frac{n}{r}}}{(1+\abs{2^jx})^{\varepsilon+\frac{n}{r}}}\quad \forall x,a\in \re,j\in \ent.
\]
Combining these two estimates completes the proof.
\end{proof}

We now prove Theorem \ref{thm:CM:TL:B}.
 
 \begin{proof}[Proof of Theorem~\ref{thm:CM:TL:B}] Consider $\Phi,$ $\Psi,$ $T_\sigma^1,$ $T_\sigma^2,$ $\{\C_j(a,b)\}_{j\in\ent,a,b\in\ent^n}$ as in Section~\ref{sec:Coifman-Meyer Multipliers}. Let $m,$ $\sigma,$ $p,$ $p_1,$ $p_2,$ $q,$ $s,$ $w_1,$ $w_2$ and $w$ be as in the hypotheses.  
For ease of notation, $p_1$ and $p_2$ will be assumed to be finite; the same proof applies for \eqref{KP:CM:B} if that is not the case, and for \eqref{KP:CM:TL2}.


We next prove \eqref{KP:CM:TL} and \eqref{KP:CM:B}. Here we will only work with $T_\sigma^1$ as the estimate for $T_\sigma^2$ is shown through symmetry. Hence we will prove that 
 \begin{align*}
 \norm{T^{1}_\sigma(f,g)}{\dot{F}^s_{p,q}(w)} \lesssim  \norm{f}{\dot{F}^{s+m}_{p_1, q}(w_1)} \norm{g}{H^{p_2}(w_2)}\quad  \text{ and }\quad 
 \norm{T^1_\sigma(f,g)}{\dot{B}^s_{p,q}(w)} \lesssim  \norm{f}{\dot{B}^{s+m}_{p_1, q}(w_1)} \norm{g}{H^{p_2}(w_2)}.
\end{align*}
  Moreover, since $\norm{\sum f_j}{\tlw{p}{s}{q}{w}}^{\min(p,q,1)}\lesssim \sum\norm{f_j}{\tlw{p}{s}{q}{w}}^{\min(p,q,1)}$  and similarly for $\besw{p}{s}{q}{w}$, it suffices to prove that, given $\varepsilon>0$ there exist $0<r_1,r_2\le 1$  such that for all $g\in \sw$ and  $f\in \swz$ ($f\in \sw\cap \tlw{p}{s}{q}{w}$ or  $f\in \sw\cap \besw{p}{s}{q}{w}$ if $m\ge 0$),  it holds that
\begin{align}
 \norm{T^{a,b}(f,g)}{\dot{F}^s_{p,q}(w)} \lesssim (1+\abs{a})^{\varepsilon+\frac{n}{r_1}}  (1+\abs{b})^{\varepsilon+\frac{n}{r_2}} \norm{f}{\dot{F}^{s+m}_{p_1, q}(w_1)} \norm{g}{H^{p_2}(w_2)}\label{eq:estbbTL},\\
 \norm{T^{a,b}(f,g)}{\dot{B}^s_{p,q}(w)} \lesssim (1+\abs{a})^{\varepsilon+\frac{n}{r_1}}  (1+\abs{b})^{\varepsilon+\frac{n}{r_2}} \norm{f}{\dot{B}^{s+m}_{p_1, q}(w_1)} \norm{g}{H^{p_2}(w_2)}\label{eq:estbbB},
\end{align}
where
\[
T^{a,b}(f,g):=\sum_{j\in\ent} \C_j(a,b) \,(\Do{ \tau_a \Psi }{j} f)\, (\So{ \tau_b \Phi }{j} g )
\]
and the implicit constants are independent of $a$ and $b.$  We will assume $q$ finite; obvious changes apply if that is not the case.

In view of the supports of $\Psi$ and $\Phi$ we have that 
\begin{equation*}
\supp (\mathcal{F}[\C_j(a,b) \,(\Do{ \tau_a \Psi }{j} f ) \, ( \So{ \tau_b \Phi }{j} g )])  \subset \{\xi \in \rn: |\xi| \lesssim 2^j \} \quad \forall j \in \ent,\,a,b\in \ent^n.
\end{equation*}

For \eqref{eq:estbbTL},
Theorem \ref{thm:Nikolskij:weighted}\eqref{item:thh:Nikolskij:TL}, the bound \eqref{eq:cjbound} for $\C_j(a,b)$, and H\"older's inequality  imply
\begin{align*}
\norm{T^{a,b}(f,g)}{\dot{F}^s_{p,q}(w)} & \lesssim \norm{\{2^{sj} \C_j(a,b) \,(\Do{ \tau_a \Psi }{j} f ) \, (\So{ \tau_b \Phi }{j} g)\}_{j\in\ent} }{L^p(w)(\ell^q)}\\
& \lesssim \norm{\left(\sum\limits_{j \in \ent}  2^{(s +m) q j}  |(\Do{ \tau_a \Psi }{j} f )(x) \, (\So{ \tau_b \Phi }{j} g)|^q   \right)^\frac{1}{q}}{L^p(w)}\\
& \le\norm{\sup\limits_{j \in \ent} |(\So{ \tau_b \Phi }{j} g)| \left(\sum\limits_{j \in \ent}  2^{(s +m) q j}  |(\Do{ \tau_a \Psi }{j} f )|^q   \right)^\frac{1}{q}}{L^p(w)}\\
& \le \norm{\left(\sum_{j \in \ent}  2^{(s +m) q j}  |\Do{ \tau_a \Psi }{j} f |^q   \right)^\frac{1}{q}}{L^{p_1}(w_1)} \norm{\sup\limits_{j \in \ent} |\So{ \tau_b \Phi }{j} g|}{L^{p_2}(w_2)}.
\end{align*}
Consider $\varphi,\psi\in\sw$ as in Section~\ref{TL_B_section} such that   $\widehat{\varphi}\equiv 1$ on $\supp(\widehat{\Phi})$ and  $\widehat{\psi}\equiv 1$ on $\supp(\widehat{\Psi}).$  Let   $0<r_1<\min(1, p_1/\tau_{w_1},q)$; by Lemma~\ref{lem:pointineq} and the weighted Fefferman-Stein inequality  we have that  
\begin{align*}
\norm{\left(\sum_{j \in \ent}  2^{(s +m) q j}  |(\Do{ \tau_a \Psi }{j} f )|^q   \right)^\frac{1}{q}}{L^{p_1}(w_1)}&\lesssim (1+\abs{a})^{\varepsilon+\frac{n}{r_1}}
\norm{\left(\sum_{j \in \ent}  2^{(s +m) q j}  |\M_{r_1}(\Do{\psi }{j} f) |^q   \right)^\frac{1}{q}}{L^{p_1}(w_1)}\\
&\lesssim (1+\abs{a})^{\varepsilon+\frac{n}{r_1}} \norm{\left(\sum_{j \in \ent}  2^{(s +m) q j}  |\Do{\psi }{j} f|^q   \right)^\frac{1}{q}}{L^{p_1}(w_1)}\\
&\sim (1+\abs{a})^{\varepsilon+\frac{n}{r_1}}  \norm{f}{\dot{F}^{s+m}_{p,q}(w_1)},
\end{align*}
where the implicit constants are independent of $a$ and $f.$ Next, let  $0<r_2<\min(1,p_2/\tau_{w_2})$; by Lemma~\ref{lem:pointineq} and the boundedness properties of the Hardy-Littlewood maximal operator on weighted Lebesgue space  we have that  
\begin{align*}
\norm{\sup_{j\in\ent}|\So{\tau_b\Phi}{j}g|}{L^{p_2}(w_2)}&\lesssim (1+\abs{b})^{\varepsilon+\frac{n}{r_2}} \norm{\M_{r_2}(\sup_{j\in\ent}|\So{\varphi}{j}g|)}{L^{p_2}(w_2)}\\
&\lesssim (1+\abs{b})^{\varepsilon+\frac{n}{r_2}} \norm{\sup_{j\in\ent}|\So{\varphi}{j}g|}{L^{p_2}(w_2)}\\
&\sim (1+\abs{b})^{\varepsilon+\frac{n}{r_2}} \norm{g}{H^{p_2}(w_2)},
\end{align*}
where the implicit constants are independent of $b$ and $g.$ Putting all together we obtain \eqref{eq:estbbTL}.


For \eqref{eq:estbbB},  Theorem \ref{thm:Nikolskij:weighted}\eqref{item:thh:Nikolskij:B}, the bound \eqref{eq:cjbound} for $\C_j(a,b)$ and H\"older's inequality  give
\begin{align*}
\norm{T^{a,b}(f,g)}{\dot{B}^s_{p,q}(w)} & \lesssim \norm{\{2^{sj} \C_j(a,b) \,(\Do{ \tau_a \Psi }{j} f ) \, (\So{ \tau_b \Phi }{j} g)\}_{j\in\ent} }{\ell^q(L^p(w))}\\
& \lesssim \left(\sum\limits_{j \in \ent}  2^{(s +m) q j}  \norm{(\Do{ \tau_a \Psi }{j} f ) \, (\So{ \tau_b \Phi }{j} g)}{L^p(w)}^q   \right)^\frac{1}{q}  \\
&\le  \left(\sum\limits_{j \in \ent}  2^{(s +m) q j}  \norm{(\Do{ \tau_a \Psi }{j} f ) }{L^{p_1}(w_1)}^q   \right)^\frac{1}{q}  \norm{\sup_{k\in \ent}|\So{ \tau_b \Phi }{k} g|}{L^{p_2}(w_2)}\\
& \lesssim  (1+\abs{a})^{\varepsilon+\frac{n}{r_1}}  (1+\abs{b})^{\varepsilon+\frac{n}{r_2}}  \norm{f}{\dot{B}^{s+m}_{p_1, q}(w_1)} \norm{g}{H^{p_2}(w_2)},
\end{align*}
where in the last inequality we have used Lemma~\ref{lem:pointineq} and the boundedness properties of $\M$ with  $0<r_j<\min(1,p_j/\tau_{w_j})$ for $j=1,2$ .


It is clear from the proof above that if $w_1=w_2,$ then  different pairs of $p_1, p_2$ related to $p$ through the H\"older condition can be used on the right-hand sides of \eqref{KP:CM:TL} and \eqref{KP:CM:B}; in such case $w=w_1=w_2.$  
\end{proof}


We now remark in the condition \ref{eq:CMm} and its importance in the proof of Theorem \ref{thm:CM:TL:B}.
For convergence purposes, the relations between $N$ in \eqref{eq:decompT1} and the powers  $\varepsilon+n/r_1$ and $\varepsilon+n/r_2$ in \eqref{eq:estbbTL} and   \eqref{eq:estbbB} must be such that $(N-\varepsilon-n/r_1)\,r^*>n$ and $(N-\varepsilon-n/r_2)\,r^*>n,$ where $r^*=\min(p,q,1).$ Moreover,  $r_1$ and $r_2$ were selected so that $0<r_j<\min(1, p_j/\tau_{w_j},q)$ in the context of  Triebel--Lizorkin spaces and  $0<r_j<\min(1,p_j/\tau_{w_j})$ in the context of Besov spaces. Therefore, if  $N>n(1/r^*+1/\min(1, p_1/\tau_{w_1},p_2/\tau_{w_2},q))$ in the Triebel--Lizorkin setting and $N>n(1/r^*+1/\min(1, p_1/\tau_{w_1},p_2/\tau_{w_2}))$ in the Besov setting,  $\varepsilon,$  $r_1$ and  $r_2$ can be chosen so that all the conditions above are satisfied. In view of this and Remark~\ref{re:numderiv1},  the multiplier $\sigma$ in Theorem~\ref{thm:CM:TL:B}  needs only satisfy  \eqref{eq:CMm} for $\abs{\alpha+\beta}\le 2( [n(1/r^*+1/\min(1,p_1/\tau_{w_1},p_2/\tau_{w_2},q))]+1)=\gamt$ in the Triebel--Lizorkin case and $\abs{\alpha+\beta}\le 2( [n(1/r^*+1/\min(1, p_1/\tau_{w_1},p_2/\tau_{w_2}))]+1)=\gamb$ in the Besov case. An analogous observation follows for the multiplier $\sigma$ in Theorem~\ref{thm:ICM:TL:B} in relation to the condition  obtained from \eqref{eq:CMm} with $\abs{\xi}+\abs{\eta}$ replaced by $1+\abs{\xi}+\abs{\eta}.$ 


 \subsection{Inhomogeneous Leibniz-type rules}
 In this section we obtain Leibniz-type rules for Coifman-Meyer multiplier operators associated to inhomogeneous symbols. Our main result is the inhomogenous counterpart to Theorem \ref{thm:CM:TL:B}. 
 
\begin{theorem}\label{thm:ICM:TL:B}  For $m \in \re,$ let $\sigma(\xi,\eta),$ $\xi,\eta\in\rn,$ be an inhomogeneous Coifman-Meyer multiplier of order $m.$ Consider  $0 < p, p_1, p_2  \le \infty$  such that $\hcline$ and  $0 < q \leq \infty;$ let  $w_1,w_2\in A_\infty$ and set $w=w_1^{{p}/{p_1}} w_2^{{p}/{p_2}}.$ 
If $0 < p,p_1,p_2 < \infty$ and  $s > \tau_{p,q}(w),$  it holds that
\begin{equation}\label{KP:ICM:TL}
\norm{T_\sigma(f,g)}{\itlw{p}{s}{q}{w}} \lesssim \norm{f}{\itlw{p_1}{s+m}{q}{w_1}} \norm{g}{h^{p_2}(w_2)} +  \norm{f}{h^{p_1}(w_1)}   \norm{g}{\itlw{p_2}{s+m}{q}{w_2}} \quad \forall f, g \in \sw.
\end{equation}
If $0< p, p_1,p_2\leq \infty$ and $s > \tau_p(w)$, it holds that
\begin{equation}\label{KP:ICM:B}
\norm{T_\sigma(f,g)}{\ibesw{p}{s}{q}{w}} \lesssim \norm{f}{\ibesw{p_1}{s+m}{q}{w_1} } \norm{g}{h^{p_2}(w_2)} +  \norm{f}{h^{p_1}(w_1)}   \norm{g}{\ibesw{p_2}{s+m}{q}{w_2} } \quad \forall f, g \in \sw,
\end{equation}
where the local Hardy spaces $h^{p_1}(w_1)$ and $h^{p_2}(w_2)$ must be replaced by $L^\infty$ if $p_1=\infty$ or $p_2=\infty,$ respectively.

If $w_1=w_2$ then different pairs of $p_1, p_2$ can be used on the right-hand sides of \eqref{KP:ICM:TL} and \eqref{KP:ICM:B}; moreover, if $w\in A_\infty,$ then 
\begin{equation*}
\norm{T_\sigma(f,g)}{\itlw{p}{s}{q}{w}} \lesssim \norm{f}{\itlw{p}{s+m}{q}{w} } \norm{g}{L^\infty} +  \norm{f}{L^\infty}   \norm{g}{\itlw{p}{s+m}{q}{w}} \quad \forall f, g \in \sw,
\end{equation*}
where $0<p<\infty,$ $0<q\le\infty$ and $s>\tau_{p,q}(w).$
\end{theorem} 

The proof of Theorem \ref{thm:ICM:TL:B} follows along the same lines as the proof of Theorem \ref{thm:CM:TL:B}. 
 
The corollaries that follow are similar to those in the homogeneous setting. As an example we state the inhomogeneous counterpart to Corrolary \ref{coro:KP:CM:Hardy}.
 
 
 \begin{corollary}\label{coro:KP:CM:Hardyloc}  Let $\sigma(\xi,\eta),$ $\xi,\eta\in\rn,$ be an inhomogeneous Coifman-Meyer multiplier of order $0.$ 
Consider  $0 < p, p_1, p_2  < \infty$  such that $\hcline;$ let  $w_1,w_2\in A_\infty$ and set $w=w_1^{{p}/{p_1}} w_2^{{p}/{p_2}}.$ 
If  $s > \tau_p(w),$ it holds that
\begin{equation*}
\norm{J^s(T_\sigma(f,g))}{h^p(w)} \lesssim \norm{J^s f}{h^{p_1}(w_1)} \norm{g}{h^{p_2}(w_2)} +  \norm{f}{h^{p_1}(w_1)}   \norm{J^s g}{h^{p_2}(w_2)} \quad \forall f, g \in \sw.
\end{equation*}
If $w_1=w_2$ then different pairs of $p_1, p_2$ can be used on the right-hand side of \eqref{KP:CM:Hardy}; moreover, if $w\in A_\infty,$ then 
\begin{equation*}
\norm{J^s(T_\sigma(f,g))}{h^p(w)} \lesssim \norm{J^s f}{h^{p}(w)} \norm{g}{L^\infty} +  \norm{f}{L^\infty}   \norm{J^s g}{h^{p}(w)} \quad \forall f, g \in \sw,
\end{equation*}
where $0<p<\infty$ and $s>\tau_{p}(w).$
\end{corollary}
 
 \section{Leibniz rules in other functions spaces}
 The method used to prove Theorem (\ref{thm:CM:TL:B}) is quite versatile and can be applied to Triebel-Lizorkin and Besov spaces that are based in other quasi-Banach spaces. Here we highlight the important features of Lebesgue spaces that are necessary for the proof of Theorem (\ref{thm:CM:TL:B}) to be adapted to other settings. We also define Triebel-Lizorkin and Besov spaces that are based in other quasi-Banach spaces. 
 
The main features of weighted Triebel-Lizorkin and Besov spaces used in the proof of Theorem (\ref{thm:CM:TL:B}) are the following:
\begin{enumerate}
\item[(i)]\label{item:first} there exists $r>0$ such that $\norm{f+g}{\itlw{p}{s}{q}{w}}^r\le \norm{f}{\itlw{p}{s}{q}{w}}^r+\norm{g}{\itlw{p}{s}{q}{w}}^r;$ similarly for the weighted inhomogeneous  Besov spaces and the weighted homogeneous  Triebel--Lizorkin and Besov spaces;
\item[(ii)] \label{item:second} H\"older's inequality in weighted Lebesgue spaces;
\item[(iii)] \label{item:third}  the boundedness properties in weighted Lebesgue spaces  of the Hardy--Littlewood maximal operator (for the Besov space setting) and the weighted Fefferman--Stein inequality (for the Triebel--Lizorkin space setting);
\item[(iv)] \label{item:last} Nikol'skij representations for weighted Triebel--Lizorkin and Besov spaces (Theorem~\ref{thm:Nikolskij:weighted}).
\end{enumerate}

In the followinig subsections we consider quasi-Banach spaces $\mathcal{X}$ such that properties (\ref{item:first})--(\ref{item:last}) hold for the homogeneous and inhomogeneous $\mathcal{X}$-based Triebel-Lizorkin and Besov spaces. Corresponding versions of Theorems (\ref{thm:CM:TL:B}) and \ref{thm:ICM:TL:B} hold in Triebel-Lizorkin and Besov spaces based in these spaces. The homogeneous $\chi$-based Triebel-Lizorkin and Besov spaces denoted by $\tl{\mathcal{X}}{s}{q}$ and $\bes{\mathcal{X}}{s}{q}$ respectively are defined similarly to the weighted, homogeneous Triebel-Lizorkin and Besov spaces with the $||\cdot||_{L^p(w)}$ quasi-norm replaced with the $||\cdot||_\mathcal{X}$ quasi-norm. The inhomogeneous spaces are defined similarly. 

 \subsection{Leibniz-type rules in the setting of Lorentz-based Triebel-Lizorkin and Besov spaces}
 For $0<p<\infty$ and $0<t\le \infty$ or $p=t=\infty,$ and an $A_\infty$ weight $w$ defined on $\rn,$  we denote by $L^{p,t}(w)$ the weighted Lorentz space consisting of complex-valued, measurable functions $f$ defined on $\rn$ such that
\[
\|f\|_{\lebw{p,t}{w}}=\left(\int_0^\infty \left(\tau^{\frac{1}{p}} f_w^*(\tau)\right)^t\,\frac{d\tau}{\tau}\right)^{\frac{1}{t}}<\infty,
\]
where $f^*_w(\tau)=\inf\{\lambda\ge 0:w_f(\lambda)\le \tau\}$ with
$w_f(\lambda)=w(\{x\in\rn : \abs{f(x)}>\lambda\}).$ 
In the case that $t=\infty$, $\|f\|_{\lebw{p,t}{w}}= \sup_{t>0}t^{\frac{1}{p}}f_w^*(t)$. We note that for $p=t$ we have $\lebw{p,p}{w} = \lebw{p}{w}$ and $\lebw{1,\infty}{w}$ is the weighted weak $L^1(w)$ space. For more details on these spaces and their properties see Hunt \cite{MR0223874}. 

We now turn our attention to the analogues to properties (\ref{item:first})-(\ref{item:last}) in the setting of weighted Lorrentz-based Triebel-Lizorkin and Besov spaces. 

Property (\ref{item:first}) follows from the work of Hunt \cite{MR0223874} that the quasi norm $||\cdot||_{\lebw{p,t}{w}}$ is comparable to a quasi-norm $|||\cdot|||_{\lebw{p,t}{w}}$ that is subadditive. Therefore we have that the norms $|||\cdot|||_{\tl{(p,t)}{s}{q}}$ and $||\cdot||_{\tl{(p,t)}{s}{q}}$ are comparable where $|||\cdot|||_{\tl{(p,t)}{s}{q}}$ is the quasi-norm with $||\cdot||_{\lebw{p,t}{w}}$ replaced by $|||\cdot|||_{\lebw{p,t}{w}}$ and for some $r>0$ $|||f+g|||^r_{\tl{(p,t)}{s}{q}} \leq |||f|||^r_{\tl{(p,t)}{s}{q}} + |||g|||^r_{\tl{(p,t)}{s}{q}}$. 

By using the following version of H\"older's inequality we obtain property (\ref{item:second}) [Theorem 4.5, \cite{MR0223874}]: Let $f,g\in \lebw{p,t}{w}$. Then for $\frac{1}{p} = \frac{1}{p_1} + \frac{1}{p_2}$ and $\frac{1}{t} = \frac{1}{t_1} + \frac{1}{t_2}$ it holds that \[||fg||_{\lebw{p,t}{w}} \lesssim ||f||_{\lebw{p_1,t_1}{w}}||g||_{\lebw{p_2,t_2}{w}} .\]

The corresponding version of property (\ref{item:third}) is: 
 If $0<p<\infty,$  $0<t,q\le\infty,$  $0<r<\min(p/\tau_w,q)$ and $0<r\le t,$ it holds that
\begin{equation}\label{eq:FTLorentz}
\norm{\left(\sum_{j\in\ent} \abs{\M_r(f_j)}^q\right)^{\frac{1}{q}}}{\lebw{p,t}{w}}\lesssim \norm{\left(\sum_{j\in\na_0} \abs{f_j}^q\right)^{\frac{1}{q}}}{\lebw{p,t}{w}}\quad \forall \{f_j\}_{j\in \na_0}\in \lebw{p,t}{w}(\ell^q);  
\end{equation}
in particular, if $0<r<p/\tau_w$ and $0<r\le t,$ it holds that 
\begin{equation*}
\norm{\M_r(f)}{\lebw{p,t}{w}}\lesssim \norm{f}{\lebw{p,t}{w}}\quad \forall f\in \lebw{p,t}{w}. 
\end{equation*}
This is established by the extrapolation theorem \cite[Theorem 4.10 and comments on page 70]{MR2797562} when $r=1,$ $1<p<\infty,$  $1\le t\le \infty$ and $1<q\le \infty.$ The remaining cases are shown using the previous cases and the following scaling property for Lorrentz spaces: For $0<s<\infty$ $\norm{\abs{f}^s}{\lebw{p,t}{w}}=\norm{f}{\lebw{sp,st}{w}}^s$.

The substitute for property \ref{item:last} is the following Nikols'kij representation in weighted Lorrentz spaces.

\begin{theorem}\label{thm:Nikolskij:Lorrentz:weighted} For $\A> 0,$ let $\{u_j\}_{j \in \naz} \subset \mathcal{S}'(\rn)$ be a sequence of tempered distributions such that
\begin{equation*}
\supp(\widehat{u_j}) \subset B(0, \A\, 2^j ) \quad \forall j \in \naz.
\end{equation*}
If $w\in A_\infty,$ then the following holds:  
\begin{enumerate}
\item[(i)]\label{item:thh:Nikolskij:Lorrentz:TL} Let $0 < p < \infty$, $0 < t,q \leq \infty$ and $s > \left(\frac{1}{\min(p/\tau_w,t,q,1)} - 1 \right)$. If $\norm{\{2^{js} u_j\}_{j\in\ent}}{L^{(p,t)}(w)(\ell^{q})} < \infty$, then the series $\sum_{j \in \naz} u_j$ converges in $\itlw{(p,t)}{s}{q}{w}$ (in $\mathcal{S}'(\rn)$ if $q=\infty$) and 
\begin{equation*}
\norm{\sum_{j \in \ent} u_j}{\itlw{(p,t)}{s}{q}{w}} \lesssim  \norm{\{2^{js} u_j\}_{j\in\naz}}{L^{(p,t)}(w)(\ell^{q})},
\end{equation*}
where the implicit constant depends only on $n,$ $\A,$ $s,$ $p$ and  $q.$
\item[(ii)]\label{item:thh:Nikolskij:Lorrentz:B} Let $0 < p, q \leq \infty$ and $s > \tau_{p,t}(w)$. If $\norm{\{2^{js} u_j\}_{j\in\naz}}{\ell^{q}(L^{(p,t)}(w))} < \infty$, then the series $\sum_{j \in \naz} u_j$ converges in  $\ibesw{(p,t)}{s}{q}{w}$ (in $\mathcal{S}'(\rn)$ if $q=\infty$) and 
\begin{equation*}
\norm{\sum_{j \in \ent} u_j}{\ibesw{(p,t)}{s}{q}{w}} \lesssim  \norm{\{2^{js} u_j\}_{j\in\naz}}{\ell^{q}(L^{(p,t)}(w))},
\end{equation*}
where the implicit constant depends only on $n,$ $\A,$ $s,$ $p$ and $q.$
\end{enumerate}
\end{theorem}
For the proof of Theorem \ref{thm:Nikolskij:Lorrentz:weighted} we need the following lemmas.

\begin{lemma}[Particular case of Corollary 2.11 in \cite{MR837335}]\label{coro:2:11}
Suppose $0 < r \leq 1,$  $A >0$, $R \geq 1$ and $\ex > n/r$. If $\phi \in \sw$ and $f$ is such that  $\supp(\widehat{f}) \subset \{\xi\in\rn:\abs{\xi}\le AR\}$, it holds that
\begin{equation*}
|\phi * f(x)| \lesssim R^{n (\frac{1}{r}  -1)} A^{-n} \norm{(1 + |A \cdot|)^\ex \phi}{L^\infty}  \M_rf(x) \quad \forall x \in \rn,
\end{equation*}
where the implicit constant is independent of $A, R, \phi,$ and $f.$  
\end{lemma}
\begin{remark} \cite[Corollary 2.11]{MR837335} incorrectly states $A^{-n/r}$ instead of $ A^{-n}$. Also, it states $A \geq 1$, but the result is true for $A >0$ as stated in Lemma~\ref{coro:2:11}.
\end{remark}

\begin{lemma}\label{coro:2:12(1)}
Suppose $w\in A_\infty,$ $0 < p \leq \infty$, $A > 0$, $R \geq 1,$  and $\ex > b > n/ \min(1,p/\tau_w).$  If $\phi \in \sw$ and $f$ is such that $\supp(\widehat{f}) \subset \{\xi\in\rn:\abs{\xi}\le AR\}$, it holds that
\begin{equation*}
\norm{\phi * f}{\lebw{p}{w}} \lesssim R^{b - n} A^{-n} \norm{(1 + |A \cdot|^\ex) \phi}{L^\infty} \norm{f}{\lebw{p}{w}},
\end{equation*}
where the implicit constant is independent of $A,$ $R,$ $\phi$ and $f.$ 
\end{lemma}

\begin{proof}  Set $r:=n/b < \min(1, p/\tau_w).$ The hypothesis $\ex > b$  means $ \ex  > n/r$ and Lemma~\ref{coro:2:11}  yields
$$
|\phi * f(x)| \lesssim R^{n (\frac{1}{r} -1)} A^{-n} \norm{(1 + |A \cdot|)^\ex \phi}{L^\infty}  \M_rf(x) \quad \forall x \in \rn.
$$
Since $r < p/\tau_w,$ we have   $\norm{\M_rf}{\lebw{p,t}{w}} \lesssim \norm{f}{\lebw{p,t}{w}}$ and therefore
$$
\norm{\phi * f}{\lebw{p,t}{w}} \lesssim R^{n (\frac{1}{r} -1)} A^{-n} \norm{(1 + |A \cdot|)^\ex \phi}{L^\infty} \norm{f}{\lebw{p,t}{w}};
$$
observing that $1/r -1 = (b -n)/n$, the desired estimate follows. 
\end{proof}

The following lemma is a modified version of \cite[Lemma 3.8]{MR837335}.
\begin{lemma}\label{eq:seriesineq} Let  $\tau < 0$, $\lambda \in \re$, $0 < q \leq \infty$, and $k_0 \in \ent$. Then, for any sequence $\{d_j\}_{j \in \ent} \subset [0, \infty)$ it holds that
\begin{equation*}
\norm{\left\{\sum_{k=k_0}^\infty 2^{\tau k} 2^{\lambda(j+k)}d_{j+k}\right\}_{j\in\ent}}{\ell^q} \lesssim \norm{\{2^{j \lambda} d_j\}_{j\in\ent}}{\ell^q},
\end{equation*}
where the implicit constant depends only on $k_0, \tau, \lambda$ and $q$.
\end{lemma}

\begin{proof}
Suppose first that $0 < q \leq 1$. Then,
\begin{align*}
& \norm{\left\{\sum_{k=k_0}^\infty 2^{\tau k} 2^{\lambda(j+k)}d_{j+k}\right\}_{j\in \ent}}{\ell^q} = \left[\sum_{j \in \ent} \left( \sum_{k=k_0}^\infty 2^{\tau k} 2^{\lambda(j+k)}d_{j+k} \right)^q \right]^\frac{1}{q}\\
& \quad\quad\quad\quad\quad \leq  \left[\sum_{j \in \ent}  \sum_{k=k_0}^\infty 2^{\tau q k} 2^{\lambda q (j+k)}d_{j+k}^q \right]^\frac{1}{q} =  \left[ \sum_{k=k_0}^\infty  2^{\tau q k} \sum_{j \in \ent} 2^{\lambda q (j+k)}d_{j+k}^q \right]^\frac{1}{q}\\
&  \quad\quad\quad \quad \quad = \left(  \sum_{k=k_0}^\infty   2^{\tau q k}     \right)^\frac{1}{q} \norm{\{2^{j \lambda} d_j\}_{j\in \ent}}{\ell^q} = C_{k_0, \tau, q} \norm{\{2^{j \lambda} d_j\}_{j\in\ent}}{\ell^q},
\end{align*}
where in the last equality we have used that $\tau < 0$. If $1 < q < \infty$ we use H\"older's inequality with $q$ and $q'$ to write
\begin{align*}
 \norm{\left\{\sum_{k=k_0}^\infty 2^{\tau k} 2^{\lambda(j+k)}d_{j+k}\right\}_{j\in\ent}}{\ell^q} 
& \le \left[\sum_{j \in \ent}   \left(\sum_{k=k_0}^\infty 2^{\tau k q/2} 2^{\lambda q(j+k)}d_{j+k}^q \right)     \left( \sum_{k=k_0}^\infty 2^{\tau kq'/2} \right)^{q/q'}   \right]^\frac{1}{q}\\
&    =C_{k_0, \tau, q} \norm{\{2^{j \lambda} d_j\}_{j\in\ent}}{\ell^q}.
\end{align*}

The case $q = \infty$ is straightforward.
\end{proof}

We now show the proof of Theorem \ref{thm:Nikolskij:Lorrentz:weighted}.

\begin{proof} We first establish the theorem for finite families of functions and then extend this result to families that are not necessarily finite.  We will do this in the homogeneous settings, with the proof in the inhomogeneous settings being similar. Suppose $\{u_j\}_{j\in\ent}$ is such that $u_j=0$ for all $j$ except those belonging to some finite subset of $\ent;$  this assumption allows us to avoid convergence issues since all the sums  considered will be finite.

For Part~\eqref{item:thh:Nikolskij:TL}, let $\A,$ $w,$ $p,$ $q$ and $s$ be as in the hypotheses. Fix $0<r<\min(1,p/\tau_w,q)$ such that  $s>n(1/r-1);$ note that the latter is possible since $s>\tau_{p,q}(w).$ 


Let $k_0 \in \ent$ be such that $2^{k_0 -1} < \A\leq 2^{k_0},$ then  
$$
\supp(\widehat{u_\ell}) \subset B(0, 2^\ell \A) \subset B(0, 2^{\ell +k_0}) \quad \forall \ell \in \ent.
$$
Define $u =\sum_{\ell \in \ent} u_\ell$ and let $\psi$ be as in the definition of $\tlw{(p,t)}{s}{q}{w}$ in Section~\ref{sec:spaces}. We have
\begin{equation}\label{psi*u}
\Delta^\psi_j u = \sum_{\ell \in \ent} \Delta^\psi_j u_\ell = \sum_{\ell = j - k_0}^\infty \Delta^\psi_j  u_\ell = \sum_{k = - k_0}^\infty \Delta^\psi_j u_{j+k}.
\end{equation}
We will use Lemma~\ref{coro:2:11} with $\phi(x) =  2^{jn} \psi(2^j x),$  $f = u_{j+k},$  $A= 2^{j  } >0,$ and $R= 2^{k + k_0}.$ (Notice that $\supp(\widehat{u_{j+k}}) \subset  B(0, 2^{j} 2^{k + k_0})$ and, since $k \geq -k_0$, we get $R \geq 1$.) Fixing  $\ex>n/r$ and applying Lemma~\ref{coro:2:11}, we get
\begin{align*}
|\Delta^\psi_j  u_{j+k}(x)| & \lesssim   2^{k_0n(\frac{1}{r} -1)} 2^{k n (\frac{1}{r} -1)}  2^{-jn} \norm{(1 + |2^j \cdot|)^\ex 2^{jn}\psi (2^j\cdot)}{L^\infty}  \M_r(u_{j+k})(x)\\
& \sim  2^{k n (\frac{1}{r} -1)} \left( \sup_{y \in \rn} (1 + |2^j y|)^\ex |\psi(2^j y)|\right) \M_r(u_{j+k})(x).
\end{align*}
 Hence,
\begin{align*}
2^{js}|\Delta^\psi_j  u_{j+k}(x)| & \lesssim 2^{k n (\frac{1}{r} -1-\frac{s}{n})} 2^{s(j+k)}\M_r(u_{j+k})(x),
\end{align*}
and then, recalling \eqref{psi*u}, 
\begin{align*}
2^{js} |\Delta^\psi_j  u(x)| & \lesssim  \sum_{k =  - k_0}^\infty 2^{k n (\frac{1}{r} -1-\frac{s}{n})} 2^{s(j+k)}\M_r(u_{j+k})(x).
\end{align*}
Since $1/r -1-s/n < 0$,   Lemma~\ref{eq:seriesineq}  yields
$$
\norm{\{ 2^{js} |\Delta^\psi_j  u|\}_{j\in\ent}}{L^{p,t}(w)(\ell^{q})} \lesssim \norm{\{2^{js} \M_ru_j\}_{j\in\ent}}{L^{p,t}(w)(\ell^{q})}
$$
with an implicit constant independent of $\{u_j\}_{j\in \ent}.$
Applying the weighted  Fefferman-Stein inequality  to the right-hand side of the last inequality leads to the desired estimate
\begin{equation*}
\norm{u}{\tlw{(p,t)}{s}{q}{w}} \lesssim \norm{\{2^{js}u_j\}_{j\in\ent}}{L^{p,t}(\ell^{q})}.
\end{equation*}


Assume that the theorem is true for finite families. Let $\{u_j\} \subset S'$ be as in the hypothesis. Then using the theorem for finite families we have that 
\begin{equation}
\norm{U_N - U_M}{\itl{(p,t)}{s}{q}} \lesssim \norm{ \{2^{js} u_j\}_{M+1\leq j \leq N}}{L^{p,t}(w)(\ell^q)}.
\end{equation}
Now we apply the following version of the dominated convergence theorem. 
\\
\bigskip
\textit{Suppose $f_n \rightarrow f$ in measure and $|f_n (x)| \leq |g(x)|$ for some $g\in L^{p,t}(w)$. Then $$\lim_{n\rightarrow\infty} \norm{f_n-f}{L^{p,t}(w)} = 0.$$}
\\
\bigskip
We need to check that the functions $U_{N,M} = \left(\sum_{j=M+1}^{N} (2^{js} u_j)^q \right)^\frac{1}{q}$, $U=0$, and $g = \left(\sum_{j=0}^\infty (2^{js}u_j)^q \right)^\frac{1}{q}$ satisfy the hypotheses of the theorem. Now we need to check that $f_{N,M}\rightarrow f$ in measure. Let $A_N = \{ x : \sum^\infty_{j=N} |2^{jsq}u_j|^q > \tau^q \}$. Because $\{u_j\} \subset L^{(p,t)}(w)(\ell^{q})$ we have that $|A_1| < \infty$. Then since $|A_{N+1}| \leq |A_{N}|$ it follows that $\lim_{N\rightarrow\infty} |A_N| = |\cap A_N|$. $\{u_j\} \subset L^{(p,t)}(w)(\ell^{q})$ so $u_j \rightarrow 0$ as $j\rightarrow\infty$. Therefore $f_N \rightarrow 0$ in measure. This implies that 
$$ \norm{U_N - U_M}{\itl{(p,t)}{s}{q}} \lesssim \norm{\{2^{js}u_j\}_{M+1\leq j \leq N}}{L^{p,t}(w)(\ell^q)} \rightarrow 0.$$
Then $\sum_{j=0}^\infty u_j$ converges in $\itl{(p,t)}{s}{q}$. 
\\
Now we consider the case where $q=\infty$. Then $\{2^{j(s-\epsilon)}\}_{j\geq0}$ belongs to $\ell^1(L^{p,t}(w))$ for any $\epsilon >0$. Then by the case for $q<\infty$ $\sum_{j=0}^\infty u_j$ converges in $B^s_{(p,t),1}$ and so it converges in $S'$. Then by using the case for finite families applied to $\{u_j\}_{0\leq j \leq N}$ (this case holds because the lemmas used only depend on the on the indexes considered and maximal function inequalities in Lorrentz spaces) we have that 
\[ \norm{U_N}{\itl{(p,t)}{s}{q}} \lesssim ||\{2^{js}u_j\}_{0\leq j \leq N}||_{F^s_{(p,t),\infty}} \leq ||\{2^{js}u_j\}||_{F^s_{(p,t),\infty}}. \] Then using the Fatou property finishes the proof.
\end{proof}

With these four properties theorems analogous those earlier in this chapter hold in the setting of weighted Lorrentz-based Triebel-Lizorkin and Besov spaces and as an example the analogue to Theorem~\ref{thm:ICM:TL:B} in the context of the spaces $\itlw{(p,t)}{s}{q}{w}$ is below.


\begin{theorem}\label{thm:ICM:Lorentz}  For $m \in \re,$ let $\sigma(\xi,\eta),$ $\xi,\eta\in\rn,$ be an inhomogeneous Coifman-Meyer multiplier of order $m.$ If $w\in A_\infty,$ $0 < p, p_1, p_2 < \infty$ and $0 < t, t_1, t_2\le \infty$  are  such that $\hcline$ and $1/t=1/t_1+1/t_2,$  $0 < q \le \infty$  and  $s > \tau_{p,t,q}(w),$  it holds that
\begin{equation*}
\norm{T_\sigma(f,g)}{\itlw{(p,t)}{s}{q}{w}} \lesssim \norm{f}{\itlw{(p_1,t_1)}{s+m}{q}{w}} \norm{g}{h^{p_2,t_2}(w)} +  \norm{f}{h^{p_1,t_1}(w)}   \norm{g}{\itlw{(p_2,t_2)}{s+m}{q}{w}} \quad \forall f, g \in \sw.
\end{equation*}
Different pairs of $p_1,p_2$ and $t_1, t_2$ can be used on the right-hand side of the inequality above. 
Moreover, if $w\in A_\infty,$ $0<p<\infty,$ $0<t,q\le \infty$   and $s > \tau_{p,t,q}(w),$  it holds that
\begin{equation*}
\norm{T_\sigma(f,g)}{\itlw{(p,t)}{s}{q}{w}} \lesssim \norm{f}{\itlw{(p,t)}{s+m}{q}{w}} \norm{g}{L^\infty} +  \norm{f}{L^\infty}   \norm{g}{\itlw{(p,t)}{s+m}{q}{w}} \quad \forall f, g \in \sw.
\end{equation*}
\end{theorem}
 
By the Fefferman--Stein inequality \eqref{eq:FTLorentz} the lifting property $ \norm{f}{\itl{(p,t)}{s}{q}}\simeq\norm{J^s f}{\itl{(p,t)}{0}{q}}$ holds true for $s \in \re,$ $0<p<\infty$ and  $0<t, q\le \infty.$ 
Then, under the assumptions of Theorem~\ref{thm:ICM:Lorentz}  we obtain, in particular,
\begin{equation*}
\norm{J^s (fg)}{\itl{(p,t)}{0}{q}(w)} \lesssim \norm{J^s f}{\itl{(p_1,t_1)}{0}{q}(w)} \norm{g}{h^{p_2,t_2}(w)} +  \norm{f}{h^{p_1,t_1}(w)}   \norm{J^sg}{\itl{(p_2,t_2)}{0}{q}(w)};
\end{equation*}
\begin{equation*}
\norm{J^s(fg)}{\itlw{(p,t)}{0}{q}{w}} \lesssim \norm{J^sf}{\itlw{(p,t)}{0}{q}{w}} \norm{g}{L^\infty} +  \norm{f}{L^\infty}   \norm{J^s g}{\itlw{(p,t)}{0}{q}{w}}.
\end{equation*}
These last two estimates supplement the results in \cite[Theorem 6.1]{MR3513582}, where related Leibniz-type rules in Lorentz spaces were obtained.
 \subsection{Morrey spaces}
 
 Given $0<p\le t<\infty$ and $w\in A_\infty,$  we denote by $\mow{p}{t}{w}$ the weighted Morrey space consisting of functions $f\in L^p_{\text{loc}}(\rn)$  such that
\[
\|f\|_{\mow{p}{t}{w}}=\sup_{B\subset \rn}w(B)^{\frac{1}{t}-\frac{1}{p}}\left( \int_B\abs{f(x)}^pw(x)\dx\right)^\frac{1}{p}<\infty,
\]
where the supremum is taken over all Euclidean balls $B$ contained in $\rn.$ We note that $\mow{p}{p}{w}=\lebw{p}{w}.$ For more details on Morrey spaces see the work Rosenthal--Schmeisser~\cite{MR3538648} and the references contained therein. The corresponding weighted inhomogeneous Triebel--Lizorkin spaces and inhomogeneous Besov spaces  are denoted by $\itlw{[p,t]}{s}{q}{w}$ and   $\ibesw{[p,t]}{s}{q}{w},$ respectively. These Morrey-based Triebel--Lizorkin and Besov  spaces are independent of the choice of $\varphi$ and $\psi$ given in  Section~\ref{sec:spaces} and are quasi-Banach spaces that contain $\sw$ (see the works Kozono--Yamazaki~\cite{MR1274547}, Mazzucato~\cite{MR1946395}, Izuki et al.~\cite{MR2792058} and the references they contain). The corresponding local Hardy spaces are denoted by $h^t_p(w).$ 

We now show the analogues to properties \ref{item:first}--\ref{item:last} for weighted Morrey spaces.

%For \ref{item:first} we use that for $0<s<\infty$ $\| |f|^s \|_{\mow{p}{t}{w}} = \|f\|^s_{\mow{sp}{st}{w}}$ and the corresponding property for weighted Lebesgue spaces to get 
\[\|f+g\|^r_{\mow{p}{t}{w}} \lesssim \|f\|^r_{\mow{p}{t}{w}} + \|g\|^r_{\mow{p}{t}{w}} \]
for $r = min(1,p)$. It follows that for $r :=min(1,p,q)$ 
\[ \|f+g\|^r_{\itlw{[p,t]}{s}{q}{w}} \lesssim \|f\|^r_{\itlw{[p,t]}{s}{q}{w}} + \|g\|^r_{\itlw{[p,t]}{s}{q}{w}} \]
with similar inequalities for inhomogeneous weighted Morry-based Besov spaces and homogeneous weighted Morry-based Triebel-Lizorkin and Besov spaces.

A version of property \ref{item:second} follows from H\"older's inequality for weighted Lebesgue spaces. For $0<p\le t<\infty,$ $0<p_1\le t_1<\infty$ and $0<p_2\le t_2<\infty$ are such that $\hcline$ and $1/t=1/t_1+1/t_2,$ then  
\begin{equation*}
\norm{fg}{\mow{p}{t}{w}}\le \norm{f}{\mow{p_1}{t_1}{w}} \norm{g}{\mow{p_2}{t_2}{w}};
\end{equation*}
also, if $0<p\le t<\infty,$ $0<p_1,p_2<\infty$ are such that $\hcline$ and $w=w_1^{p/p_1}w_2^{p/p_2}$ for weights $w_1$ and $w_2,$ then 
\begin{equation*}
\norm{fg}{\mow{p}{t}{w}}\le \norm{f}{\mow{p_1}{\frac{p_1t}{p}}{w_1}} \norm{g}{\mow{p_2}{\frac{p_2t}{p}}{w_2}}.
\end{equation*}

The analogue to property \ref{item:third} is the following Fefferman-Stein inequality:
Let $0<p\leq t<\infty$, $0<q\leq\infty$ and $0<r<min(p/\tau_w,q)$, then
\begin{equation}\label{eq:FTMorrey}
\norm{\left(\sum_{j\in\ent} \abs{\M_r(f_j)}^q\right)^{\frac{1}{q}}}{\mow{p}{t}{w}}\lesssim \norm{\left(\sum_{j\in\na_0} \abs{f_j}^q\right)^{\frac{1}{q}}}{\mow{p}{t}{w}}\quad \forall \{f_j\}_{j\in \na_0}\in \mow{p}{t}{w}(\ell^q).
\end{equation}
If $0<p\leq t <\infty$ and $0<r<p/\tau_w$, then 
\begin{equation*}
\norm{\M_r(f)}{\mow{p}{t}{w}}\lesssim \norm{f}{\mow{p}{t}{w}}\quad \forall f\in \mow{p}{t}{w}. 
\end{equation*}
The case for $r=1,$ $1<p\le t<\infty$ and $1<q\le \infty$ are shown using extrapolation and the Fefferman-Stein inequality in weighted Lebesgue spaces. For the extrapolation theorem see \cite[Theorem 5.3]{MR3538648}. The remaining cases are shown using that for $0<s<\infty$ $\| |f|^s \|_{\mow{p}{t}{w}} = \|f\|^s_{\mow{sp}{st}{w}}$ and the previous case.

The Nikol'skij representation for weighted Morrey-based Triebel-Lizorkin and Besov spaces is as follows.
\begin{theorem}\label{thm:Nikolskij:Morrey:weighted} For $\A> 0,$ let $\{u_j\}_{j \in \naz} \subset \mathcal{S}'(\rn)$ be a sequence of tempered distributions such that
\begin{equation*}
\supp(\widehat{u_j}) \subset B(0, \A\, 2^j ) \quad \forall j \in \naz.
\end{equation*}
If $w\in A_\infty,$ then the following holds:  
\begin{enumerate}
\item[(i)]\label{item:thh:Nikolskij:Morrey:TL} Let $0 < p \leq t < \infty$, $0 < q \leq \infty$ and $s > \left(\frac{1}{\min(p/\tau_w,t,q,1)} - 1 \right)$. If $\norm{\{2^{js} u_j\}_{j\in\ent}}{\mow{p}{t}{w}(\ell^{q})} < \infty$, then the series $\sum_{j \in \naz} u_j$ converges in $\mathcal{S}'(\rn)$ and 
\begin{equation*}
\norm{\sum_{j \in \ent} u_j}{\itlw{[p,t]}{s}{q}{w}} \lesssim  \norm{\{2^{js} u_j\}_{j\in\naz}}{\mow{p}{t}{w}(\ell^{q})},
\end{equation*}
where the implicit constant depends only on $n,$ $\A,$ $s,$ $p$ and  $q.$
\item[(ii)]\label{item:thh:Nikolskij:Morrey:B} Let $0 < p, q \leq \infty$ and $s > \tau_{p,t}(w)$. If $\norm{\{2^{js} u_j\}_{j\in\naz}}{\ell^{q}(\mow{p}{t}{w})} < \infty$, then the series $\sum_{j \in \naz} u_j$ converges in  $\ibesw{[p,t]}{s}{q}{w}$ (in $\mathcal{S}'(\rn)$ if $q=\infty$) and 
\begin{equation*}
\norm{\sum_{j \in \ent} u_j}{\ibesw{[p,t]}{s}{q}{w}} \lesssim  \norm{\{2^{js} u_j\}_{j\in\naz}}{\ell^{q}(\mow{p}{t}{w})},
\end{equation*}
where the implicit constant depends only on $n,$ $\A,$ $s,$ $p$ and $q.$
\end{enumerate}
\end{theorem}

The proof of Theorem (\ref{thm:Nikolskij:weighted}) uses Lemma \ref{coro:2:11}, Lemma \ref{eq:seriesineq}, and a modified version of Lemma (\ref{coro:2:12(1)}).

\begin{lemma}\label{morrey:lemma}
Let $0<p\leq t < \infty$, $A>0$, $R\geq 1$ and $d>b>\frac{n}{min(p/\tau_{w},1)}$. If $\phi \in \sw$ and $supp(\widehat{f}) \subset \{\xi: |\xi| \leq AR \}$. Then 
\begin{equation}
\norm{\phi \ast f}{\mow{p}{t}{w}} \lesssim R^{b-n}A^{-n}\norm{(1+|A\cdot|^d) \phi}{L^\infty} \norm{f}{\mow{p}{t}{w}}
\end{equation}
with the implicit constant independent of $R, A, \phi, \text{ and } f.$
\end{lemma}

The proof follows along the same lines as Lemma \ref{coro:2:11}.

\begin{proof}
Proof of part (ii): Using Lemma 4.3 we have that $$\norm{\Delta^\psi_j u_{j+k}}{\mow{p}{t}{w}} \lesssim 2^{k(b-n)} \norm{u_{j+k}}{\mow{p}{t}{w}}$$ where we have used boundedness of the Hardy-Littlewood maximal function on Morrey spaces when $0<r<p/\tau_w$. Then by setting $r^\ast = min(1,p,q)$ and using property (1) for Morrey spaces 
$$ 2^{jsr^\ast}\norm{\Delta^\psi_j u}{\mow{p}{t}{w}}^{r^\ast} \lesssim 2^{js r^*}\sum_{k = - k_0}^\infty \norm{\Delta^\psi_j  u_{j+k}}{\mow{p}{t}{w}}^{r^*}
 =  \sum_{k = - k_0}^\infty 2^{k(b-n -s) r^*} 2^{s r^* (j+k)}   \norm{u_{j+k}}{\mow{p}{t}{w}}^{r^*}.  $$ 
Now by applying Lemma A.3  we have 
\begin{align*}
\norm{u}{\besw{[p,t]}{s}{q}{w}}  \lesssim  \norm{\left\{ \sum_{k = - k_0}^\infty 2^{k(b-n -s) r^*}  2^{s r^* (j+k)}   \norm{u_{j+k}}{\mow{p}{t}{w}}^{p^*}\right\}_{j\in\naz}}{\ell^{q/r*}}^\frac{1}{r^*}\lesssim  \norm{\{2^{js} u_j\}_{j\in\naz}}{\ell^{q}(\mow{p}{t}{w})} 
\end{align*}
and part (ii) is shown for the finite family case. For a family that is not necessarily finite we apply the finite family case to $U_N - U_M := \sum_{j=M+1}^N u_j$. For finite q this gives us 
\begin{align*}
\norm{U_N - U_M}{\besw{[p,t]}{s}{q}{w}} \lesssim \norm{\{2^{js} u_j\}_{M+1\leq j \leq N}}{\ell^{q}(\mow{p}{t}{w})} \leq \norm{\{2^{js} u_j\}_{j\in\naz}}{\ell^{q}(\mow{p}{t}{w})} < \infty
\end{align*}
and by the dominated convergence theorem the left side converges to 0 as $M,N \rightarrow \infty$. So $U_N$ converges in $\ibesw{[p,t]}{s}{q}{w}.$ If $q = \infty$ then $\{2^{j(s-\epsilon)}\} \in \ell^1 (\mow{p}{t}{w})$. By the case for finite q we have that $\sum_{j=0}^N u_j$ converges in $\ibesw{[p,t]}{s-\epsilon}{q}{w}$. Therefore it converges in $\mathcal{S}'$. 
\\
Proof of part (i): First we will assume that $u_j = 0$ for all but finitely many $j$. From Lemma 4.2 we have that 
$$ |\Delta^\psi_j u_{j+k}(x)| \lesssim 2^{kn(\frac{1}{r} - 1 - \frac{s}{n})} 2^{j(s+k)}\mathcal{M}_r(u_{j+k})(x) $$
where $\Delta^\psi_j u = \sum_{\ell \in \naz} \Delta^{\psi}_j u_\ell.$ By Lemma 3.4 and the Fefferman-Stein inequality for Morrey spaces we get 
\[ \norm{ \{ 2^{js} |\Delta^\psi_j u| \}_{j\in\naz}}{M^p_t (w)(\ell^q)} \lesssim \norm{ \{ 2^{js} \mathcal{M}_r u_j \}_{j\in\naz}}{M^p_t (w)(\ell^q)} \lesssim \norm{ \{ 2^{js} u_j \}_{j\in\naz}}{M^p_t (w)(\ell^q)}\] 
for $0<r<min(\frac{p}{\tau_w},q)$.
Now we prove the theorem for families with are not necessarily finite. Let $U_N := \sum_{j=0}^N u_j$. Then by the finite family case we have that 
$$ \norm{u_N}{\itlw{[p,t]}{s}{q}{w}} \lesssim \norm{{\{2^{js}}u_j\}_{0\leq j\leq N} }{\mow{t}{p}{w}(\ell^q)} \lesssim \norm{{\{2^{js}}u_j\}}{\mow{t}{p}{w}(\ell^q)} <\infty.$$ Assume that $1<q<\infty$. Because $\norm{{\{2^{js}}u_j\}}{\mow{t}{p}{w}(\ell^q)} <\infty$ we have that $\sup_j\norm{{\{2^{js}}u_j\}}{\mow{t}{p}{w}} <\infty$ which falls in the $q=\infty$ case for part (ii). Then $\sum_{j=0}^\infty u_j$ converges in $S'$ and from the Fatou property 
\[\liminf_{N\rightarrow\infty} \norm{U_N}{\itlw{[p,t]}{s}{q}{w}} \leq \norm{U}{\itlw{[p,t]}{s}{q}{w}}v\lesssim \norm{{\{2^{js}}u_j\}}{\mow{t}{p}{w}(\ell^q)}. \]

If $q=\infty$ then $\{2^{j(s-\epsilon)}u_j\}_{j\in\naz}$ is in $\ell^1(M^t_p)(w))$ for any $\epsilon>0$. Then the case for finite $q$ shows that $\{2^{j(s-\epsilon)}u_j\}_{j\in\naz}$ converges in ${\ibesw{[p,t]}{s-\epsilon}{1}{w}}$ and we have convergence in $\mathcal{S}'$. Then using the finite family case we have 


\[ \norm{U_N}{F^s_{[p,t],\infty}(w)} \lesssim \norm{ \{ 2^{js}u_j \}_{0\leq j\leq N }}{\ell^1(M^t_p)(w)} \leq \norm{ \{ 2^{js}u_j \} }{\ell^1(M^t_p)(w)} < \infty \] 

 Then after using the Fatou property of  $\itlw{[p,t]}{s}{q}{w}$ we are finished.
\end{proof}
\subsection{Variable Lebesgue spaces}
 
 Let  $\P_0$ be the collection of  measurable functions $\pp : \rn \rightarrow (0,\infty)$  such that
\begin{equation*}
p_- := \essinf_{x\in \rn} p(x)>0 \quad\text{ and } \quad p_+ := \esssup_{x\in \rn} p(x)<\infty.
\end{equation*}
For $\pp\in\P_0,$   the variable-exponent Lebesgue space $\Lp$
consists of all measurable functions $f$ such that 
\begin{equation*}
\norm{f}{L^{p(\cdot)}}:=\inf\left\{\lambda>0: \int_{\rn} \abs{\frac{f(x)}{\lambda}}^{p(x)}\,dx\le 1\right\}<\infty;
\end{equation*}
such quasi-norm turns $\Lp$ into  a quasi-Banach  space (Banach space if $p_-\ge1$). 
We note that if $\pp=p$ is constant then  $\Lp\simeq L^p$ with equality of norms and that
\begin{equation}\label{eq:power}
\norm{\abs{f}^t}{\Lp}=\norm{f}{L^{t\pp}}^t\quad \forall\, t>0.
\end{equation} 

Let $\B$ be the family of all $\pp \in \P_0$ such that $\M,$  the Hardy--Littlewood maximal operator, is bounded from $\Lp$ to $\Lp.$ Such exponents satisfy $p_- > 1$ and the following log-H\"older continuity properties
\begin{itemize}
\item there exists a constant $C_0$ such that for all $x,y\in \rn$, $|x-y|<1/2$
\[ |p(x)-p(y)| \leq \frac{C_0}{-log(|x-y|)}, \]
\item there exist constants $C_\infty$ and $p_\infty$ such that for all $x\in\rn$ 
\[|p(x) - p_\infty| \leq \frac{C_\infty}{log(e + |x|)}.	 \]
\end{itemize}
Furthermore if 
$\tau_0 >0$ 
is such that 
$p(\cdot)/\tau_0 \in \B$ 
then $\pp/\tau \in \B$ for $0<\tau<\tau_0$. 
Indeed, by Jensen's inequality it holds that 
$\M(f)^{\tau_0/\tau}(x) 
\leq \M(|f|^{\tau_0/\tau})(x)$ 
so by \ref{eq:power} we have that 
\begin{align*}
\norm{\M f}{L^{\frac{p(\cdot)}{\tau}}} &\leq \norm{\M(|f|^{\tau_0/\tau})^{\tau/\tau_0}}{L^{\frac{p(\cdot)}{\tau}}} \\
& = \norm{\M(|f|^{\tau_0/\tau})}{L^{\frac{\pp}{\tau_0} }}^{\tau/\tau_0}\\
& \leq \norm{|f|^{\frac{\tau_0}{\tau}}}{L^{\frac{p(\cdot)}{\tau}}}^{\frac{\tau}{\tau_0}} \\
& = \norm{f}{L^{\frac{p(\cdot)}{\tau}}}.
\end{align*}
We then define
\begin{equation*}
\tau_{\pp}=\sup\{\tau>0:\fr{\pp}{\tau}\in \B\},\quad \pp\in \P_0^*,
\end{equation*}
where $\P_0^*$ denotes the class of variable exponents in $\P_0$ such that $\pp/\tau_0\in\B$ for some $\tau_0>0.$ Note that $\tau_{\pp}\le p_-.$


Given $s\in\re,$ $0<q\le \infty$ and $\pp\in\P_0,$ the corresponding inhomogeneous Triebel-Lizorkin and Besov spaces are denoted by $\itl{\pp}{s}{q}$  and  $\ibes{\pp}{s}{q},$
respectively. If $\pp\in\P_0^*,$ these spaces are independent of the functions $\psi$ and $\varphi$ given in  Section~\ref{TL_B_section} (see  Xu~\cite{MR2431378}), contain $\sw$ and are quasi-Banach spaces. If $\pp\in\B$ and $s>0,$ $\itl{\pp}{s}{2}$ coincides with the variable-exponent  Sobolev space $W^{s,\pp}$ (see Gurka et al.~\cite{MR2339558} and Xu~\cite{MR2449626}). More general versions of variable-exponent  Triebel--Lizorkin and Besov spaces, where $s$ and  $q$ are also allowed to be functions, were introduced in  Diening at al.~\cite{MR2498558} and Almeida--H\"ast\"o~\cite{MR2566313}, respectively. 
The local Hardy space with variable exponent $\pp\in \P_0,$  denoted $h^{\pp},$  is defined analogously to $h^p(w)$ with the quasi-norm in $\lebw{p}{w}$ replaced by the quasi-norm in $\Lp.$ 

We now consider the analogues of properties \ref{item:first}-\ref{item:last} in the setting of variable Lebesgue based spaces. 

For \ref{item:first} we apply \ref{eq:power} to get for $r = \text{min}(p_-,q,1)$
\[\norm{f+g}{\itl{\pp}{s}{q}}^r \leq \norm{f}{\itl{\pp}{s}{q}}^r + \norm{g}{\itl{\pp}{s}{q}}^r, \]
\[\norm{f+g}{\ibes{\pp}{s}{q}}^r \leq \norm{f}{\ibes{\pp}{s}{q}}^r + \norm{g}{\ibes{\pp}{s}{q}}^r. \]

To prove \ref{item:second} we use \cite[Corollary 2.28]{MR3026953} and \ref{eq:power} to get the following version of H\"older's inequality: If $\ppo,\, \ppt,\, \pp \in \P_0$ are such that
$ {1}/{\pp} = {1}/{\ppo} + {1}/{\ppt}$ then
\begin{equation*}
 \|fg\|_{L^\pp} \lesssim \|f\|_{L^\ppo}\|g\|_{L^\ppt}\quad \forall f\in L^\ppo, g\in L^\ppt.
 \end{equation*}
The case when $\pp \in \P_0$ has $p_- \geq 1$ is shown in \cite[Corollary 2.28]{MR3026953}. If $0<p_-<1$ then we use \ref{eq:power} to get 
\begin{align*}
\norm{fg}{L^\pp} = \norm{|fg|^{p_-}}{L^{\frac{\pp}{p_-}}}^{\frac{1}{p_-}}
\end{align*}
and then use the first case since $\frac{\pp}{p_-}>1.$

A Fefferman-Stein inequality in variable exponent Lebesgue spaces follows from the discussion in \cite[Section 5.6.8]{MR3026953} and \eqref{eq:power}. For property \ref{item:third} we have the following:
If $\pp\in\P_0^*,$  $0<q\le \infty$  and $0<r<\min(\tau_{\pp},q)$ then
\begin{equation*}
\norm{\left(\sum_{j\in\ent} \abs{\M_r(f_j)}^q\right)^{\frac{1}{q}}}{\Lp}\lesssim \norm{\left(\sum_{j\in\na_0} \abs{f_j}^q\right)^{\frac{1}{q}}}{\Lp}\quad \forall \{f_j\}_{j\in \na_0}\in \Lp(\ell^q);
\end{equation*}
in particular, if $0<r<\tau_{\pp}$ it holds that 
\begin{equation*}
\norm{\M_r(f)}{\Lp}\lesssim \norm{f}{\Lp}\quad \forall f\in \Lp.
\end{equation*}

Property \ref{item:last}, the Nikol\'skij representation for $\itl{\pp}{s}{q}$ and $\ibes{\pp}{s}{q}$, for variable exponent Lebesgue spaces is stated below.
\begin{theorem}\label{thm:Nikolskij:variable} For $\A> 0,$ let $\{u_j\}_{j \in \ent} \subset \mathcal{S}'(\rn)$ be a sequence of tempered distributions such that $\supp(\widehat{u_j}) \subset B(0, \A\, 2^j ) $ for all $j \in \ent.$
Let $ \pp\in \P_0^*$, $0 < q \le \infty$ and $s > n(1/\min(\tau_{\pp},q,1)-1)$. If $\norm{\{2^{js} u_j\}_{j\in\ent}}{\Lp(\ell^{q})} < \infty$, then the series $\sum_{j \in \ent} u_j$ converges in $\itl{\pp}{s}{q}$ (in $\swp$ if $q=\infty$) and 
\begin{equation*}
\norm{\sum_{j \in \na_0} u_j}{\itl{\pp}{s}{q}} \lesssim  \norm{\{2^{js} u_j\}_{j\in\na_0}}{L^\pp(\ell^{q})},
\end{equation*}
where the implicit constant depends only on $n,$ $\A,$ $s,$ $\pp$ and  $q.$  An analogous statement holds true for $\ibes{\pp}{s}{q}$ with $s> n(1/\min(\tau_{\pp},1)-1)$.
\end{theorem}

\begin{lemma}\label{lemma:variableLebesgue}
Let $p(\cdot) \in \mathcal{P}_0$, $A>0$, $R\geq 1$ and $d>b>\frac{n}{min(\tau_{p(\cdot)},1)}$. If $\phi \in \sw$ and $supp(\widehat{f}) \subset \{\xi: |\xi| \leq AR \}$. Then 
\begin{equation}
\norm{\phi \ast f}{\Lp} \lesssim R^{b-n}A^{-n}\norm{(1+|A\cdot|^d)\phi}{L^\infty}\norm{f}{\Lp},
\end{equation}
with the implicit constant independent of $R$, $A$, $\phi$, and $f$.
\end{lemma}



\begin{proof}
We first prove the theorem for finite families. Assume that $\{u_j\}_{j \in \ent} \subset \mathcal{S}'(\rn)$ is such that $u_j \equiv 0$ for all but finitely many $j$. Let $D$, $\pp$, $q$, and $s$ be as in the hypotheses. Fix $0<r<min(1,p_-,q)$ such that $s>n(1/r-1)$; note that this is possible since $s>n(1/\min(\tau_{\pp},q,1)-1)$. Let $k_0 \in \ent$ be such that $2^{k_0 -1} < \A\leq 2^{k_0},$ then  
\[
\supp(\widehat{u_\ell}) \subset B(0, 2^\ell \A) \subset B(0, 2^{\ell +k_0}) \quad \forall \ell \in \ent.
\]
Define $u =\sum_{\ell \in \ent} u_\ell$ and let $\psi$ be as in the definition of $\tlw{\pp}{s}{q}{w}$. We have
\begin{equation}\label{psi*u}
\Delta^\psi_j u = \sum_{\ell \in \ent} \Delta^\psi_j u_\ell = \sum_{\ell = j - k_0}^\infty \Delta^\psi_j  u_\ell = \sum_{k = - k_0}^\infty \Delta^\psi_j u_{j+k}.
\end{equation}
We will use Lemma~\ref{coro:2:11} with $\phi(x) =  2^{jn} \psi(2^j x),$  $f = u_{j+k},$  $A= 2^{j}>0,$ and $R= 2^{k + k_0}.$ (Notice that $\supp(\widehat{u_{j+k}}) \subset  B(0, 2^{j} 2^{k + k_0})$ and, since $k \geq -k_0$, we get $R \geq 1$.) Fixing  $\ex>n/r$ and applying Lemma~\ref{coro:2:11}, we get
\begin{align*}
|\Delta^\psi_j  u_{j+k}(x)| & \lesssim   2^{k_0n(\frac{1}{r} -1)} 2^{k n (\frac{1}{r} -1)}  2^{-jn} \norm{(1 + |2^j \cdot|)^\ex 2^{jn}\psi (2^j\cdot)}{L^\infty}  \M_r(u_{j+k})(x)\\
& \sim  2^{k n (\frac{1}{r} -1)} \left( \sup_{y \in \rn} (1 + |2^j y|)^\ex |\psi(2^j y)|\right) \M_r(u_{j+k})(x).
\end{align*}
 Hence,
\begin{align*}
2^{js}|\Delta^\psi_j  u_{j+k}(x)| & \lesssim 2^{k n (\frac{1}{r} -1-\frac{s}{n})} 2^{s(j+k)}\M_r(u_{j+k})(x),
\end{align*}
and then, recalling \eqref{psi*u}, 
\begin{align*}
2^{js} |\Delta^\psi_j  u(x)| & \lesssim  \sum_{k =  - k_0}^\infty 2^{k n (\frac{1}{r} -1-\frac{s}{n})} 2^{s(j+k)}\M_r(u_{j+k})(x).
\end{align*}
Since $1/r -1-s/n < 0$,   Lemma~\ref{eq:seriesineq}  yields
$$
\norm{\{ 2^{js} |\Delta^\psi_j  u|\}_{j\in\ent}}{L^p(w)(\ell^{q})} \lesssim \norm{\{2^{js} \M_ru_j\}_{j\in\ent}}{L^p(w)(\ell^{q})}
$$
with an implicit constant independent of $\{u_j\}_{j\in \ent}.$
Applying the weighted  Fefferman-Stein inequality  to the right-hand side of the last inequality leads to the desired estimate
\begin{equation*}
\norm{u}{\tlw{p}{s}{q}{w}} \lesssim \norm{\{2^{js}u_j\}_{j\in\ent}}{L^p(\ell^{q})}.
\end{equation*}

For the space $\besw{\pp}{s}{q}$, let $\A,$ $w,$ $p,$ $q$ and $s$ be as in the hypotheses and $k_0$ be as above. 
Consider $\Delta^\psi_j  u_{j+k}$  in  \eqref{psi*u} and  apply  Lemma~\ref{coro:2:12(1)}  with $\phi(x)=2^{jn}\psi(2^{-j} x)$, $f= u_{j+k},$ $A=2^{j},$ $R=2^{k +  k_0}$,  $\ex>b$ and $  n /\min(1,p/\tau_w)<b<n+s;$ note that such $b$ exists since $s>\tau_p(w).$ We get
\begin{align*}
 \norm{\Delta^\psi_j  u_{j+k}}{\Lp}  \lesssim 2^{(k + k_0)(b-n)} 2^{- j  n} \norm{(1 + |2^j \cdot|)^\ex 2^{jn}\psi(2^{-j}\cdot)}{L^\infty} \norm{u_{j+k}}{L^p(w)}\sim 2^{k(b-n)}   \norm{u_{j+k}}{\Lp},
\end{align*}
and setting $p^*:=\min(p_-,1)$ we obtain
\begin{align*}
2^{js p^*}\norm{\Delta^\psi_j  u}{\Lp}^{p^*} \lesssim 2^{js p^*}\sum_{k = - k_0}^\infty \norm{\Delta^\psi_j  u_{j+k}}{\Lp}^{p^*}
 =  \sum_{k = - k_0}^\infty 2^{k(b-n -s) p^*} 2^{s p^* (j+k)}   \norm{u_{j+k}}{\Lp}^{p^*}.
\end{align*}
Hence, applying Lemma~\ref{eq:seriesineq}, it follows that
\[
\norm{u}{\bes{\pp}{s}{q}}  
\lesssim  
\norm{\left\{ \sum_{k = - k_0}^\infty 2^{k(b-n -s) p^*}  2^{s p^* (j+k)}   \norm{u_{j+k}}{\Lp}^{p^*}\right\}_{j\in\ent}}{{\ell^{q/p*}}^\frac{1}{p^*}}
\lesssim  
\norm{\{2^{js} u_j\}_{j\in\ent}}{\ell^{q}(\Lp)}, 
\]
as desired.
Now for families that are not necessarily finite let $U_N := \sum_{j=0}^N u_j$. First we assume that $0<q<\infty$. Then for $M\leq N$ we have
\begin{equation}
\norm{U_N -U_M}{\tl{\pp}{s}{q}} \lesssim \norm{\{2^{js} u_j\}_{M+1\leq j \leq N}}{\Lp.(\ell^q)}.
\end{equation}
Now we use the following dominated convergence theorem in variable exponent Lebesgue spaces: 
\textit{Suppose $f_n \rightarrow f$ pointwise a.e. and $|f_n (x)| \leq |g(x)|$ for some $g\in \Lp.$. Then $f_n \rightarrow f$ in $\Lp.$}
\\
\medskip
Applying this theorem with $f_{N,M} = \sum_{j=M+1}^N 2^{js}U_j$, $f = 0$, and $g = \sum_{j=0}^\infty u_j$ to get 
\begin{equation}
\norm{U_N -U_M}{\itl{\pp}{s}{q}} \lesssim \norm{\{2^{js} u_j\}_{M+1\leq j \leq N}}{\Lp.(\ell^q)} \rightarrow 0 \text{ as } M,N \rightarrow \infty.
\end{equation}
Because $\tl{\pp}{s}{q}$ is a quasi-Banach space it is complete so $U_N$ converges in $\itl{\pp}{s}{q}$ and 
\begin{equation}
\norm{\sum_{j=0}^\infty}{\itl{\pp}{s}{q}} \lesssim \norm{\{2^{js} u_j\}}{\Lp(\ell^q)}.
\end{equation}

If $q=\infty,$  we use that $\{2^{(s-\varepsilon)j}u_j\}_{j\ge 0}$ and $\{2^{(s+\varepsilon)j}u_j\}_{j< 0}$ belong to $\ell^1(\Lp)$ for any $\varepsilon>0$ and apply Theorem~\ref{thm:Nikolskij:weighted} under the case of finite $q$ to conclude that $\sum_{j=0}^Nu_j$ and $\sum_{j=-N}^{-1}u_j$ converge in $\bes{\pp}{s-\varepsilon}{1}$ and $\bes{\pp}{s+\varepsilon}{1},$ respectively (choosing $\varepsilon>0$ so that $s-\varepsilon>n(1/\min(\tau_{\pp},1)-1)$). Therefore, $U_N$ convergence in $\mathcal{S}'(\rn).$ Moreover, by  Theorem~\ref{thm:Nikolskij:weighted} applied to the finite sequence $\{u_j\}_{-N\le j\le N},$ we have that $U_N\in \tl{\pp}{s}{\infty}$ and 
\[
\norm{U_N}{\tl{\pp}{s}{\infty}}\lesssim \norm{\{2^{js} u_j\}_{-N\le j\le N}}{\Lp(\ell^\infty)}\le \norm{\{2^{js} u_j\}_{j\in \ent}}{\Lp(\ell^\infty)},
\]
with the implicit constant independent of $N$ and $\{u_j\}_{j\in\ent}.$
Since $\tl{\pp}{s}{\infty}$ has the Fatou property, we conclude that $\lim_{N\to \infty} U_N=\sum_{j\in\ent}u_j$ belongs to $\tl{\pp}{s}{\infty}$ and 
\[
\norm{\sum_{j\in\ent}u_j}{\tl{\pp}{s}{\infty}}\lesssim  \norm{\{2^{js} u_j\}_{j\in \ent}}{\Lp(\ell^\infty)}.
\]
\end{proof}

As a model result the we state the Leibniz type rule for variable exponent Triebel-Lizorkin spaces.

\begin{theorem} \label{thm:ICM:variable} For $m \in \re,$ let $\sigma(\xi,\eta),$ $\xi,\eta\in\rn,$ be an inhomogeneous Coifman-Meyer multiplier of order $m.$ If  $\pp,\ppo,\ppt\in \P_0^*$  are such that $ {1}/{\pp} = {1}/{\ppo} + {1}/{\ppt},$   $0 < q \le \infty$ and  
$s > n(1/\min(\tau_{\pp},q,1)-1),$  it holds that
\begin{equation*}
\norm{T_\sigma(f,g)}{\itl{\pp}{s}{q}} \lesssim \norm{f}{\itl{\ppo}{s+m}{q}} \norm{g}{h^{\ppt}} +  \norm{f}{h^{\ppo}}   \norm{g}{\itl{\ppt}{s+m}{q}} \quad \forall f, g \in \sw.
\end{equation*}
Moreover, if $\pp\in \P_0^*,$     $0 < q \le \infty$ and  
$s > n(1/\min(\tau_{\pp},q,1)-1),$  it holds that
\begin{equation*}
\norm{T_\sigma(f,g)}{\itl{\pp}{s}{q}} \lesssim \norm{f}{\itl{\pp}{s+m}{q}} \norm{g}{L^\infty} +  \norm{f}{L^\infty}   \norm{g}{\itl{\pp}{s+m}{q}} \quad \forall f, g \in \sw.
\end{equation*}
\end{theorem}

 \section{Applications to scattering properties of PDEs}

In this section we discuss applications of Theorem \ref{thm:CM:TL:B} to systems of partial differential equations involving powers of the Laplacian. The systems of partial differential equations that we study are on functions $u=u(t,x)$, $v=v(t,x)$, and $w=w(t,x)$, with $t\geq 0$ and $x\in\rn$, are of the form

\begin{equation} \label{eq:a:b:}
\left\{ \begin{array}{lll}  \partial_t u =vw, & \partial_t v +a(D) v = 0, & \partial_t w + b(D) w = 0, \\
  u(0,x)=0,&v(0,x)=f(x),&w(0,x) = g(x),
 \end{array} \right.
\end{equation}

Here the operators $a(D)$ and $b(D)$ are \textit{linear} Fourier multiplier operators with the symbols $a(\xi)$ and $b(\xi)$ respectively; that is, $\widehat{a(D)f}(\xi) = a(\xi)\widehat{f}(\xi)$ and $\widehat{b(D)f}(\xi) = b(\xi)\widehat{f}(\xi)$. Then formally, without taking issues of convergence into account, we get that 
\begin{equation}
v(t,x) = \int_\rn e^{-ta(\xi)} \widehat{f}(\xi)e^{2\pi i x\cdot \xi} d\xi.
\end{equation}
Indeed, using the system (\ref{eq:a:b:}) we obtain
\begin{align*}
\partial_t v +a(D) v &= \int_\rn (\widehat{\partial_t v(t,\cdot)}(\xi) + a(\xi)\widehat{v(t,\cdot)}(\xi))e^{2\pi i \xi \cdot x} d\xi \\
&=0,
\end{align*}
so we must have $\widehat{\partial_t v(t,\cdot)}(\xi) + a(\xi)\widehat{v(t,\cdot)}(\xi) = 0$ where the Fourier transform is taken with the variable $x$. Then by interchanging the Fourier transform with the derivative with respect to $t$ we get $\widehat{v}(t,\cdot)(\xi) = e^{-t a(\xi)} F(\xi)$ for some function $F$. Setting $t=0$ and using system (\ref{eq:a:b:}) it is apparent that $F(\xi) = \widehat{f}(\xi)$ and by inverting the Fourier transform we have $$v(t,x) = \int_\rn e^{-t a(\xi)} \widehat{f}(\xi) e^{2\pi i x\cdot \xi} d\xi$$. A similar calculation shows that $$w(t,x) = \int_\rn e^{-tb(\eta)} \widehat{g}(\eta) e^{2\pi i x\cdot \eta} d\eta.$$
These expressions for $v$ and $w$ yield that
\begin{align*}
u(t,x) & = \int_0^t v(s,x) w(s,x) \,ds  = \int_{\rtn} \left(\int_0^t e^{-s (a(\xi)+b(\eta))} \,ds \right) \widehat{f}(\xi) \widehat{g}(\eta) \, \eixxe \dxi\deta.
\end{align*}
Setting $\lambda(\xi,\eta)=a(\xi)+b(\eta)$ and assuming that $\lambda$ never vanishes, the solution $u(t,x)$ can then be written as the action on $f$ and $g$ of the bilinear multiplier  with symbol $\frac{1-e^{-t\lambda(\xi,\eta)}}{\lambda(\xi,\eta)},$ that is,  
\begin{equation}\label{u:T:lambda}
u(t,x) = T_{\frac{1-e^{-t\lambda}}{\lambda}}(f,g)(x).
\end{equation}
Following Bernicot--Germain~\cite[Section 9.4]{MR2680189}, suppose there exists   $u_\infty\in \swp$ such that 
\begin{equation}\label{def:u:infty}
\lim\limits_{t \to \infty} u(t, \cdot ) = u_\infty \quad \text{in } \swp;
\end{equation}
then, given a function space $X$, we say that the solution $u$ of  \eqref{eq:a:b:} scatters in the function space $X$ if $u_\infty \in X.$

As an application of Theorems \ref{thm:CM:TL:B} and \ref{thm:ICM:TL:B} we obtain the following scattering properties for solutions to  systems of the type \eqref{eq:a:b:} involving powers of the Laplacian. 

  

For $0<p_1,p_2, p, q\le \infty$ and $w_1,w_2\in A_\infty,$ set
\begin{align*}
\gamt&=2( [n(1/\min(p,q,1)+1/\min(1,p_1/\tau_{w_1},p_2/\tau_{w_2},q))]+1),\\
 \gamb&=2( [n(1/\min(p,q,1)+1/\min(1,p_1/\tau_{w_1},p_2/\tau_{w_2}))]+1).
\end{align*}


For $\delta>0$ define
$$
\Ss_{\delta}=\{(\xi,\eta)\in \re^{2n}: \abs{\eta}\le \delta^{-1}\abs{\xi}\text{ and }\abs{\xi}\le \delta^{-1}\abs{\eta}\}.
$$

 
 \begin{theorem}\label{thm:scattering} Consider  $0 < p, p_1, p_2  \le \infty$  such that $\hcline$ and  $0 < q \leq \infty;$ let  $w_1,w_2\in A_\infty$ and set $w=w_1^{{p}/{p_1}} w_2^{{p}/{p_2}}.$ 
Fix $\gamma>0;$ if $\gamma$ is even, or $\gamma\ge \gamt$ in the setting of Triebel--Lizorkin spaces, or $\gamma\ge \gamb$ in the setting of Besov spaces, assume $f, g \in \swz;$ otherwise, assume that $f,g\in\swz$ are such that $\fhat(\xi)\ghat(\eta)$ is supported in $\Ss_{\delta}$ for some $0<\delta\ll1.$ Consider the system 
\begin{equation}\label{eq:Ds:Ds}
\left\{ \begin{array}{lll}  \partial_t u =vw, & \partial_t v +D^\gamma v = 0, & \partial_t w + D^\gamma w = 0, \\
  u(0,x)=0,&v(0,x)=f(x),&w(0,x) = g(x).
 \end{array} \right.
\end{equation}
If $0 < p,p_1,p_2 < \infty$ and  $s > \tau_{p,q}(w),$ the solution $u$ of \eqref{eq:Ds:Ds}  scatters in $\tlw{p}{s}{q}{w}$ to a function $u_\infty$ that satisfies the following estimates: 
\begin{equation}\label{eq:scattering1}
\norm{u_\infty}{\tlw{p}{s}{q}{w}} \lesssim \norm{f}{\tlw{p_1}{s-\gamma}{q}{w_1} } \norm{g}{H^{p_2}(w_2)} +  \norm{f}{H^{p_1}(w_1)}   \norm{g}{\tlw{p_2}{s-\gamma}{q}{w_2} },
\end{equation}
where the implicit constant is independent of $f$ and $g.$
If $0< p, p_1,p_2\leq \infty$ and $s > \tau_p(w)$, the solution $u$ of \eqref{eq:Ds:Ds}  scatters in $\besw{p}{s}{q}{w}$ to a function $u_\infty$ that satisfies the following estimates
\begin{equation}\label{eq:scattering2}
\norm{u_\infty}{\besw{p}{s}{q}{w}} \lesssim \norm{f}{\besw{p_1}{s-\gamma}{q}{w_1} } \norm{g}{H^{p_2}(w_2)} +  \norm{f}{H^{p_1}(w_1)}   \norm{g}{\besw{p_2}{s-\gamma}{q}{w_2} },
\end{equation}
where the Hardy spaces $H^{p_1}(w_1)$ and $H^{p_2}(w_2)$ must be replaced by $L^\infty$ if $p_1=\infty$ or $p_2=\infty,$ respectively, and the implicit constant is independent of $f$ and $g.$ 
If $w_1=w_2$ then different pairs of $p_1, p_2$ can be used on the right-hand sides of \eqref{eq:scattering1} and \eqref{eq:scattering2}; moreover, if $w\in A_\infty,$ then 
\begin{equation*}
\norm{u_\infty}{\tlw{p}{s}{q}{w}} \lesssim \norm{f}{\tlw{p}{s-\gamma}{q}{w} } \norm{g}{L^\infty} +  \norm{f}{L^\infty}   \norm{g}{\tlw{p}{s-\gamma}{q}{w}},
\end{equation*}
where $0<p<\infty,$ $0<q\le\infty,$  $s>\tau_{p,q}(w),$ and the implicit constant is independent of $f$ and $g.$ 
\end{theorem} 


For $\delta >0$ 
\[ \tilde{\Ss}_{\delta}=\{(\xi,\eta)\in \re^{2n}: \abs{\eta}\le \delta^{-1}(1+\abs{\xi}^2)^\frac{1}{2} \text{ and }\abs{\xi}\le \delta^{-1}(1+\abs{\eta}^2)^\frac{1}{2}\}.\]



\begin{theorem}\label{thm:scattering2} Consider  $0 < p, p_1, p_2  \le \infty$  such that $\hcline$ and  $0 < q \leq \infty;$ let  $w_1,w_2\in A_\infty$ and set $w=w_1^{{p}/{p_1}} w_2^{{p}/{p_2}}.$ Fix $\gamma>0;$ if $\gamma$ is even, or $\gamma\ge \gamt$ in the setting of Triebel--Lizorkin spaces, or $\gamma\ge \gamb$ in the setting of Besov spaces, assume $f, g \in \sw;$ otherwise, assume that $f,g\in\sw$ are such that $\fhat(\xi)\ghat(\eta)$ is supported in $\widetilde{\Ss}_{\delta}$ for some $0<\delta\ll1.$ Consider the system 
\begin{equation}\label{eq:Js:Js}
\left\{ \begin{array}{lll}  \partial_t u =vw, & \partial_t v +J^\gamma v = 0, & \partial_t w + J^\gamma w = 0, \\
  u(0,x)=0,&v(0,x)=f(x),&w(0,x) = g(x).
 \end{array} \right.
\end{equation}
If $0 < p,p_1,p_2 < \infty$ and  $s > \tau_{p,q}(w),$ the solution $u$ of \eqref{eq:Js:Js}  scatters in $\itlw{p}{s}{q}{w}$ to a function $u_\infty$ that satisfies the following estimates: 
\begin{equation}\label{eq:scattering21}
\norm{u_\infty}{\itlw{p}{s}{q}{w}} \lesssim \norm{f}{\itlw{p_1}{s-\gamma}{q}{w_1} } \norm{g}{h^{p_2}(w_2)} +  \norm{f}{h^{p_1}(w_1)}   \norm{g}{\itlw{p_2}{s-\gamma}{q}{w_2} },
\end{equation}
where the implicit constant is independent of $f$ and $g.$  
If $0< p, p_1,p_2\leq \infty$ and $s > \tau_p(w)$, the solution $u$ of \eqref{eq:Js:Js}  scatters in $\ibesw{p}{s}{q}{w}$ to a function $u_\infty$ that satisfies the following estimates
\begin{equation}\label{eq:scattering22}
\norm{u_\infty}{\ibesw{p}{s}{q}{w}} \lesssim \norm{f}{\ibesw{p_1}{s-\gamma}{q}{w_1} } \norm{g}{h^{p_2}(w_2)} +  \norm{f}{h^{p_1}(w_1)}   \norm{g}{\ibesw{p_2}{s-\gamma}{q}{w_2} },
\end{equation}
where the Hardy spaces $h^{p_1}(w_1)$ and $h^{p_2}(w_2)$ must be replaced by $L^\infty$ if $p_1=\infty$ or $p_2=\infty,$ respectively, and the implicit constant is independent of $f$ and $g.$  
If $w_1=w_2$ then different pairs of $p_1, p_2$ can be used on the right-hand sides of \eqref{eq:scattering21} and \eqref{eq:scattering22}; moreover, if $w\in A_\infty,$ then 
\begin{equation*}
\norm{u_\infty}{\itlw{p}{s}{q}{w}} \lesssim \norm{f}{\itlw{p}{s-\gamma}{q}{w} } \norm{g}{L^\infty} +  \norm{f}{L^\infty}   \norm{g}{\itlw{p}{s-\gamma}{q}{w}},
\end{equation*}
where $0<p<\infty,$ $0<q\le\infty,$  $s>\tau_{p,q}(w),$ and the implicit constant is independent of $f$ and $g.$  
\end{theorem}  
 
 
 \begin{proof}[Proof of Theorem~\ref{thm:scattering}] We have $a(\xi)=\abs{\xi}^\gamma$ and $b(\eta)=\abs{\eta}^\gamma;$ therefore, $\lambda(\xi,\eta)=\abs{\xi}^\gamma+\abs{\eta}^\gamma.$ Note that all corresponding integrals for $v(t,x),$ $w(t,x)$ and $u(t,x)$ are absolutely convergent for $t>0,$ $x\in\rn$ and $f,g\in \sw.$ If we further assume that $f,g\in \swz,$ the Dominated Convergence Theorem implies that $u(t,\cdot)\to u_\infty$  both pointwise and in $\swp,$ where
$$
u_\infty(x)=\int_{\re^{2n}} (a(\xi)+b(\eta))^{-1}\fhat(\xi)\ghat(\eta)\eixxe\dxi\deta=T_{\lambda^{-1}} (f,g)(x).
$$
Indeed, by using the Taylor expansion of $\widehat{f}$ and $\widehat{g}$ we obtain
\[\widehat{f}(\xi) = 
\sum_{|\alpha|\leq 
\left\lfloor\gamma\right\rfloor} 
\frac{\partial^\alpha \widehat{f}(0)}{\alpha !}\xi^\alpha
 + \sum_{|\alpha| = 
 \left\lfloor\gamma\right\rfloor + 1} \frac{\partial^\alpha \widehat{f}(c_1 \xi)}{\alpha !}\xi^\alpha \]
and 
\[\widehat{g}(\eta) = \sum_{\abs{\alpha}\leq \left\lfloor\gamma\right\rfloor} \frac{\partial^\alpha \widehat{g}(0)}{\alpha !}\eta^\alpha + \sum_{\abs{\alpha} = \left\lfloor\gamma\right\rfloor + 1} \frac{\partial^\alpha \widehat{g}(c_2 \eta)}{\alpha !}\eta^\alpha \]
for some $0<c_1 , c_2 <1$. Then by using that $f,g \in \swz$ we have $partial^\alpha \widehat{f}(0) = 0$ and $partial^\alpha \widehat{g}(0) = 0$ for all $\alpha \in \mathbb{N}_0$. From this we obtain 
\begin{align*}
\abs{\frac{1-e^{-t(\abs{\xi}^\gamma + \abs{\eta}^\gamma)}}{\abs{\xi}^\gamma + \abs{\eta}^\gamma}\widehat{f}(\xi)\widehat{g}(\eta) e^{2\pi i x\cdot (\xi + \eta)}} &\leq \sum_{\abs{\alpha} = \left\lfloor\gamma\right\rfloor + 1} \abs{\frac{1}{\abs{\xi}^\gamma} \frac{\partial^\alpha \widehat{f}(c_1 \xi)}{\alpha !}\xi^\alpha\frac{\partial^\alpha \widehat{g}(c_2 \eta)}{\alpha !}\eta^\alpha}, 
\end{align*}
and the right-hand side is integrable in $\xi$ and $\eta$. So by applying the Dominated Convergence Theorem and letting $t \rightarrow \infty$ we get that $u(t,\cdot)$ converges to $u_\infty$ both pointwise and in $\mathcal{S}'$. 

Next we note that $\lambda^{-1}$ is homogeneous of degree $-\gamma$ so $\partial^\alpha_\xi \partial^\beta_\eta \lambda^{-1}$ is homogeneous of degree $-\gamma -|\alpha + \beta|$. That is $\partial^\alpha_\xi \partial^\beta_\eta \lambda^{-1} (r\xi, r\eta) = r^{-\gamma -|\alpha+\beta|}\partial^\alpha_\xi \partial^\beta_\eta \gamma^{-1}(\xi,\eta)$ for any $r>0$. By letting $r = (\abs{\xi} + \abs{\eta})^{-1}$ we obtain 
$$\partial^\alpha_\xi \partial^\beta_\eta \gamma^{-1}(\xi,\eta) = (\abs{\xi} + \abs{\eta})^{-\gamma -|\alpha+\beta|} \partial^\alpha_\xi \partial^\beta_\eta \lambda^{-1} (\frac{\xi}{(\abs{\xi} + \abs{\eta})^{-1}}, \frac{\eta}{(\abs{\xi} + \abs{\eta})^{-1}}).$$
We now want to show that $\lambda^{-1}$ is a Coifman-Meyer multiplier; that is $\lambda^{-1}$ satisfies \eqref{eq:CMm}.

If $\gamma$ is an even positive integer then $\lambda^{-1}$ satisfies the estimates \eqref{eq:CMm} with $m=-\gamma$ for all $\alpha,\beta\in\na_0^n.$ Then,  all estimates from Theorem~\ref{thm:CM:TL:B} hold for $T_{\lambda^{-1}}$  and therefore the desired estimates follow for $u_\infty$ with constants independent of $f,g\in\swz.$ 

Let $p_1,p_2,p,q, w_1,w_2$ be as in the hypotheses.  If $\gamma>0$ and $\gamma$ is not an even integer, then $\lambda^{-1}$ satisfies the estimates \eqref{eq:CMm} with $m=-\gamma$ as long as  $\alpha,\beta\in\na_0^n$ are such that $\abs{\alpha}<\gamma$ and $\abs{\beta}<\gamma;$ in particular, $\lambda^{-1}$ satisfies \eqref{eq:CMm} with $m=-\gamma$ for $\alpha,\beta\in \na_0^n$ such that $\abs{\alpha+\beta}<\gamma.$
In view of Remark~\ref{re:numderiv2}, all estimates from Theorem~\ref{thm:CM:TL:B} hold for $T_{\lambda^{-1}}$ if $\gamma\ge \gamt$ in the context of Triebel--Lizorkin spaces and if $\gamma\ge \gamb$ in the context of Besov spaces; as a consequence, the desired estimates follow for $u_\infty$ with constants independent of $f,g\in\swz$ for such values of $\gamma.$


On the other hand, if $0<\gamma<\gamt$ in the Triebel-Lizorkin space setting or $0<\gamma<\gamb$ in the Besov space setting, and $\gamma$ is not an even positive integer, consider  $h\in\Ss(\re^{2n}) $ such that $\supp(h)\subset \Ss_{\delta/2}$ and $h\equiv 1$ on $\Ss_{\delta}.$ Then, for $f,g\in \swz$ such that $\fhat(\xi)\ghat(\eta)$ is supported in $\Ss_{\delta}$ we have $h(\xi,\eta)\fhat(\xi)\ghat(\eta)=\fhat(\xi)\ghat(\eta);$   therefore, $T_{\lambda^{-1}}(f,g)=T_{\Lambda}(f,g),$ where
$\Lambda(\xi,\eta)=h(\xi,\eta)/(\abs{\xi}^\gamma+\abs{\eta}^\gamma).$ The multiplier $\Lambda$ verifies  \eqref{eq:CMm} with $m=-\gamma$  for all $\alpha,\beta\in\na_0^n$ (with constants that depend on $\delta$). Then all estimates from Theorem~\ref{thm:CM:TL:B} hold for $T_\Lambda$ and therefore the desired estimates follow for $u_\infty$ with constants dependent on $\delta$ and independent of $f,g\in\swz$ such that  $\fhat(\xi)\ghat(\eta)$ is supported in $\Ss_{\delta}.$
\end{proof}

\begin{proof}[Proof of Theorem~\ref{thm:scattering2}] We proceed as in the proof of Theorem~\ref{thm:scattering} with $\lambda(\xi,\eta)=(1+\abs{\xi}^2)^{\gamma /2}+(1+\abs{\eta}^2)^{\gamma /2}$ and an application of  Theorem~\ref{thm:ICM:TL:B}.  
\end{proof}  