\cleardoublepage

\chapter{Weighted Leibniz-type rules with applications to scattering properties of PDEs}
\label{makereference2}

In this chapter we discuss bilinear multiplier operators associated to Coifman-Meyer multipliers and Leibniz-type rules in the settings of weighted Triebel-Lizorkin and Besov spaces.  Additionally we obtain applications of these results to scattering properties of certain systems of parial differential equations. One of the main results in this chapter is in the setting of Triebel-Lizorkin spaces based in weighted Lebesgue spacs and Hardy spaces. It is as follows. 

\begin{theorem}\label{thm:CM:TL:B}  For $m \in \re,$ let $\sigma(\xi,\eta),$ $\xi,\eta\in\rn,$ be a Coifman-Meyer multiplier of order $m.$ Consider  $0 < p, p_1, p_2  \le \infty$  such that $\hcline$ and  $0 < q \leq \infty;$ let  $w_1,w_2\in A_\infty$ and set $w=w_1^{{p}/{p_1}} w_2^{{p}/{p_2}}.$ 
If $0 < p,p_1,p_2 < \infty$ and  $s > \tau_{p,q}(w),$  it holds that
\begin{equation}\label{KP:CM:TL}
\norm{T_\sigma(f,g)}{\tlw{p}{s}{q}{w}} \lesssim \norm{f}{\tlw{p_1}{s+m}{q}{w_1} } \norm{g}{H^{p_2}(w_2)} +  \norm{f}{H^{p_1}(w_1)}   \norm{g}{\tlw{p_2}{s+m}{q}{w_2} } \quad \forall f, g \in \swz.
\end{equation}
If $0< p, p_1,p_2\leq \infty$ and $s > \tau_p(w)$, it holds that
\begin{equation}\label{KP:CM:B}
\norm{T_\sigma(f,g)}{\besw{p}{s}{q}{w}} \lesssim \norm{f}{\besw{p_1}{s+m}{q}{w_1} } \norm{g}{H^{p_2}(w_2)} +  \norm{f}{H^{p_1}(w_1)}   \norm{g}{\besw{p_2}{s+m}{q}{w_2} } \quad \forall f, g \in \swz,
\end{equation}
where the Hardy spaces $H^{p_1}(w_1)$ and $H^{p_2}(w_2)$ must be replaced by $L^\infty$ if $p_1=\infty$ or $p_2=\infty,$ respectively.

If $w_1=w_2$ then different pairs of $p_1, p_2$ can be used on the right-hand sides of \eqref{KP:CM:TL} and \eqref{KP:CM:B}; moreover, if $w\in A_\infty,$ then 
\begin{equation}\label{KP:CM:TL2}
\norm{T_\sigma(f,g)}{\tlw{p}{s}{q}{w}} \lesssim \norm{f}{\tlw{p}{s+m}{q}{w} } \norm{g}{L^\infty} +  \norm{f}{L^\infty}   \norm{g}{\tlw{p}{s+m}{q}{w}} \quad \forall f, g \in \swz,
\end{equation}
where $0<p<\infty,$ $0<q\le\infty$ and $s>\tau_{p,q}(w).$
\end{theorem}

In the following sections the function spaces and multipliers used in the hypotheses are defined and discussed. We will then remark on corollaries to Theorem (\ref{thm:CM:TL:B}) and their connection to the Leibniz rules in the previous chapter.  theorem and the proof of Theorem (\ref{thm:CM:TL:B}). The proof is quite flexible and can be readily adapted to Triebel-Lizorkin and Besov spaces based in other function spaces. 

\section{Definitions}

\subsection{Coifman-Meyer Multipliers}

The symbols used in Theorem (\ref{thm:CM:TL:B}) and results later in this chapter are Coifman-Meyer multipliers. Such multipliers are defined as follows.

\begin{dfn}\label{CM_def}
For $m\in\rn$, a smooth function $\sigma = \sigma(\xi,\eta)$, $\xi,\eta\in\rn$, is a \textit{Coifman-Meyer multiplier} of order $m$ if for all multi-indices $\alpha,\beta\in\mathbb{N}^n_0$ there exists a positive constant $C_{\alpha,\beta}$ such that 
\begin{equation}\label{eq:CMm}
|\partial_\xi^\alpha \partial_\eta^\beta \sigma(\xi, \eta)| \leq C_{\alpha, \beta} (|\xi|+|\eta|)^{m -(\abs{\alpha}+ \abs{\beta})} \quad \forall (\xi, \eta) \in \re^{2n} \setminus \{(0,0)\}.
\end{equation}
\end{dfn}
Operators associated to these multipliers have been widely studied. For instance in Grafakos-Torres \cite{MR1880324} operators associated to Coifman-Meyer multipliers were studied because of their connection to a larger class of operators called Calder\'on-Zygmund operators. In particular it holds that 
\[ \norm{T_\sigma(f,g)}{L^p}\lesssim \norm{f}{L^{p_1}}\norm{g}{L^{p_2}} \]
where $\sigma$ is a Coifman-Meyer multiplier, $\frac{1}{p} = \frac{1}{p_1} + \frac{1}{p_2}$, and $1<p_1,p_2<\infty$. 
We also note that in particular Coifman-Meyer multipliers of order $m$ belong to the bilinear H\"ormander class $\dot{BS}^m_{1,1}$. These symbols and operators associated to them will be discussed in the following chapter. 

\subsection{Function spaces}

\subsubsection{Weighted spaces}

\begin{dfn}
A \textit{weight} $w(x)$ defined on $\rn$ is a nonnegative, locally integrable function such that $0<w(x)<\infty$ for almost every $x\in\rn$.
\end{dfn}

Given a weight $w(x)$ and $0<p<\infty$ we define the weighted Lebesgue space $L^p(w)$ as the space of all measurable functions satisfying 
\[ \norm{f}{L^p(w)} = \left(\int_\rn |f(x)|^p w(x)dx\right)^{\frac{1}{p}} < \infty. \]
In the case that $p=\infty$ we define $L^\infty (w) = L^\infty$. 

The specific classes of weights in the hypotheses of our results are Muckenhoupt weights. 
\begin{dfn}
For $1<p<\infty$ the \textit{Muckenhoupt class} $A_p$ consists of all weights $w$ on $\rn$ satisfying 
 \begin{equation}\label{weight_condition}}
 \sup_B\left(\frac{1}{\abs{B}}\int_Bw(x)\dx\right)\left(\frac{1}{\abs{B}}\int_Bw(x)^{-\frac{1}{p-1}}\dx\right)^{p-1}<\infty,
 \end{equation}
where the supremum is taken over all Euclidean balls $B\subset\rn$ and $|B|$ is the Lebesgue measure of $B$. For $p=\infty$ we define $A_\infty = \cup_{1<p} A_p$. 
\end{dfn}

From this definition it follows that $A_p \subset A_q$ when $p\leq q$. For a $w\in A_\infty$ we set $\tau_w = \inf\{\tau \in (1,\infty]: w\in A_\tau \]. The condition (\ref{weight_condition}) is motivated by the following fact: for $f\in L^p(w)$ the Hardy-Littlewood maximal function $\mathcal{M}(f)(x)$ is bounded on $L^p(w)$ if and only if $w\in A_p$. That is for for $1<p<\infty$ $w\in A_p$ if and only if 
\[\norm{\mathcal{M}(f)}{L^p (w)} \lesssim \norm{f}{L^p (w)}. \]
Later in this chapter we will use the maximal function $\mathcal{M}_r (f) = \left(\mathcal{M}(|f|^r)\right)^{\frac{1}{r}}$. By the properties for the Hardy-Littlewood maximal function above it holds that for $0<r<p$ $\mathcal{M}_r$ is bounded on $L^p (w)$ for $w\in A_{p/r}$ and in this case $0<r<\frac{p}{\tau_w}$. The following theorem is a vector valued version of the previous statement called a weighted Fefferman-Stein inequality.

\begin{theorem}
If $0<p<\infty,$ $0<q\le \infty,$  $0<r <\min(p,q)$ and $w \in A_{p/r}$ (i.e. $0<r<\min(p/\tau_w,q)$), then for all sequences $\{f_{j}\}_{j\in\ent}$ of locally integrable functions defined on $\rn,$ we have
 \begin{equation*}
 \norm{\left(\sum_{j\in\ent}\abs{\M_r (f_j)}^q\right)^{\frac{1}{q}}}{\lebw{p}{w}}\lesssim
 \norm{\left(\sum_{j\in\ent}\abs{f_j}^q\right)^{\frac{1}{q}}}{\lebw{p}{w}},\label{eq:wFS}
 \end{equation*}
where the implicit constant depends on $r,$ $p,$ $q,$ and $w$ and the summation in $j$ should be replaced by the supremum in $j$ if $q=\infty.$
\end{theorem}
For more detail on the Muckenhoupt classes see Grafakos \cite{MR3243741}.

\subsubsection{Triebel-Lizorkin and Besov spaces}
Here we describe the function spaces in which (\ref{thm:CM:TL:B}) is bases and some properties of such spaces. 

Let $\psi$ and $\varphi$ be functions in $\sw(\rn)$ satisfying the following conditions:
\begin{itemize}
\item supp$(\widehat{\psi})\subset\{\xi\in\rn : \frac{1}{2} < |\xi| <\2\}$,
\item $|\widehat{\psi}(\xi)|>c$ for all $\xi$ such that $\frac{3}{5} < |\xi| < \frac{5}{3}$ for some $c>0$,
\item supp$(\widehat{\varphi}\subset \{\xi\in\rn : |\xi| < 2\}$,
\item $|\widehat{\varphi}(\xi)|>c$. 
\end{itemize}
For $\psi$ supported in an annulus and $j\in\ent$ we define the operator $\Delta^\psi_j(f)$ through its Fourier transform as \[\widehat{\Delta^\psi_j (f)}(\xi) = \psi(2^{-j}\xi)\widehat{f}(\xi)\]
For such $\psi$ we define the homogeneous Triebel-Lizorkin and Besov spaces as follows.

\begin{dfn}\label{TL_B_def}
Let $s\in\mathbb{R}$, $0<p<\infty$, and $0<q\leq\infty$.
\begin{itemize}
\item The weighted \textit{homogeneous Triebel-Lizorkin space} $\tlw{p}{s}{q}{w}$ consists of all $f\in \swp/\mathcal{P}(\rn)$ such that 
\begin{equation*}
\norm{f}{\tlw{p}{s}{q}{w}}=\norm{\left(\sum_{j\in\ent}(2^{sj}|\Delta^\psi_jf|)^q\right)^{\frac{1}{q}}}{\lebw{p}{w}}<\infty.
\end{equation*}
\item The weighted \textit{homogeneous Besov space} $\besw{p}{s}{q}{w}$ consists of all $f\in \swp/\mathcal{P}(\rn)$ such that 
\begin{equation}
\norm{f}{\besw{p}{s}{q}{w}} = \left(\sum_{j\in\int} (2^{js}\norm{\Delta^\psi_jf}{L^p(w)})^q\right)^{\frac{1}{q}} < \infty.
\end{equation}
\end{itemize}
\end{dfn}

Given $\varphi$, $\psi$, $S^\varphi_0$, and $\Delta^\psi_j$ as above the weighted inhomogeneous Triebel-Lizorkin and Besov spaces are defined as follows. 

\begin{dfn}
Let $s\in\mathbb{R}$, $0<p<\infty$, and $0<q\leq\infty$.
\begin{itemize}
\item The weighted \textit{inhomogeneous Triebel-Lizorkin space} $\itlw{p}{s}{q}{w}$ is the class of all $f\in\sw'$ such that
\begin{equation*}
\norm{f}{\itlw{p}{s}{q}{w}}= \norm{S^\varphi_0 f}{L^p(w)} + \norm{\left(\sum_{j\in\naz}(2^{sj}|\Delta^\psi_jf|)^q\right)^{\frac{1}{q}}}{\lebw{p}{w}}<\infty.
\end{equation*}
\item The weighted \textit{inhomogeneous Besov space} $\besw{p}{s}{q}{w}$ is the class of all $f\in\sw'$ such that 
\begin{equation}
\norm{f}{\ibesw{p}{s}{q}{w}} = \norm{S^\varphi_0 f}{L^p(w)} + \left(\sum_{j\in\naz} (2^{js}\norm{\Delta^\psi_jf}{L^p(w)})^q\right)^{\frac{1}{q}} < \infty.
\end{equation}
\end{itemize}
\end{dfn}

All of the definitions above are independent of the choice of $\varphi$ and $\psi$. These spaces are generally quasi-Banach spaces and if $1\leq p,q <\infty$ they are Banach spaces. 
